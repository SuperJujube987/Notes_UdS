\documentclass{report}
\usepackage[letterpaper,portrait,margin=2cm]{geometry}
\usepackage{titlesec}
\usepackage{amsmath,amssymb,amsthm}
\usepackage{tabularx}
\usepackage{enumitem}
\usepackage{indentfirst}
\usepackage{dsfont}
\usepackage{tikz}
\usepackage{hyperref}
\usepackage{mathtools}
\usepackage{mathrsfs}
\usepackage{enumitem}
\usepackage{xfrac}
\usepackage{fancyhdr}
\usepackage[french]{babel}

\hypersetup{
	colorlinks=false
}

\fancyfoot{}
\fancyhead{}
\fancyfoot[c]{\thepage}
\fancyhead[l]{\leftmark}

\setcounter{tocdepth}{3}

\pagestyle{fancy}

\title{MAT346 - Analyse II

Donn\'e par Mario Lambert}
\author{Julien Houle}
\date{Automne 2025}

\NewDocumentCommand{\bornee}{O{a,b}}{\mathcal{B}\left[ #1 \right]}
\NewDocumentCommand{\continue}{O{a,b}}{\mathcal{C}\left[ #1 \right]}
\NewDocumentCommand{\riemann}{O{a,b}}{\mathcal{R}\left[ #1 \right]}
\NewDocumentCommand{\partitions}{O{a,b}}{\Omega\left[ #1 \right]}
\NewDocumentCommand{\msup}{mO{a,b}}{\overline{M}\left( #1,\left[ #2 \right] \right)}
\NewDocumentCommand{\minf}{mO{a,b}}{\underline{M}\left( #1,\left[ #2 \right] \right)}
\NewDocumentCommand{\ssup}{mO{\Delta}}{\overline{S}\left( #1,#2 \right)}
\NewDocumentCommand{\sinf}{mO{\Delta}}{\underline{S}\left( #1,#2 \right)}
\NewDocumentCommand{\srie}{mO{\Delta}O{\{\bar{x}_i\}}}{S\left( #1,#2,#3 \right)}

\newcommand*{\Ssup}[1]{\overline{S}(#1)}
\newcommand*{\Sinf}[1]{\underline{S}(#1)}

\newcommand*{\raffinement}[2]{#1 \vee #2}
\newcommand*{\norme}[1]{\left\| #1 \right\|}
\newcommand*{\abs}[1]{\left| #1 \right|}

\newcommand*{\eps}{\varepsilon}

\newcommand*{\reels}{\mathbb{R}}
\newcommand*{\entiers}{\mathbb{Z}}
\newcommand*{\rationels}{\mathbb{Q}}
\newcommand*{\naturels}{\mathbb{N}}
\newcommand*{\complexes}{\mathbb{C}}

\newcommand*{\sech}{\mathrm{sech}}

\renewcommand{\thesection}{ \arabic{chapter}.\arabic{section}}
\renewcommand{\thesubsection}{}
\renewcommand{\thesubsubsection}{}

\titleformat{\chapter}[hang]{\bfseries\huge\centering}{Chapitre \arabic{chapter}}{1em}{}[]
\titleformat{\section}[hang]{\bfseries\large}{Section\thesection}{1em}{}[]
\titleformat{\subsection}[hang]{\bfseries\normalsize}{}{0pt}{}[]
\titleformat{\subsubsection}[hang]{\slshape\normalsize}{}{0pt}{}[]

\newtheorem*{thm}{Th\'eor\`eme}
\newtheorem*{lem}{Lemme}
\newtheorem*{prop}{Proposition}
\newtheorem*{coro}{Corollaire}
\theoremstyle{definition}
\newtheorem*{defin}{D\'efinition}
\theoremstyle{remark}
\newtheorem*{exem}{Exemple}
\newtheorem*{exer}{Exercice}
\newtheorem*{nota}{Notation}
\newtheorem*{rema}{Remarque}
\newtheorem*{rapp}{Rappel}

\begin{document}
	\maketitle
	\tableofcontents
	\pagenumbering{roman}
	\newpage
	\pagenumbering{arabic}

	\chapter{Int\'egration}
	\section{Int\'egrales de Riemann}

	\begin{nota}
		~

		$\bornee[c,d]=\left\lbrace f:[a,b] \to \reels \middle| f \text{ est born\'ee} \right\rbrace$.

		$\riemann=\left\lbrace f:[a,b] \to \reels \middle| f \text{ est born\'ee et int\'egrable} \right\rbrace$.

		$\continue=\left\lbrace f:[a,b] \to \reels \middle| f \text{ est born\'ee et continue} \right\rbrace$.
	\end{nota}

	On suppose nos fonctions born\'ees.

	\begin{defin}
		~

		\begin{enumerate}[label=\alph*)]
			\item Une partition de $[a,b]$ est un ensemble fini de points $\Delta=\{x_0,x_1,\dotsc,x_n\} \subseteq [a,b]$ t.q. $a=x_0 < x_1 < x_2 < \dotsc < x_{n-1} < x_n=b$.
			\item L'ensemble des partitions de $[a,b]$ est $\partitions$.
			\item On dit $\Delta'$ est \emph{plus fine} que $\Delta$, not\'e $\Delta' \geq \Delta$, si $\Delta' \supseteq \Delta$.
			\item \emph{Raffinement commun} de $\Delta_1$ et $\Delta_2$, not\'e $\raffinement{\Delta_1}{\Delta_2}$, est la partition de $[a,b]$ form\'ee de $\Delta_1 \cup \Delta_2$ ordonn\'es.
			\item La \emph{norme} de $\Delta$, not\'ee $\norme{\Delta}$, est $\norme{\Delta}=\displaystyle\max\limits_{i=1}^{n}\abs{x_i-x_{i-1}}$.
			\item
			\begin{align*}
				\msup{f}[x_{i-1},x_1]&= \displaystyle\sup\limits_{x \in \left[ x_{i-1},x_1 \right]}f(x)\\
				\minf{f}[x_{i-1},x_1]&= \displaystyle\inf\limits_{x \in \left[ x_{i-1},x_1 \right]}f(x)
			\end{align*}
		\end{enumerate}
		\begin{rema}
			\begin{align*}
				\norme{x}&\geq0\\
				\norme{\lambda x}&=\abs{\lambda} \norme{x}\\
				\norme{x+y}&= \norme{x} + \norme{y}
			\end{align*}
		\end{rema}
	\end{defin}

	\begin{defin}
		~

		\begin{enumerate}[label=\alph*)]
			\item La \emph{somme de Riemann par exc\`es} (ou sup\'erieure) de $f$ pour la partition $\Delta$ est
			\[
			\ssup{f}= \displaystyle\sum_{i=1}^{n}\msup{f}[x_{i-1},x_i] \cdot (x_i-x_{i-1})
			\]
			\item La \emph{somme de Riemann par d\'efaut} (ou inf\'erieure) de $f$ pour la partition $\Delta$ est
			\[
			\sinf{f}= \displaystyle\sum_{i=1}^{n}\minf{f}[x_{i-1},x_i] \cdot (x_i-x_{i-1})
			\]
		\end{enumerate}
	\end{defin}

	\newpage
	\begin{prop}
		~

		\begin{enumerate}[label=\alph*)]
			\item
			\[
			\minf{f} \cdot (b-a) \leq \sinf{f}, \forall\Delta\in\partitions
			\]
			\item
			\[
			\sinf{f} \leq \ssup{f}
			\]
			\item
			\[
			\ssup{f} \leq \msup{f} \cdot (b-a)
			\]
		\end{enumerate}
	\end{prop}

	\begin{prop}
		Si $\Delta'\geq\Delta$, alors $\ssup{f}[\Delta']\leq\ssup{f}$.
		\begin{proof}[d\'emonstration]~

			Sans perte de g\'en\'eralit\'e, supposons
			\[
			\Delta:a=x_0<x_1<\dotsc<x_{i-1}<x_i<\dotsc<x_n=b
			\]
			\[
			\Delta':a=x_0<x_1<\dotsc<x_{i-1}<\bar{x}<x_i<\dotsc<x_n=b
			\]

			On a
			\begin{align*}
				\ssup{f}-\ssup{f}[\Delta']&= \left[ \msup{f}[x_{i-1},x_i] \cdot (x_1-x_{i-1}) \right]\\
				&- \left[ \msup{f}[x_{i-1},\bar{x}] \cdot (\bar{x}-x_{i-1}) + \msup{f}[\bar{x},x_i] \cdot (x_i-\bar{x}) \right]\\
				&= (x_i-\bar{x}) \left[ \msup{f}[x_{i-1},x_i] - \msup{f}[\bar{x},x_i] \right]\\
				&+ (\bar{x}-x_{i-1}) \left[ \msup{f}[x_{i-1},x_i] - \msup{f}[x_{i-1},\bar{x}] \right]\\
				&\geq 0
			\end{align*}
		\end{proof}
	\end{prop}

	\begin{prop}
		Si $\Delta'\geq\Delta$, alors $\sinf{f}[\Delta'] \geq \sinf{f}$
		\begin{proof}[d\'emonstration]~

			\begin{center}
				\begin{tikzpicture}
					\draw[->] (0,-.1) -- (0,2) node[above] {$y$};
					\draw[->] (-.1,0) -- (6.1,0) node[right] {$x$};
					\node[anchor=north] (x0) at (1,0) {$x_0$};
					\node[anchor=north] (xi-1) at (2,0) {$x_{i-1}$};
					\node[anchor=north] (x) at (3,0) {$\bar{x}$};
					\node[anchor=north] (xi) at (4,0) {$x_i$};
					\node[anchor=north] (xn) at (5,0) {$x_n$};
					\draw plot[domain=0:6] (\x,.3*\x) node[above left] {$f$};
					\node (y0) at (1,.3) {};
					\node (yi-1) at (2,.6) {};
					\node (y) at (3,.9) {};
					\node (yi) at (4,1.2) {};
					\node (yn) at (5,1.5) {};
					\filldraw[fill=red, fill opacity=.3] (yi-1) rectangle (xi.north);
					\filldraw[fill=blue, fill opacity=.3] (yi-1) rectangle (x.north);
					\filldraw[fill=blue, fill opacity=.3] (y) rectangle (xi.north);
				\end{tikzpicture}
			\end{center}
		\end{proof}
		\begin{rema}
			$\sinf{f}=-\ssup{-f}$.
		\end{rema}
		\begin{coro}
			$\forall \Delta_1,\Delta_2 \in \partitions, \sinf{f}[\Delta_1] \leq \ssup{f}[\Delta_2]$
			\begin{proof}[d\'emonstration]~

				On a $\raffinement{\Delta_1}{\Delta_2} \geq \Delta_1$. Ainsi,
				\begin{align*}
					\sinf{f}[\Delta_1]&\leq \sinf{f}[\raffinement{\Delta_1}{\Delta_2}]\\
					&\leq \ssup{f}[\raffinement{\Delta_1}{\Delta_2}]\\
					&\leq \ssup{f}[\Delta_2]
				\end{align*}
			\end{proof}
		\end{coro}
	\end{prop}

	\newpage
	\begin{defin}
		~

		\begin{enumerate}[label=\alph*)]
			\item La somme par d\'efaut de $f$ est $\Sinf{f}=\displaystyle\sup\limits_{\Delta\in\partitions}\sinf{f}$.
			\item La somme par exc\`es de $f$ est $\Ssup{f}=\displaystyle\inf\limits_{\Delta\in\partitions}\ssup{f}$.
		\end{enumerate}
	\end{defin}

	\begin{thm}
		$\Sinf{f} \leq \Ssup{f}$
		\begin{proof}[d\'emonstration]~

			Soit $\Delta_1 \in \partitions$

			$\Sinf{f}=\sup\sinf{f}$ est le plus petit majorant des $\sinf{f}$ avec $\Delta \in \partitions$.

			Du corollaire pr\'ec\'edant, on a que $\sinf{f} \leq \ssup{f}[\Delta_1]$.

			Donc, $\ssup{f}[\Delta_1]$ est un majorant des $\sinf{f}$.

			Ainsi, $\Sinf{f} \leq \ssup{f}[\Delta_1]$.

			De m\^eme, $\Ssup{f}=\inf\ssup{f}$ est le plus grand minorant des $\ssup{f}$ avec $\Delta \in \partitions$.

			Comme $\Sinf{f}$ est un minorant des $\ssup{f}$, on a que $\Sinf{f} \leq \Ssup{f}$.
		\end{proof}
	\end{thm}

	\begin{defin}
		~

		Soit $f \in \bornee$. On dit que $f$ est \emph{int\'egrable au sens de Riemann sur $[a,b]$} si $\Sinf{f}=\Ssup{f}$ et on note $f \in \riemann$. La valeur commune de $\Sinf{f}$ et $\Ssup{f}$ est not\'ee $\int_{a}^{b}f(x)\ dx$
	\end{defin}

	\subsection{Crit\`ere d'int\'egrabilit\'e}
	\begin{thm}[Crit\`ere d'int\'egrabilit\'e]
		~

		Soit $f \in \bornee$. Alors $f \in \riemann$ si, et seulement si, $\left( \forall \eps >0 \right) \left( \exists\Delta = \Delta(\eps) \in \partitions \right)$ t.q. $\ssup{f}-\sinf{f} < \eps$.
		\begin{proof}[d\'emonstration]~

			\begin{itemize}
				\item[$(\Rightarrow)$] Supposons $f \in \riemann$.

				Soit $\eps>0$.

				On a $\displaystyle\int_{a}^{b}f=\Ssup{f}=\inf\ssup{f}$.

				Comme $\Ssup{f}+\dfrac{\eps}{2}$ ne peut minorer $\ssup{f}$, alors $\exists\Delta_1 \in \partitions$ t.q. $\ssup{f}[\Delta_1] < \Ssup{f}+\dfrac{\eps}{2}$.

				De m\^eme, $\displaystyle\int_{a}^{b}f=\Sinf{f}=\sup\sinf{f}$.

				Comme $\Sinf{f}-\dfrac{\eps}{2}$ ne peut majorer $\sinf{f}$, alors $\exists\Delta_2 \in \partitions$ t.q. $\sinf{f}[\Delta_2] > \Sinf{f}-\dfrac{\eps}{2}$.

				Posons $\Delta = \Delta(\eps) = \raffinement{\Delta_1}{\Delta_2}$.

				On a
				\begin{align*}
					\ssup{f} - \sinf{f}&\leq \ssup{f}[\Delta_1] - \sinf{f}[\Delta_2]\\
					&< \Ssup{f} + \dfrac{\eps}{2} - \left( \Sinf{f} - \dfrac{\eps}{2} \right)\\
					&= \left( \Ssup{f} - \Sinf{f} \right) + \eps\\
					&= \eps
				\end{align*}
				\newpage
				\item[$(\Leftarrow)$] Soit $\eps>0$.

				Alors $\exists\Delta$ t.q. $\ssup{f}-\sinf{f}<\eps$.

				Mais alors,
				\begin{align*}
					\eps&> \ssup{f}-\sinf{f}\\
					&\geq \Ssup{f}-\Sinf{f}\\
					&\geq 0
				\end{align*}

				Du th\'eor\`eme du sandwich, $\Ssup{f}=\Sinf{f}$, car $\eps>0$ est arbitraire.

				Donc, $f \in \riemann$.
			\end{itemize}
		\end{proof}
		\begin{coro}
			S'il existe $\Delta \in \partitions$ t.q. $\ssup{f}=\sinf{f}$, alors $f \in \riemann$.
		\end{coro}
	\end{thm}
	\begin{thm}
		Toute fonction continue sur $[a,b]$ est int\'egrable sur $[a,b]$.
		\begin{proof}[d\'emonstration]~

			Soit $f \in \continue$.

			Soit $\eps>0$.

			Par la proposition d'Archim\`ede, $\exists n\in\entiers$ t.q. $n\eps>b-a$.

			\begin{rapp}~

				$f$ est uniform\'ement continue sur $[a,b]$ si $\left( \forall\eps>0 \right) \left( \exists\delta>0 \right)$ t.q. pour $x,y \in [a,b]$, $\abs{x-y}<\delta \Rightarrow \abs{f(x)-f(y)}<\eps$.
			\end{rapp}
			\begin{rapp}~

				Si $f$ est continue sur $[a,b]$, alors $f$ est uniform\'ement continue sur $[a,b]$.
			\end{rapp}

			Comme $f \in \continue$, elle est uniform\'ement continue sur $[a,b]$.

			Alors, $\exists\delta>0$ t.q. pour $x,y \in [a,b]$, $\abs{x-y}<\delta \Rightarrow \abs{f(x)-f(y)}<\frac{1}{n}$.

			Soit donc $\Delta \in \partitions: a=x_0<x_1,\dotsc<x_n=b$ avec $\norme{\Delta}<\delta$.

			Alors, $\msup{f}[x_{i-1},x_i] - \minf{f}[x_{i-1},x_i]<\frac{1}{n}$.
			\begin{rema}
				$\msup{f}[x_{i-1},x_i] - \minf{f}[x_{i-1},x_i]$ peut \^etre not\'e $\mathrm{osc}_f(\left[ x_{i-1},x_i \right])$.
			\end{rema}

			On obtient
			\begin{align*}
				\ssup{f}-\sinf{f}&= \sum_{i=1}^{n}\left[ \msup{f}[x_{i-1},x_i] - \minf{f}[x_{i-1},x_i] \right] \left( x_i-x_{i-1} \right)\\
				&< \dfrac{1}{n} \sum_{i=1}^{n}\left( x_i-x_{i-1} \right)\\
				&=\dfrac{b-a}{n}\\
				&< \eps
			\end{align*}

			Donc $f \in \riemann$.
		\end{proof}
	\end{thm}
	\begin{thm}
		Toute $f:[a,b] \to \reels$ monotone est int\'egrable.
		\begin{proof}[d\'emonstration]~

			\begin{enumerate}[label=(\arabic*)]
				\item Si $f$ est constante, alors $\ssup{f}-\sinf{f}=0<\eps$.
				\item Si $f$ est croissante,

				Soit $\eps>0$

				Soit $n \in \naturels$ t.q. $n\eps>(b-a)(f(b)-f(a))$

				Soit $\Delta:a=x_0<x_1<\dotsc<x_n=b$ avec $x_i=a+i\frac{b-a}{n}, i\in[0..n]$

				On a
				\begin{align*}
					\ssup{f}-\sinf{f}&= \sum_{i=1}^{n}\left[ \msup{f}[x_{i-1},x_i] - \minf{f}[x_{i-1},x_i] \right] (x_i-x_{i-1})\\
					&= \sum_{i=1}^{n}\left[ f(x_i) - f(x_{i-1}) \right] \left( \frac{b-a}{n} \right)\\
					&= \frac{b-a}{n} \left[ f(b)-f(a) \right]\\
					&< \eps
				\end{align*}

				Donc, $f \in \riemann$.
				\item Si $f$ est d\'ecroissante, alors $-f$ est croissante et $-f \in \riemann$.

				Donc, $f \in \riemann$.
			\end{enumerate}
		\end{proof}
	\end{thm}
	\begin{thm}
		~

		Si $f_1,f_2 \in \riemann$, alors $f_1+f_2 \in \riemann$ et $\int \left( f_1+f_2 \right) =\int f_1 + \int f_2$.
		\begin{proof}[d\'emonstration]~

			Soit $\eps>0$.

			Comme $f_i \in \riemann$, $\exists\Delta_i \in \partitions$ t.q. $\ssup{f_i}[\Delta_i]-\sinf{f_i}[\Delta_i]<\frac{\eps}{2}$.

			Soit $\Delta = \raffinement{\Delta_1}{\Delta_2}$.

			Alors, $\ssup{f_i}-\sinf{f_i}<\frac{\eps}{2}$.

			Supposons $\Delta:a=x_0<x_1<\dotsc<x_n=b$.

			On a
			\begin{align*}
				\ssup{f_1+f_2}&\leq \ssup{f_1} + \ssup{f_2}\\
				\sinf{f_1+f_2}&\geq \sinf{f_1} + \sinf{f_2}
			\end{align*}

			Car $\sup(f_1+f_2) \leq \sup f_1 + \sup f_2$ et $\inf(f_1+f_2) \geq \inf f_1 + \inf f_2$.

			Alors,
			\begin{align*}
				\ssup{f_1+f_2}-\sinf{f_1+f_2}&\leq \ssup{f_1} + \ssup{f_2} - \sinf{f_1} - \sinf{f_2}\\
				&< \frac{\eps}{2} + \frac{\eps}{2}\\
				&= \eps
			\end{align*}

			Donc, $f_1+f_2 \in \riemann$.

			De plus,
			\begin{align*}
				\int_{a}^{b}f_1+f_2&\leq \ssup{f_1+f_2}\\
				&\leq \ssup{f_1} + \ssup{f_2}\\
				&< \sinf{f_1}+\frac{\eps}{2} + \sinf{f_2}+\frac{\eps}{2}\\
				&\leq \int_{a}^{b}f_1+\frac{\eps}{2} + \int_{a}^{b}f_2+\frac{\eps}{2}
			\end{align*}

			Ainsi, $\int_{a}^{b}f_1+f_2 < \int_{a}^{b}f_1 + \int_{a}^{b}f_2 + \eps$, $\forall\eps>0$.

			Donc, $\int_{a}^{b}f_1+f_2 \leq \int_{a}^{b}f_1 + \int_{a}^{b}f_2$.

			De m\^eme, on peut montrer que $\int_{a}^{b}f_1+f_2 \geq \int_{a}^{b}f_1 + \int_{a}^{b}f_2$.

			Donc, $\int_{a}^{b}f_1+f_2 = \int_{a}^{b}f_1 + \int_{a}^{b}f_2$.
		\end{proof}
	\end{thm}
	\begin{thm}
		~

		Si $f \in \riemann$ et $\lambda \in \reels$, alors $\lambda f \in \riemann$ et $\int \lambda f = \lambda \int f$.
		\begin{proof}[d\'emonstration]~

			Laiss\'e en exercice.

			Utiliser $\frac{\eps}{\lambda}$ et $\ssup{\lambda f} = \lambda\ssup{f}$.
		\end{proof}
		\begin{coro}
			~

			Si $f,g \in \riemann$, alors $f \leq g \Rightarrow \int f \leq \int g$.
			\begin{proof}[d\'emonstration]~

				$g-f \geq 0 \Rightarrow \int g-f \geq 0 \Rightarrow \int g - \int f \geq 0$.
			\end{proof}
		\end{coro}
	\end{thm}

	\subsection{In\'egalit\'e du triangle}
	\begin{thm}[In\'egalit\'e du triangle]
		~

		Si $f \in \riemann$, alors $\abs{f} \in \riemann$ et $\abs{\int f} \leq \int\abs{f}$.
		\begin{proof}[d\'emonstration]~

			Soit $\eps>0$

			Alors, $\exists\Delta \in \partitions$ t.q. $\ssup{f}-\sinf{f}<\eps$.

			On a
			\begin{align*}
				\ssup{\abs{f}}-\sinf{\abs{f}}&= \sum_{i=1}^{n}\left[ \msup{\abs{f}}[x_{i-1},x_i] - \minf{\abs{f}}[x_{i-1},x_i] \right] \left( x_i-x_{i-1} \right)\\
				&\leq \sum_{i=1}^{n}\left[ \msup{f}[x_{i-1},x_i] - \minf{f}[x_{i-1},x_i] \right] \left( x_i-x_{i-1} \right)\\
				&= \ssup{f}-\sinf{f}\\
				&< \eps
			\end{align*}

			Donc, $\abs{f} \in \riemann$.

			Enfin,
			\begin{align*}
				-\abs{f} \leq f \leq \abs{f}&\Rightarrow -\int\abs{f} \leq \int f \leq \int\abs{f}\\
				&\Rightarrow \int f \leq \int\abs{f}
			\end{align*}
		\end{proof}
	\end{thm}
	\begin{thm}
		~

		Si $f \in \riemann$ et $a \leq c < d \leq b$, alors $f|_{[c,d]} \in \riemann$.
		\begin{proof}[d\'emonstration]~

			Soit $\eps>0$

			Comme $f \in \riemann$, $\exists\Delta_1 \in \partitions$ t.q. $\ssup{f}[\Delta_1]-\sinf{f}[\Delta_1] < \eps$.

			Soit $\Delta_2$ le raffinement de $\Delta_1$ en ajoutant les points $c$ et $d$.

			Alors, $\ssup{f}[\Delta_2]-\sinf{f}[\Delta_2] \leq \ssup{f}[\Delta_1]-\sinf{f}[\Delta_1] < \eps$

			Donc, $f \in \riemann[c,d]$.
		\end{proof}
	\end{thm}
	\begin{thm}
		~

		Si $f \in \riemann$ et $a<c<b$, alors $\int_{a}^{b}f = \int_{a}^{c}f + \int_{c}^{b}f$.
		\begin{proof}[d\'emonstration]~

			Soit $\eps>0$

			$f \in \riemann \Rightarrow f \in \riemann[a,c] \Rightarrow \exists\Delta_1 \in \partitions[a,c]$ t.q. $\ssup{f}[\Delta_1] - \sinf{f}[\Delta_1] < \frac{\eps}{2}$.

			De m\^eme, $\exists\Delta_2 \in \partitions[c,b]$ t.q. $\ssup{f}[\Delta_2] - \sinf{f}[\Delta_2] < \frac{\eps}{2}$.

			Posons $\Delta = \raffinement{\Delta_1}{\Delta_2}$. Alors, $\Delta \in \partitions$ et
			\begin{align*}
				\int_{a}^{b}f&\leq \ssup{f}\\
				&= \ssup{f}[\Delta_1] + \ssup{f}[\Delta_2]\\
				&< \sinf{f}[\Delta_1] + \frac{\eps}{2} + \sinf{f}[\Delta_2] + \frac{\eps}{2}\\
				&= \sinf{f}[\Delta_1] + \sinf{f}[\Delta_2] + \eps\\
				&\leq \int_{a}^{c}f + \int_{c}^{b}f + \eps
			\end{align*}

			Comme $\eps>0$ est arbitraire, on a $\int_{a}^{b}f \leq \int_{a}^{c}f + \int_{c}^{b}f$.

			De m\^eme, $\int_{a}^{b}f \geq \int_{a}^{c}f + \int_{c}^{b}f$.
		\end{proof}
	\end{thm}
	\begin{thm}
		Soit $f \in \bornee$. Soit $n \in \naturels$.

		Si $f$ poss\`ede $n$ discontinuit\'es dans $[a,b]$, alors $f \in \riemann$.
		\begin{proof}[d\'emonstration]~

			Pour $n=0$, $f \in \continue$, donc $f \in \riemann$ est un r\'esultat connu.

			Supposons l'\'enonc\'e vrai pour $n$.

			Supposons que $f \in \bornee$ admet $n+1$ discontinuit\'es.

			Soit $\eps>0$.

			Soit $M=\sup\limits_{x\in[a,b]}\abs{f(x)}$

			Il y a deux cas \`a consid\'erer

			\begin{enumerate}
				\item $a$ ou $b$ est une discontinuit\'e

				SPDG, supposons que $a$ est la discontinuit\'e.

				Soit $\eta \in \reels^+$ t.q. $a$ est l'unique discontinuit\'e de $[a,a+\eta]$ et $\eta < \dfrac{\eps}{4M}$.

				Alors, $[a+\eta,b]$ contient $n$ discontinuit\'es.

				De l'hypoth\`ese de r\'ecurrence, $f \in \riemann[a+\eta,b]$.

				Il existe donc $\Delta \in \partitions[a+\eta,b]$ t.q. $\ssup{f} - \sinf{f} < \dfrac{\eps}{2}$.

				Posons $\Delta_\eps = \raffinement{\Delta}{\{a\}}$.

				On a donc
				\begin{align*}
					\ssup{f}[\Delta_\eps] - \sinf{f}[\Delta_\eps]&= \left( \ssup{f} - \sinf{f} \right) + \left( \msup{f}[a,a+\eta] - \minf{f}[a,a+\eta] \right) \eta\\
					&< \frac{\eps}{2} + 2M\eta\\
					&< \dfrac{\eps}{2} + \dfrac{\eps}{2}\\
					&= \eps
				\end{align*}
				\item ni $a$ ni $b$ ne sont des discontinuit\'es

				Soit $c \in ]a,b[$ qui est une discontinuit\'e de $f$.

				Soit $\eta \in \reels^+$ t.q. $c$ est l'unique discontinuit\'e de $[c-\eta,c+\eta] \subset [a,b]$ et $\eta < \dfrac{\eps}{8M}$.

				Alors, $[a,c-\eta]$ et $[c+\eta,b]$ contiennent au plus $n$ discontinuit\'es, par l'hypoth\`ese de r\'ecurrence

				$\exists\Delta_1\in \partitions[a,c-\eta]$ t.q.  $\ssup{f}[\Delta_1] - \sinf{f}[\Delta_1] < \frac{\eps}{4}$.

				$\exists\Delta_2\in \partitions[c+\eta,b]$ t.q.  $\ssup{f}[\Delta_2] - \sinf{f}[\Delta_2] < \frac{\eps}{4}$.

				Posons $\Delta_\eps = \raffinement{\Delta_1}{\Delta_2}$.

				On a donc
				\begin{align*}
					\ssup{f}[\Delta_\eps] - \sinf{f}[\Delta_\eps] &= \left[ \ssup{f}[\Delta_1] - \sinf{f}[\Delta_1] \right] + \left[ \ssup{f}[\Delta_2] - \sinf{f}[\Delta_2] \right]\\
					&+ \left[ \msup{f}[c-\eta,c+\eta] - \minf{f}[c-\eta,c+\eta] \right] \left( 2\eta \right)\\
					&< \dfrac{\eps}{4} + \dfrac{\eps}{4} + 4M\eta\\
					&< \dfrac{\eps}{4} + \dfrac{\eps}{4} + \dfrac{\eps}{2}\\
					&= \eps
				\end{align*}
			\end{enumerate}
		\end{proof}
	\end{thm}

	\begin{thm}
		Soient $f:[a,b] \to [c,d] \in \riemann$ et $g:[c,d] \to \reels \in \continue[c,d]$.

		Alors $g \circ f \in \riemann$.

		\begin{rema}
			L'hypoth\`ese que $g \in \continue[c,d]$ est n\'ecessaire.

			\begin{exem}
				~

				$
				\begin{array}{rcl}
					f:[0,1]&\to&\reels\\
					x&\mapsto&\left\lbrace \begin{array}{lcl}
						\frac{1}{n}&\text{si}&x=\frac{m}{n} \text{ et } \mathtt{pgcd}(m,n)=1\\
						0&\text{sinon}
					\end{array}\right.
				\end{array}$

				Fonction de Dirichlet modifi\'ee.

				$
				\begin{array}{rcl}
					g:[0,1]&\to&\reels\\
					x&\mapsto&\left\lbrace \begin{array}{lcl}
						1&\text{si}&x>0\\
						0&\text{sinon}
					\end{array}\right.
				\end{array}$

				$f,g \in \riemann$.

				$g \circ f(x)=\left\lbrace \begin{array}{lcl}
					1&\text{si}&x \in \rationels\\
					0&\text{sinon}
				\end{array}\right.$

				$g \circ f \not\in \riemann$.

				Fonction de Dirichlet.
			\end{exem}
		\end{rema}
	\end{thm}

	\begin{lem}
		Si $f \in \bornee$, $\Delta,\Delta' \in \partitions$ et $\Delta'$ s'obtient de $\Delta$ en ajoutant un unique point, alors $\ssup{f} - \ssup{f}[\Delta'] \leq 2\msup{\abs{f}}[a,b] \cdot \norme{\Delta}$.

		\begin{proof}[d\'emonstration]~

			Soient $\Delta:a=x_0 < x_1 < \dotsc < x_{i-1} < x_i < \dotsc < x_n=b$.

			$\Delta':a=x_0 < x_1 < \dotsc < x_{i-1} < \bar{x} < x_i < \dotsc < x_n=b$.

			\begin{align*}
				\ssup{f} - \ssup{f}[\Delta'] &= \msup{f}[x_{i-1},x_i] \cdot \left( x_i-x_{i-1} \right) - \msup{f}[x_{i-1},\bar{x}] \cdot \left( \bar{x}-x_{i-1} \right) - \msup{f}[\bar{x},x_i] \cdot \left( x_i-\bar{x} \right)\\
				&= \left( \bar{x}-x_{i-1} \right) \left( \msup{f}[x_{i-1},x_i] - \msup{f}[x_{i-1},\bar{x}] \right) + \left( x_i-\bar{x} \right) \left( \msup{f}[x_{i-1},x_-i] - \msup{f}[\bar{x},x_i] \right)\\
				&\leq 2\msup{\abs{f}}[x_{i-1}x_i] \left( \left( \bar{x}-x_{i-1} \right) - \left( x_i-\bar{x} \right) \right)\\
				&\leq 2\msup{\abs{f}} \norme{\Delta}
			\end{align*}
		\end{proof}

		\begin{coro}
			Si $f \in \bornee$, $\Delta,\Delta' \in \partitions$ et $\Delta'$ s'obtient de $\Delta$ en ajoutant $p$ points, au plus un point par sous-intervalle de $\Delta$, alors $\ssup{f} - \ssup{f}[\Delta'] \leq 2p\msup{\abs{f}}[a,b] \cdot \norme{\Delta}$.
		\end{coro}
	\end{lem}

	\subsection{Th\'eor\`eme de Darboux}
	\begin{thm}[Darboux, 1875]
		~

		Si $f \in \bornee$, alors
		\begin{align*}
			\begin{split}
				\Ssup{f}&= \lim\limits_{\norme{\Delta} \to 0} \ssup{f}
			\end{split}
			&
			\begin{split}
				\Sinf{f}&= \lim\limits_{\norme{\Delta} \to 0} \sinf{f}
			\end{split}
		\end{align*}

		\begin{proof}[d\'emonstration]~

			Soit $\eps>0$.

			Puisque $\Ssup{f} = \inf \ssup{f}$, on a que $\Ssup{f} + \dfrac{\eps}{2}$ n'est pas un minorant des $\ssup{f}$.

			Ainsi, $\exists\Delta_0 : a=x_0 < \dotsc < x_n=b$ t.q. $\ssup{f}[\Delta_0] < \Ssup{f} + \dfrac{\eps}{2}$.

			Soit $\delta>0$ t.q. $\delta < \min\limits_{i \in [1..n]}\abs{x_i-x_{i-1}}$ et $\delta < \dfrac{\eps}{4 (n-1) \msup{\abs{f}}}$.

			Soit $\Delta \in \partitions$ t.q. $\norme{\Delta} < \delta$.

			Alors, $\norme{\Delta} \xrightarrow[n \to \infty]{} 0$.

			Consid\'erons $\Delta' = \raffinement{\Delta}{\Delta_0}$.

			Comme $\norme{\Delta'} \leq \norme{\Delta} < \norme{\Delta_0}$, aucun sous-intervalle ouvert de $\Delta$ ne contient plus d'un point de $\Delta_0$.

			Comme $\Delta'$ s'obtient de $\Delta$ en ajoutant au plus $n-1$ points $\left( x_1, x_2, \dotsc, x_{n-1} \right)$,
			\begin{align*}
				\ssup{f} - \ssup{f}[\Delta'] &\leq 2(n-1) \msup{\abs{f}} \norme{\Delta}\\
				&< 2(n-1) \msup{f} \delta\\
				&< \dfrac{\eps}{2}
			\end{align*}

			On a donc
			\begin{align*}
				\ssup{f} &\leq \ssup{f}[\Delta'] + \dfrac{\eps}{2}\\
				&< \ssup{f}[\Delta_0] + \dfrac{\eps}{2}\\
				&< \Ssup{f} + \dfrac{\eps}{2} + \dfrac{\eps}{2}\\
				&= \Ssup{f} + \eps
			\end{align*}

			Comme $\eps>0$ est arbitraire, on a donc
			\begin{align*}
				\lim\limits_{\norme{\Delta} \to 0}\ssup{f}&= \Ssup{f}
			\end{align*}

			Enfin,
			\begin{align*}
				\lim\limits_{\norme{\Delta} \to 0} \sinf{f}&= \lim\limits_{\norme{\Delta} \to 0} -\ssup{-f}\\
				&= -\lim\limits_{\norme{\Delta} \to 0} \ssup{-f}\\
				&= -\Ssup{-f}\\
				&= \Sinf{f}
			\end{align*}
		\end{proof}
	\end{thm}

	\begin{defin}


		Soit $f \in \bornee$.

		Soit $\Delta : a=x_0 < x_1 < \dotsc < x_n=b \in \partitions$.

		Soient $\bar{x}_i \in [x_{i-1},x_i]$, pour $i \in [1..n]$.

		Le nombre r\'eel
		\begin{align*}
			\srie{f}&= \sum_{i=1}^{n}f(\bar{x}_i) \cdot (x_i-x_{i-1})
		\end{align*}

		est appel\'e \emph{somme de Riemann} de la fonction $f$ correspondant \`a la partition $\Delta$ et aux points $\{\bar{x}_i\}_{i \in [1..n]}$.
	\end{defin}

	\begin{thm}
		~

		Soit $f \in \riemann$.

		Alors, $(\forall\eps>0)$, $(\exists\delta=\delta(\eps))$ t.q. pour toute partition $\Delta$ de $[a,b]$ avec $\norme{\Delta}<\delta$ et pour tout choix de points $\{\bar{x}_i\}$, on a
		\[
		\abs{\int_{a}^{b}f - S(f, \Delta, \{\bar{x}_i\})} < \eps
		\]

		c'est-\`a-dire.
		\[
		\lim_{\norme{\Delta} \to 0}\srie{f} = \int_{a}^{b}f
		\]
		\begin{proof}[d\'emonstration]~

			On a $\sinf{f} \leq S(f, \Delta, \{\bar{x}_i\}) \leq \ssup{f}$.

			Donc,
			\[
			\Sinf{f} = \lim_{\norme{\Delta} \to 0}\sinf{f} \leq \lim_{\norme{\Delta} \to 0}\srie{f} \leq \lim_{\norme{\Delta} \to 0}\ssup{f} = \Ssup{f}
			\]

			Comme $f \in \riemann$, on a $\Sinf{f} = \Ssup{f} = \displaystyle\int_{a}^{b}f$.

			Par le th\'eor\`eme du sandwich, $\srie{f} = \displaystyle\int_a^bf$.
		\end{proof}
	\end{thm}

	\subsection{Loi de la moyenne}
	\begin{thm}[Loi de la moyenne]~

		Soit $f \in \riemann$.

		Alors, $\exists\mu \in \left[ \minf{f}[a,b], \msup{f}[a,b] \right]$ t.q. $\displaystyle\int_{a}^{b}f = (b-a) \cdot \mu$.
		\begin{proof}[d\'emonstration]~

			Soit $\phi$ la fonction donn\'ee par $\phi(x)=(b-a)x$.

			On a $\minf{f} \leq f \leq \msup{f}$.

			Donc
			\[
			\phi(\minf{f}) = (b-a)\minf{f} = \int_{a}^{b}\minf{f} \leq \int_{a}^{b}f \leq \msup{f} = (b-a)\msup{f} = \phi(\msup{f})
			\]
			\begin{rapp}
				\begin{thm}[Th\'eor\`eme de valeur interm\'ediaire]~

					$f$ continue sur $[a,b]$, $f(a)<c<f(b)$ implique $\exists x_0 \in [a,b]$ t.q. $f(x_0)=c$.
				\end{thm}
			\end{rapp}

			Comme $\phi$ est continue sur $[\minf{f},\msup{f}]$, du TVI, $\exists\mu\in[\minf{f},\msup{f}]$ t.q. $\phi(\mu)=c$ pour tout $c \in [\phi(\minf{f}),\phi(\msup{f})]$.

			En particulier, si $c=\displaystyle\int_{a}^{b}f$, $\exists\mu$ t.q. $\phi(\mu)=\displaystyle\int_{a}^{b}c$, c'est-\`a-dire t.q. $(b-a)\mu=\displaystyle\int_{a}^{b}f$.
		\end{proof}
	\end{thm}

	\begin{thm}
		~

		Soit $f \in \riemann$.

		Soit
		$\begin{array}{rcl}
			F:[a,b]&\to&\reels\\
			x&\mapsto&F(x) = \displaystyle\int_{a}^{x}f(f) dt
		\end{array}$

		Alors,
		\begin{enumerate}[label=\alph*)]
			\item $\abs{F(x_1)-F(x_2)} \leq \msup{\abs{f}}(b-a) \cdot \abs{x_1-x_2}$ pour tous $x_1,x_2 \in [a,b]$;
			\item $F$ est uniform\'ement continue sur $[a,b]$;
			\item Si $f$ est continue, alors $F$ est diff\'erentiable et $F'=f$.
		\end{enumerate}
		\begin{proof}[d\'emonstration]~

			\begin{enumerate}[label=\alph*)]
				\item Supp $x_1>x_2$

				On a
				\begin{align*}
					\abs{F(x_1)-F(x_2)}&= \abs{\int_{a}^{x_1}f - \int_{a}^{x_2}f}\\
					&= \abs{\int_{x_2}^{x_1}f}\\
					&\leq \int_{x_2}^{x_1}\abs{f}\\
					&\leq \int_{x_2}^{x_1}\msup{\abs{f}}\\
					&= \msup{\abs{f}} \cdot \abs{x_1-x_2}
				\end{align*}
				\item Soit $\eps>0$.

				Prenons $\delta = \dfrac{\eps}{\msup{\abs{f}}}$.

				Soient $x,y \in [a,b]$ avec $\abs{x-y}<\delta$.

				Alors,
				\begin{align*}
					\abs{F(x)-F(y)}&\leq \msup{\abs{f}} \cdot \abs{x-y}\\
					&< \msup{\abs{f}} \cdot \delta\\
					&= \eps
				\end{align*}
				\item Soit $x_0 \in [a,b]$.

				On a
				\begin{align*}
					F'(x_0)&= \lim_{x \to x_0}\dfrac{F(x)-F(x_0)}{x-x_0}\\
					&= \lim_{x \to x_0}\dfrac{\int_{a}^{x}f - \int_{a}^{x_0}}{x-x_0}\\
					&= \lim_{x \to x_0}\dfrac{\int_{x_0}^{x}f}{x-x_0}\\
					&\overset{\text{Loi de la moyenne}}{=} \lim_{x \to x_0}\dfrac{(x-x_0)f(x_0+\theta(x-x_0))}{x-x_0}&&\theta \in [0,1]\\
					&= \lim_{x \to x_0}f(x_0+\theta(x-x_0))\\
					&= f(x_0)
				\end{align*}
			\end{enumerate}
		\end{proof}
	\end{thm}

	\subsection{Th\'eor\`eme fondamental du calcul diff\'erentiel et int\'egral}
	\begin{nota}
		$F$ est une primitive de $f$.
	\end{nota}
	\begin{coro}
		Si $f$ est continue, alors $f$ admet au moins une primitive.
	\end{coro}
	\begin{coro}
		Si $F_1$ et $F_2$ sont deux primitives de $f$, alors $F_1-F_2=C$ pour une constante $C$.
	\end{coro}
	\begin{thm}[Th\'eor\`eme fondamental du calcul int\'egral]~

		Si $f \in \riemann$ et $F$ est une primitive de $f$, alors $\int_{a}^{b}f=F(b)-F(a)$.
		\begin{proof}[d\'emonstration]~

			Soit $\Delta:a=x_0<x_1<\dotsc<x_n=b \in \partitions$.

			Comme $F$ est continue et diff\'erentiable sur $[a,b]$ et a fortiori sur $[x_{i-1},x_i]$, le th\'eor\`eme de la moyenne donne $t_i \in [x_{i-1},x_i]$ t.q. $\dfrac{F(x_i) - f(x_{i-1})}{x_i-x_{i-1}} = F'(t_i) = f(t_i)$.

			On a
			\begin{align*}
				F(b)-F(a)&= \sum_{i=1}^{n}\left[ F(x_i)-F(x_{i-1}) \right]\\
				&= \sum_{i=1}^{n}(x_i-x_{i-1}) \cdot f(t_i)
			\end{align*}

			De plus,
			\[
			\begin{array}{rrcccl}
				&\minf{f}[x_{i-1},x_i] &\leq& f(t_i) &\leq& \msup{f}[x_{i-1},x_i]\\
				\Rightarrow&\displaystyle\sum_{i=1}^{n}(x_i-x_{i-1})\minf{f}[x_{i-1},x_i] &\leq& \displaystyle\sum_{i=1}^{n}(x_i-x_{i-1})f(t_i) &\leq& \displaystyle\sum_{i=1}^{n}(x_i-x_{i-1})\msup{f}[x_{i-1},x_i]\\
				\Rightarrow&\sinf{f} &\leq& F(b)-F(a) &\leq& \ssup{f}\\
				\Rightarrow&\displaystyle\int_{a}^{b}f=\Sinf{f}=\lim\limits_{\norme{\Delta}\to0}\sinf{f} &\leq& F(b)-F(a) &\leq& \lim\limits_{\norme{\Delta}\to0}\ssup{f}=\Ssup{f}=\displaystyle\int_{a}^{b}f
			\end{array}
			\]

			Donc, $\int_{a}^{b}f=F(b)-F(a)$.
		\end{proof}
	\end{thm}
	\begin{prop}
		Soit $f:[a,b] \to \reels$ t.q. $f(x)=0$ sauf peut-\^etre en un nombre fini de points. Alors,
		\begin{enumerate}[label=\alph*)]
			\item $f \in \riemann$;
			\item $\int_{a}^{b}f=0$.
		\end{enumerate}
		\begin{proof}[d\'emonstration]~

			\begin{enumerate}[label=\alph*)]
				\item d\'ej\`a fait
				\item Soit $p$ le nombre de points o\`u $f \neq 0$.

				Pour $p=0$, c'est trivial.

				Supposons que la propri\'et\'e est vraie pour $p$.

				Supposons que $f \neq 0$ en $p+1$ points.

				Il y a deux cas \`a consid\'erer
				\begin{enumerate}[label=\arabic*)]
					\item $\exists c \in [a,b]$ avec $f(c) \neq 0$.

					Soit $\eps>0$.

					Soit $\eta>0$ t.q.
					\begin{enumerate}[label=\roman*)]
						\item $a<c-\eta<c+\eta<b$;
						\item $c$ est le seul point de $[c-\eta,c+\eta]$ o\`u $f\neq0$;
						\item $\eta<\min\left\lbrace \dfrac{\eps}{4\msup{f}}, \dfrac{-\eps}{4\minf{f}} \right\rbrace$.
					\end{enumerate}

					Par l'hypoth\`ese de r\'ecurrence,
					\[
					\begin{array}{rcccl}
						\displaystyle\int_{a}^{c-\eta}f&=&0&\displaystyle\int_{c+\eta}^{b}f
					\end{array}
					\]

					Du crit\`ere d'int\'egrabilit\'e, $\exists \Delta_1:a<x_0<x_1,\dotsc<x_n=c-\eta$, $\exists \Delta_2:c+\eta=y_0<y_1<\dotsc<y_n=b$ t.q. $\ssup{f}[\Delta_i] \leq \ssup{f}[\Delta_i]-\sinf{f}[\Delta_i]<\frac{\eps}{4}$, pour $i\in\{1,2\}$.

					Prenons $\Delta=\raffinement{\Delta_1}{\Delta_2}$.

					On a
					\begin{align*}
						\begin{split}
							\ssup{f}&= \ssup{f}[\Delta_1] + 2\eta \msup{f}[c-\eta,c+\eta] + \ssup{f}[\Delta_2]\\
							&\leq \ssup{f}[\Delta_1] + 2\eta \msup{f} + \ssup{f}[\Delta_2]\\
							&< \frac{\eps}{4} + \frac{\eps}{2} + \frac{\eps}{4}\\
							&= \eps
						\end{split}
					\end{align*}

					De m\^eme, $\sinf{f}[\Delta_i] > \ssup{f}[\Delta_i] < \frac{\eps}{4}$, pour $i\in\{1,2\}$.

					et
					\begin{align*}
						\sinf{f}&= \sinf{f}[\Delta_1] + 2\eta \minf{f}[c-\eta,c+\eta] + \sinf{f}[\Delta_2]\\
						&\leq \sinf{f}[\Delta_1] + 2\eta \minf{f} + \sinf{f}[\Delta_2]\\
						&< -\frac{\eps}{4} - \frac{\eps}{2} - \frac{\eps}{4}\\
						&= -\eps
					\end{align*}

					Donc, $\eps < \sinf{f} \leq \ssup{f} < \eps$.

					Comme $\eps>0$ est arbirtaire, on en d\'eduit que $\int_{a}^{b}f=0$.
					\item $f(c)\neq0$ en $a$ ou en $b$.

					On proc\`ede de la m\^eme mani\`ere avec un sous-intervalle de largeur $\eta$ autour de $a$ et de $b$ t.q. $a$ et $b$ sont les seules discontinuit\'es dans ces intervalles.
				\end{enumerate}
			\end{enumerate}
		\end{proof}
	\end{prop}
	\begin{coro}
		Si $f \in \riemann$ et $g:[a,b] \to \reels$ t.q. $f=g$ sauf peut-\^etre en nu nombre fini de points, alors $\int_{a}^{b}f = \int_{a}^{b}g$.
	\end{coro}

	\section{Techniques d'int\'egration}
	\begin{thm}[Int\'egration par parties]~

		Soient $f,g:[a,b] \to \reels$ deux fonctions diff\'erentiables t.q. $f',g' \in \riemann$. Alors,
		\[
		\int_{a}^{b}fg'=fg\mid_a^b-\int_{a}^{b}gf'
		\]
		\begin{proof}[d\'emonstration]~

			Posons $h=fg$.

			Alors,
			\begin{align*}
				h'&= f'g + fg'\\
				\int_{a}^{b} h'&= \int_{a}^{b} f'g + \int_{a}^{b} fg'\\
				\int_{a}^{b} fg'&= \int_{a}^{b} h - \int_{a}^{b} f'g\\
				&= h\mid_a^b - \int_{a}^{b} f'g\\
				&= fg\mid_a^b - \int_{a}^{b} f'g
			\end{align*}
		\end{proof}
	\end{thm}
	\begin{thm}[Changement de variable/Substitution]~

		Soient $f:[a,b] \to \reels$ continue et $\phi:[\alpha,\beta] \to [a,b]$ de classe $C^1$, c'est-\`a-dire $\phi$ est d\'erivable et $\phi'$ est continue.

		Si $\phi(\alpha)=a$ et $\phi(\beta)=b$, alors
		\begin{align*}
			\int_{a}^{b}f(x) dx&= \int_{\alpha}^{\beta}f(\phi(t))\phi'(t) dt
		\end{align*}
		\begin{proof}[d\'emonstration]~

			Posons $h(x)=\int_{a}^{x}f(t) dt$.

			Du th\'eor\`eme fondamental, $h$ est uniform\'ement continue, diff\'erentiable et $h'=f$.

			Soit $g(t)=(h \circ \phi)(t) = h(\phi(t)) = \int_{a}^{\phi(t)}f(t) dt$.

			On a $h,\phi$ diff\'erentiables, donc $g$ l'est aussi et
			\begin{align*}
				g'(t)&= h'(\phi(t)) \cdot \phi'(t)\\
				&= f(\phi(t)) \cdot \phi'(t)
			\end{align*}

			Enfin,
			\begin{align*}
				\int_{\alpha}^{\beta}f(\phi(t)) \cdot \phi'(t) dt&= \int_{\alpha}^{\beta}g'(t) dt\\
				&= g(\beta)-g(\alpha)\\
				&= h(\phi(\beta)) - h(\phi(\alpha))\\
				&= h(b)-h(a)\\
				&= \int_{a}^{b}f(t) dt - \int_{a}^{a}f(t) dt\\
				&= \int_{a}^{b}f(t) dt
			\end{align*}
		\end{proof}
	\end{thm}

	\subsection{Fractions partielles}
	\begin{multline*}
		\dfrac{P(x)}{(x-a)^n(x^2+bx+c)^m} = \dfrac{A_1}{x-a} + \dfrac{A_2}{(x-a)^2} + \dotsb + \dfrac{A_n}{(x-a)^n}+\\
		\dfrac{B_1x+C_1}{x^2+bx+c} + \dfrac{B_2x+C_2}{(x^2+bx+c)^2} + \dotsb + \dfrac{B_mx+C_m}{(x^2+bx+c)^m}
	\end{multline*}

	Avec $b^2-4ac<0$.

	On ram\`ene sur d\'enominateurs communs.

	On ram\`ene en une fraction.

	On r\'esoud le syst\`eme d'\'equations avec $P(x)$.

	On int\`egre chaque fraction.

	\subsubsection{Quelques substitutions}
	\begin{enumerate}
		\item $f\left( x,\left( \frac{ax+b}{cx+d} \right)^\frac{m_1}{n_1}, \left( \frac{ax+b}{cx+d} \right)^\frac{m_2}{n_2}, \dotsb \right)$, ou $f$ est une fonction rationnelle $f=\frac{P(x)}{Q(x)}$.

		On pose $t^n=\frac{ax+b}{cx+d}$ o\`u $n$ est un multiple commun de $n_1,n_2,\dotsb$.
		\begin{exem}
			~

			$\displaystyle\int\dfrac{x^2}{\sqrt{x-1}} = \displaystyle\int x^2(x-1)^{\sfrac{-1}{2}}$.

			Posons $x-1=t^2$.

			Alors, $dx=2t dt$.

			On obtient $\displaystyle\int x^2(x-1)^{\sfrac{-1}{2}} = \displaystyle\int (t^2-1)^2(t^2)^{\sfrac{-1}{2}}2tdt = 2 \int (t^2+1)^2 dt$.
		\end{exem}
		\item $\displaystyle\int x^\alpha(a+bx^\beta)^\gamma dx$ avec $\alpha,\beta,\gamma \in \rationels$.

		On pose $t=x^\beta$.

		Alors, $dt = \beta x^{\beta-1} dx$.

		Donc, $dx = \dfrac{dt}{\beta x^{\beta-1}} = \dfrac{dt}{\beta t^{\sfrac{\beta-1}{\beta}}}$.
		\begin{exem}~

			$\displaystyle\int\dfrac{dx}{x^4\sqrt{1+x^2}} = \displaystyle\int x^{-4}(1+x^2)^{\sfrac{-1}{2}} dx$.

			Posons $t=x^2$.

			On a $dt=2xdx \Rightarrow dx=\dfrac{dt}{2x} = \dfrac{dt}{2\sqrt{t}}$.

			On obtient
			\begin{align*}
				\int x^{-4}(1+x^2)^{\sfrac{-1}{2}} dx&= \int t^{-2} (1+t)^{\sfrac{-1}{2}} \dfrac{dt}{2\sqrt{t}}\\
				&= \dfrac{1}{2} \int t^{\sfrac{-5}{2}}(1+t)^{\sfrac{-1}{2}} dt\\
				&=\dfrac{1}{2}\int t^{-3}(\dfrac{1+t}{t})^{\sfrac{-1}{2}} dt
			\end{align*}

			Posons $u^2=\dfrac{1+t}{t} = 1+\dfrac{1}{t}$. On a $u^2-1=\dfrac{1}{t} \Rightarrow t=\dfrac{1}{u^2-1}$.

			Alors, $dt = (-1) (u^2-1)^{-2} (2u) du$.

			Ainsi,
			\begin{align*}
				\dfrac{1}{2} \int t^{-3} \left( \dfrac{1+t}{t} \right)^{\sfrac{-1}{2}} dt&= -\dfrac{1}{2} \int (u^2-1)^3 u^{-1} (u^2-1)^{-1} 2udu\\
				&= -\int (u^2-1) du
			\end{align*}
		\end{exem}
		\item $f(\sin x, \cos x)$.

		On pose $t=\tan\frac{x}{2}$. On a $x=2 \arctan t$.

		Alors, $dx = \dfrac{2}{1+t^2} dt$.
		\begin{exem}~

			$\displaystyle\int \dfrac{dx}{2+\cos x}$.

			On a
			\begin{align*}
				\cos x&= \cos^2\frac{x}{2} - \sin^2\frac{x}{2}\\
				&= \dfrac{\cos^2\frac{x}{2}} {\cos^2\frac{x}{2} - \sin^2\frac{x}{2}} - \dfrac{\sin^2\frac{x}{2}} {\cos^2\frac{x}{2} + \sin^2\frac{x}{2}}\\
				&= \dfrac{1}{\dfrac{\cos^2 + \sin^2} {\cos^2}} - \dfrac{\dfrac{1}{\cos^2} (\sin^2)} {\dfrac{1}{\cos^2} (\cos^2 + \sin^2)}\\
				&= \dfrac{1}{1 + \tan^2 \frac{x}{2}} - \dfrac{\tan^2 \frac{x}{2}} {1 + \tan^2 \frac{x}{2}}\\
				&= \dfrac{1 - \tan^2 \frac{x}{2}}{1 + \tan^2 \frac{x}{2}}\\
				&= \dfrac{1-t^2}{1+t^2}
			\end{align*}

			Alors,
			\begin{align*}
				\int\dfrac{dx}{2+\cos x}&= \int\dfrac{2dt}{1+t^2} \dfrac{1}{2 + \dfrac{1 - t^2}{1 + t^2}}\\
				&= \int\dfrac{2dt}{3+t^2}\\
				&= 2\int\dfrac{dt}{t^2+\sqrt{3}^2}\\
				&= \dfrac{2}{\sqrt{3}} \arctan\dfrac{t}{\sqrt{3}}\\
				&= \dfrac{2}{\sqrt{3}} \arctan \dfrac{\tan \frac{x}{2}}{\sqrt{3}}
			\end{align*}
		\end{exem}
		\item $f\left( x,\sqrt{a^2-x^2} \right)$.

		On pose $x=a\sin t$. On a $dx=a\cos t dt$.
		\begin{exem}
			\begin{align*}
				\int x^2\sqrt{a^2-x^2} dx&= \int a^2\sin^2t \sqrt{a^2-a^2\sin^2t} a\cos t dt\\
				&= \int a^4 \sin^2t \cos^2 t dt\\
				&= \int a^4 \dfrac{1}{4} \sin^2 2t dt\\
				&= \int \dfrac{a^4}{4} \cdot \dfrac{1-\cos 4t}{2} dt\\
				&= \dfrac{a^4}{8} \left[ t - \dfrac{\sin 4t}{4} \right]
			\end{align*}
		\end{exem}
		\newpage
		\item
		\begin{rapp}
			\begin{align*}
				\begin{split}
					\sinh x&= \dfrac{e^x-e^{-x}}{2}
				\end{split}
				&
				\begin{split}
					\cosh x&= \dfrac{e^x+e^{-x}}{2}
				\end{split}
				&
				\begin{split}
					\tanh x&= \dfrac{\sinh x}{\cosh x}\\
					&= \dfrac{e^{2x}-1}{e^{2x}+1}
				\end{split}
			\end{align*}
			\begin{align*}
				\begin{split}
					(\sinh x)'&= \dfrac{e^x+e^{-x}}{2}\\
					&= \cosh x
				\end{split}
				&
				\begin{split}
					(\cosh x)'&= \dfrac{e^x-e^{-x}}{2}\\
					&= \sinh x
				\end{split}
			\end{align*}
			\begin{align*}
					-\cosh^2x+\sinh^2x&= -\left( \dfrac{e^x+e^{-x}}{2} \right)^2 + \left( \dfrac{e^x-e^{-x}}{2} \right)^2\\
					&= -\dfrac{e^{2x}+2e^xe^{-x}+2^{-2x}}{4} + \dfrac{e^{2x}-2e^xe^{-x}+2^{-2x}}{4}\\
					&= -1
			\end{align*}

			Donc, $\cosh^2x-\sinh^2=1$.

			Alors, $1-\tanh^2x = \sech^2x \Rightarrow \sech^2x+\tanh^2x = 1$.

			Formellement,
			\begin{align*}
				\begin{split}
					\sinh x&= \dfrac{e^z-e^{-z}}{2}
				\end{split}
				&
				\begin{split}
					\cosh x&= \dfrac{e^z+e^{-z}}{2}
				\end{split}
			\end{align*}
			avec $z \in \complexes$.
		\end{rapp}

		$f\left( x,\sqrt{x^2-a^2} \right)$.

		On pose $x = a \cosh t$. On a $dx = a \sinh t dt$.
		\begin{exem}
			\begin{align*}
				\int \dfrac{x^2}{\sqrt{x^2-a^2}} dx &= \int \dfrac{a^2\cosh^2t}{\sqrt{a^2\cosh^2t-a^2}}a\sinh tdt\\
				&= \int \dfrac{a^2\cosh^2t}{a\sinh t}a\sinh tdt\\
				&= \int a^2\cosh^2tdt\\
				&= \int a^2 \dfrac{a+\cosh 2t}{2}dt\\
				&= \dfrac{a^2t}{2} + \dfrac{a^2}{2}\dfrac{\sinh2t}{2}
			\end{align*}
		\end{exem}
		\begin{rema}
			$\mathrm{arccosh}t = \ln\left( t+\sqrt{t^2-1} \right)$, $\mathrm{arcsinh}t = \ln\left( t+\sqrt{1+t^2} \right)$.
		\end{rema}
		\item $f\left( x, \sqrt{x^2+a^2} \right)$.

		On pose $x = a\sinh t$. On a $dx = a\cosh t dt$.
		\begin{exem}
			\begin{align*}
				\int \dfrac{x^3}{(x^2+a^2)^{\sfrac{3}{2}}} dx &= \int \dfrac{a^3\sinh^3t}{(a^2\sinh^2t+a^2)^{\sfrac{3}{2}}} a\cosh t dt\\
				&= \int \dfrac{a^3\sinh^3t}{a^3\cosh^3t} a\cosh t dt\\
				&= a\int \dfrac{\sinh^3t}{\cosh^2t} dt\\
				&= a\int \dfrac{\sinh t \sinh^2t}{\cosh^2t} dt\\
				&= a\int \dfrac{\sinh t (\cosh^2t-1)}{\cosh^2t} dt\\
				&= a\int \left( \sinh t - \dfrac{\sinh t}{\cosh^2t} \right) dt\\
				&\text{posons }u=\cosh t, du=\sinh t dt\\
				&= a\cosh t + \dfrac{a}{\cosh t}\\
				&= a\sqrt{\cosh^2t} + \dfrac{a}{\sqrt{\cosh^2t}}\\
				&= a\sqrt{1+\sinh^2t} + \dfrac{a}{\sqrt{1+\sinh^2t}}\\
				&= a\sqrt{1+ \left( \frac{x}{a} \right) ^2} + \dfrac{a}{\sqrt{1+ \left( \frac{x}{a} \right) ^2}}
			\end{align*}
		\end{exem}
		\item $f\left( \sqrt{x^2+2bx+c} \right)$, avec $x^2+2bx+c$ irr\'eductible dans $\reels$.

		On a $x^2+2bx+c = (x+b)^2+(c-b^2)$.

		On pose $t = x+b$. On a $dt = dx$.
		\begin{exem}
			\begin{align*}
				\int \dfrac{x}{\sqrt{x^2+4x+5}} dx &= \int \dfrac{x}{\sqrt{(x+2)^2}+1} dx\\
				&\text{posons }t=x+2, dt=dx\\
				&= \int \dfrac{t-2}{\sqrt{t^2+1}} dt\\
				&= \int \dfrac{t}{\sqrt{t^2+1}} dt - 2\int \dfrac{dt}{\sqrt{t^2+1}}\\
				&\text{posons }u=t^2+1, du=2tdt\\
				&\text{posons }t=\sinh v, dt = \cosh v dv\\
				&= \dfrac{1}{2}\int \dfrac{du}{\sqrt{u}} - 2\int \dfrac{\cosh v dv}{\sqrt{\sinh^2v+1}}\\
				&= \sqrt{u} - 2\int dv\\
				&= \sqrt{u} - 2v\\
				&= \sqrt{t^2+1} - 2 \mathrm{arcsinh}t\\
				&= \sqrt{t^2+1} - 2\ln(t+\sqrt{1+t^2})\\
				&= \sqrt{(x+2)^2+1} - 2\ln\left( x+2+\sqrt{1+(x+2)^2} \right)
			\end{align*}
		\end{exem}
	\end{enumerate}

	\section{Int\'egrales impropres}
	\begin{defin}
		$f:[a,\infty[$ continue par morceaux.

		L'int\'egrale impropre (de 1\textsuperscript{\`ere} esp\`ece) de $f$ est $\displaystyle\int_{a}^{\infty}f(x)dx = \lim\limits_{y \to \infty}\displaystyle\int_{a}^{y}f(x)dx$.

		Si la limite existe, on dit que l'int\'egrale converge.
		\begin{exem}~

			\begin{enumerate}
				\item \begin{align*}
					\int_{1}^{\infty}\dfrac{dx}{x}&= \lim_{y \to \infty}\int_{1}^{y}\dfrac{dx}{x}\\
					&= \lim_{y \to \infty}(\ln y - \ln1)\\
					&= \infty
				\end{align*}
				diverge
				\item \begin{align*}
					\int_{1}^{\infty}\dfrac{dx}{x^p}&= \lim_{y \to \infty}\int_{1}^{y}x^{-p}dx\\
					&= \left. \lim_{y \to \infty}\dfrac{x^{-p+1}}{-p+1}\right| ^y_1\\
					&= \lim_{y \to \infty}\left( \dfrac{y^{1-p}}{1-p} - \dfrac{1}{1-p} \right)
				\end{align*}

				Si $p>1$, alors $1-p<0$ et $y^{1-p} \xrightarrow[y \to \infty]{} 0$.

				On a alors $\displaystyle\int_{1}^{\infty}\dfrac{dx}{x^p} = \dfrac{1}{1-p}$.

				Si $p<1$, alors $1-p>0$ et $y^{1-p} \xrightarrow[y \to \infty]{} \infty$.

				On a alors que $\displaystyle\int_{1}^{\infty}\dfrac{dx}{x^p}$ diverge.

				Si $p=1$, c'est le cas 1, qui diverge.
				\item \begin{align*}
					\int_{0}^{\infty}e^{-sx}dx&= \lim_{y \to \infty}\int_{0}^{y}e^{-sx}\\
					&= \left. \lim_{y \to \infty}\dfrac{e^{-sx}}{-s}\right|^y_0\\
					&= \lim_{y \to \infty}\left( \dfrac{e^{-sy}}{-s} + \dfrac{1}{s} \right)
				\end{align*}

				Si $s<0$, alors $-sy>0$ et l'int\'egrale diverge.

				Si $s>0$, alors $-sy<0$ et l'int\'egrale converge vers $\frac{1}{s}$.

				Si $s=0$, alors $\displaystyle\int_{0}^{\infty}e^{-sx}dx = \displaystyle\int_{0}^{\infty}dx$ diverge.
				\item \begin{align*}
					\int_{0}^{\infty}\dfrac{dx}{x^2+1}&= \lim_{y \to \infty}\int_{0}^{\infty}\dfrac{dx}{x^2+1}\\
					&= \lim_{y \to \infty}(\arctan y - \arctan0)\\
					&= \frac{\pi}{2}
				\end{align*}
			\end{enumerate}
		\end{exem}
	\end{defin}
	\begin{defin}
		$f:]a,b]$ continue, mais t.q. $\lim_{x \to a^+}f(x)$ n'existe pas.

		L'int\'egrale impropre (de 2\textsuperscript{\`eme} esp\`ece) de $f$ est $\displaystyle\int_{a}^{b}f(x)dx = \lim\limits_{y \to a^+}\displaystyle\int_{y}^{b}f(x)dx$.

		Si le limite existe, on dit que l'int\'egrale converge.
		\begin{exem}~

			\begin{enumerate}
				\item \begin{align*}
					\int_{0}^{1}\dfrac{dx}{x}&= \lim_{y \to 0^+}\int_{y}^{b}\dfrac{dx}{x}\\
					&= \lim_{y \to 0^+}(\ln1-\ln y)\\
					&= \infty
				\end{align*}
				diverge
				\item \begin{align*}
					\int_{0}^{1}\dfrac{dx}{x^p}&= \lim_{y \to 0^+}\int_{y}^{1}\dfrac{dx}{x^p}\\
					&= \lim_{y \to 0^+}\left. \dfrac{x^{-p+1}}{-p+1} \right|^1_y\\
					&= \dfrac{1}{1-p} - \lim_{y \to 0^+}\dfrac{y^{1-p}}{1-p}
				\end{align*}

				Si $p>1$, alors $1-p<0$ et l'int\'egrale diverge.

				Si $p<1$, alors $1-p>0$ et l'int\'egrale converge vers $\frac{1}{1-p}$.

				Si $p=1$, alors c'est le cas 1, qui diverge.
				\item \begin{align*}
					\int_{0}^{1}\dfrac{dx}{\sqrt{1-x^2}}&= \lim_{y \to 1^-}\int_{0}^{y}\dfrac{dx}{\sqrt{1-x^2}}\\
					&= \lim_{y \to 1^-}(\arcsin y - \arcsin0)\\
					&= \dfrac{\pi}{2}
				\end{align*}
			\end{enumerate}
		\end{exem}
	\end{defin}
	\begin{rema}~

		\begin{enumerate}
			\item \begin{align*}
				\int_{0}^{\infty}\dfrac{dx}{x}&= \overbrace{\int_{0}^{b}\dfrac{dx}{x}}^{\text{2\textsuperscript{\`eme} esp}} + \overbrace{\int_{b}^{\infty}\dfrac{dx}{x}}^{\text{1\textsuperscript{\`ere} esp}}\\
				&= \lim_{y \to 0^+}\int_{y}^{b}\dfrac{dx}{x} + \lim_{y \to \infty}\int_{b}^{y}\dfrac{dx}{x}
			\end{align*}

			$\displaystyle\int_{0}^{\infty}\dfrac{dx}{x}$ converge ssi les deux limites existent.
			\item \begin{align*}
				\int_{-\infty}^{\infty}\sin xdx&= \lim_{y \to \infty}\int_{-y}^{y}\sin xdx\\
				&= \left. \lim_{y \to \infty} \cos x\right|^y_{-y}\\
				&= \lim_{y \to \infty}(\cos y - \cos (-y)) = \lim_{y \to \infty}0\\
				&= 0
			\end{align*}

			Cependant,
			\begin{align*}
				\int_{-\infty}^{\infty}\sin xdx&= \int_{-\infty}^{0}\sin xdx + \int_{0}^{\infty}\sin xdx\\
				&= \lim_{y \to \infty}\int_{-y}^{0}\sin xdx + \lim_{y \to \infty}\int_{0}^{y}\sin xdx\\
				&= \lim_{y \to \infty}(\cos(-y)-\cos y) + \lim_{y \to \infty}(\cos0-\cos y)
			\end{align*}
			diverge

			Ainsi, l'int\'egrale diverge.
		\end{enumerate}
	\end{rema}
\end{document}

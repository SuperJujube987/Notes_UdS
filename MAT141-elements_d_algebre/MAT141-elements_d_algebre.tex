\documentclass{report}
\usepackage[letterpaper,portrait,margin=2cm]{geometry}
\usepackage{titlesec}
\usepackage{amsmath,amssymb,amsthm}
\usepackage{tabularx}
\usepackage{enumitem}
\usepackage{indentfirst}
\usepackage{dsfont}
\usepackage{tikz}
\usepackage{hyperref}
\usepackage{mathtools}
\usepackage{mathrsfs}
\usepackage{enumitem}
\usepackage{wasysym}
\usepackage{fancyhdr}
\usepackage[french]{babel}

\hypersetup{
	colorlinks=false
}

\fancyfoot{}
\fancyhead{}
\fancyfoot[c]{\thepage}
\fancyhead[l]{\leftmark}

\pagestyle{fancy}

\newlist{nlist}{enumerate}{3}
\setlist[nlist, 1]{label=(\arabic*)}
\setlist[nlist, 2]{label=(\alph*)}
\setlist[nlist, 3]{label=(\roman*)}
\newlist{ulist}{itemize}{3}
\setlist[ulist, 1]{label=\textbullet}
\setlist[ulist, 2]{label=|}
\setlist[ulist, 3]{\textperiodcentered}

\title{MAT141 - \'El\'ements d'alg\`ebre

Donn\'e par Jean-Philippe Burelle}
\author{Julien Houle}
\date{Automne 2025}

\newcounter{cours}
\setcounter{cours}{1}
\newcommand*{\cours}{\section*{Cours \thecours}\stepcounter{cours}}

\newcommand*{\abs}[1]{\left| #1 \right|}
\newcommand*{\card}[1]{\left| #1 \right|}

\newcommand*{\reels}{\mathbb{R}}
\newcommand*{\entiers}{\mathbb{Z}}
\newcommand*{\rationels}{\mathbb{Q}}
\newcommand*{\naturels}{\mathbb{N}}

\renewcommand{\thesection}{ \arabic{chapter}.\arabic{section}}
\renewcommand{\thesubsection}{}
\renewcommand{\thesubsubsection}{}

\titleformat{\chapter}[hang]{\bfseries\huge\centering}{Chapitre \arabic{chapter}}{1em}{}[]
\titleformat{\section}[hang]{\bfseries\large}{Section\thesection}{1em}{}[]
\titleformat{\subsection}[hang]{\bfseries\normalsize}{}{0pt}{}[]
\titleformat{\subsubsection}[hang]{\slshape\normalsize}{}{0pt}{}[]

\newtheorem*{thm}{Th\'eor\`eme}
\newtheorem*{lem}{Lemme}
\newtheorem*{prop}{Proposition}
\theoremstyle{definition}
\newtheorem*{defin}{D\'efinition}
\theoremstyle{remark}
\newtheorem*{exem}{Exemple}
\newtheorem*{exer}{Exercice}
\newtheorem*{nota}{Notation}
\newtheorem*{rema}{Remarque}
\newtheorem*{rappel}{Rappel}

\begin{document}
	\maketitle
	\tableofcontents
	\pagenumbering{roman}
	\newpage
	\pagenumbering{arabic}

	\chapter{Ensembles}
	\cours
	Id\'ee: ensemble=patate

	\begin{tikzpicture}
		\path[smooth cycle,very thick,draw=black] plot coordinates {(0,-.3) (-1,-1) (-1.5,-3) (-.8,-3.9) (0,-4) (.8,-3.8) (1.4,-3) (.8,-1.5)};
		\node[font=\Large] () at (0,-1) {$a$};
		\node[font=\Large] () at (-.5,-2) {$b$};
		\node[font=\Large] () at (.5,-3) {$c$};
		\node[anchor=east] () at (-1,-.3) {$E=$};
		\node[anchor=west] () at (1,-1) {$=\{a,b,c\}$};
	\end{tikzpicture}

	\begin{nota}
		$E \subseteq F \Leftarrow \forall x \in E, x \in F$.
		\begin{rema}
			$E \subseteq E$.
		\end{rema}
	\end{nota}

	\begin{nota}
		La cardinalit\'e d'un ensemble, $\card{E}$, est le nombre d'\'el\'ements d'un ensemble.
	\end{nota}

	\begin{defin}
		D\'efinition d'un ensemble par \emph{compr\'ehension}: $E=\left\lbrace n \in \entiers \middle| 1 \leq n \leq 20 \right\rbrace$.
	\end{defin}

	\begin{nota}
		$E=F \Leftrightarrow E \subseteq F$ et $F \subseteq E$.
	\end{nota}

	\begin{defin}
		Produit cart\'esien: $E \times F=\left\lbrace (x,y) \middle| x \in E, y \in F \right\rbrace$.
	\end{defin}

	\begin{defin}
		Fonction/Application

		$f: A \to B$, $A$ et $B$ des ensembles, associe \`a \emph{chaque} $x \in A$ un \emph{unique} \'el\'ement $f(x) \in B$.
	\end{defin}

	\cours
	\begin{rappel}
		~

		\begin{tabularx}{.9\textwidth}{cl>{\raggedright\arraybackslash}X}
			\textbullet&Ensemble&collection d'objets\\
			\textbullet&$\in$&``\'el\'ement'' d'un ensemble\\
			\textbullet&\emph{sous-ensemble} $(\subseteq)$&$E \subseteq F$ si $x \in E$ \emph{implique} $x \in F$\\
			\textbullet&$E=F$&ssi $E \subseteq F$ et $F \subseteq E$\\
			\textbullet&$\cup$&union\\
			&$\cap$&intersection\\
			\textbullet&$E \times F$&produit cart\'esien (paires $(x,y)$)\\
			\textbullet&$f:E \to F$&\emph{fonction} ou \emph{application}, associe \`a \underline{chaque $x \in E$} un unique $\underline{f(x)} \in F$, image de $x$ par $f$\\
			\textbullet&$\mathds{1}$&$\mathds{1}_E:E \to E$ est d\'efinie comme $\mathds{1}_E(x)=x$
		\end{tabularx}
	\end{rappel}


	\subsection{Mani\`eres de d\'efinir une fonction}
	\begin{ulist}[noitemsep]
		\item \'enum\'erer $f(x)$ pour chaque $x \in E$
		\item donner une formule

		une formule ne d\'efinit pas toujours une fonction, elle doit \^etre valide pour chaque $x$ de l'ensemble de d\'epart.
		\item en mots (d\'ecrire la valeur pour chaque $x \in E$)
		\item m\'elange de formule et mots
	\end{ulist}

	\begin{defin}
		Une fonction $f:E \to F$ est \emph{inversible} s'il existe une fonction $\underbrace{g:F \to E}_{*}$ telle que $\underbrace{g(f(x))=x}_{**}$ pour tout $x \in E$ et $\underbrace{f(g(y))=y}_{***}$ pour tout $y \in F$.
	\end{defin}

	\begin{exem}
		$f:\entiers \to \entiers$, $f(x)=x+1$ est inversible d'inverse $g(y)=y-1$

		\begin{proof}[d\'emo.]
			~

			On v\'erifie que
			\begin{align*}
				g(f(x))&= x& g(f(x))&= g(x+1)\\
				&&&= (x+1)-1\\
				&&&= x\\
				f(g(y))&= y& f(g(y))&= f(y-1)\\
				&&&= (y-1)+1\\
				&&&= y
			\end{align*}
		\end{proof}
	\end{exem}

	\begin{prop}
		Si $f$ admet un inverse, celui-ci est unique.

		\begin{proof}[d\'emo.]
			~

			Supposons que $g_1$ et $g_2$ sont tous deux inverses de $f$ et montrons qu'elles sont \'gales.

			(Pour d\'emontrer que deux fonctions sont \'egales, il suffit de montrer que $g_1(y)=g_2(y)$ pour tout $y \in F$)

			Soit $y \in F$.

			On a
			\begin{align*}
				g_1(y)&\overbrace{=}^{***} g_1(\underbrace{f(g_2(y))}_{*})\\
				&\overbrace{=}^{**} g_2(y)
			\end{align*}
		\end{proof}
	\end{prop}

	\begin{defin}
		Si $f:E \to F$ et $g:F \to G$, alors la compos\'ee de $f$ et $g$ est la fonction $g \circ f:E \to G$ d\'efinie par la formule $g \circ f(x)=g(f(x))$.
	\end{defin}

	\begin{defin}[Red\'efinition de l'inverse]
		$g \circ f=\mathds{1}_E$

		$f \circ g=\mathds{1}_F$
	\end{defin}

	\begin{exem}
		$A=\{a,b,c\}$

		$B=\{d,e,f\}$

		$f:A \to B$, $a \mapsto d, b \mapsto e, c \mapsto f$

		$g:B \to A$, $d \mapsto a, e \mapsto b, f \mapsto c$

		$g \circ f:A \to A$, $g \circ f(x)=x$, $g \circ f=\mathds{1}_A$.

		De la \^m mani\`ere, $f \circ g=\mathds{1}_B$.

		~

		Ainsi, $g$ est l'inverse de $f$.
	\end{exem}

	\begin{nota}
		On note $g=f^{-1}$ l'inverse de $f$.
	\end{nota}


	\begin{rappel}
		Pour trouver l'inverse d'une fonction $f:\reels \to \reels$ donn\'ee par une formule $f(x)=y$, on isole $x$ en fonction de $y$.
	\end{rappel}


	\begin{exem}
		\begin{align*}
			f(x)&= 3x-8\\
			y&= 3x-8\\
			y+8&= 3x\\
			\dfrac{y+8}{3}&= x\\
			g(y)&= \dfrac{y+8}{3}
		\end{align*}

		Dans un devoir, on commence par la formule de l'inverse et on v\'erifie $g(f(x))=x$ et $f(g(y))=y$.
	\end{exem}

	\begin{defin}
		On dit que $f:E \to F$ est une fonction \emph{injective} si $f(x_1)=f(x_2)$ implique $x_1=x_2$.
	\end{defin}

	\begin{defin}
		On dit que $f:E \to F$ est une fonction \emph{surjective} si pour tout $y \in F,~\exists~x \in E$ t.q. $f(x)=y$.
	\end{defin}

	\begin{defin}
		On dit que $f:E \to F$ est une fonction \emph{bijective} si elle est injective \textbf{et} surjective.
	\end{defin}

	\begin{exem}
		$f:\reels \to \reels^{\geq0}$, $f(x)=\abs{x}$

		$f$ n'est pas injective, car $f(1)=\abs{1}=1$ et $f(-1)=\abs{-1}=1$, mais $1 \neq -1$.

		$f$ est surjective, car soit $y \in \reels^{\geq0}$, alors pour $x=y$, on a $f(x)=f(y)=\abs{y}=y$.

		~

		\[
		\begin{array}{rrcl}
			f:&\naturels&\to&\naturels\\
			&x&\mapsto&x+1
		\end{array}
		\]

		$f$ est injective:

		Soient $x_1,x_2 \in \naturels$.

		On suppose $f(x_1)=f(x_2)$.

		\begin{align*}
			x_1+1&= x_2+1\\
			x_1&= x_2
		\end{align*}

		$f$ n'est pas surjective

		$y=0\in\naturels$ n'est pas \'egal \`a $f(x)$ pour $x\in\naturels$. Si il existait $x$ avec $f(x)=0$, $x+1=0$, $x=-1$, $x \not\in \naturels$.

		~

		$f:\reels\to\reels$, $x \mapsto 2x+3$.

		$f$ est injective:

		Soient $x_1,x_2 \in \reels$.

		supposons $f(x_1)=f(x_2)$, $2x_1+3=2x_2+3$, $2x_1=2x_2$, $x_1=x_2$.

		$f$ est surjective:

		Soit $y \in \reels$.

		On cherche $x$ t.q. $f(x)=y$.

		Posons $x=\dfrac{y-3}{2} \in \reels$.

		Alors, $f(x) = f\left( \dfrac{y-3}{2} \right) = 2 \cdot  \dfrac{y-3}{2} + 3 = y-3+3 = y$.

		Ainsi, $f$ est bijective.

		~

		$f:A \to B$, avec $A=\{1,48,57\}$ et $B=\{a,b,c\}$.

		$1 \mapsto a$, $48 \mapsto a$, $57 \mapsto b$.

		$f$ n'est pas injective, car $1 \mapsto a$ et $48 \mapsto a$  avec $1 \neq 48$.

		$f$ n'est pas surjective, car aucun \'el\'ement de $x \in A \mapsto c$.
	\end{exem}

	\begin{rema}
		La fonction $f':A \to B'$ avec $B'=\{a,b\}$ est surjective.
	\end{rema}


	\cours
	\begin{rappel}
		$A, B$ deux ensembles

		\begin{ulist}[noitemsep]
			\item $f:A \to B$ une fonction, associe \`a chaque $x \in A$ un unique $f(x) \in B$. $x \mapsto f(x)$.
			\item $f$ est \emph{inversible} s'il existe $g:B \to A$ t.q. $g(f(a))=a$ pour tout $a \in A$ et $f(g(b))=b$ pour tout $b \in B$.
			\item l'inverse est \emph{unique}.
			\item La composition de $f:A \to B$ avec $g:B \to C$ est $g \circ f:A \to C$ avec $(g \circ f)(a)=g(f(a))$.
			\item $f$ est injective si $f(x_1)=f(x_2) \Rightarrow x_1=x_2$.
			\item $f$ est surjective si pour tout $b \in B$ il existe $a \in A$ t.q. $f(a)=b$.
			\item $f$ est bijective si elle est injective et surjective.
		\end{ulist}
	\end{rappel}


	\begin{prop}
		$f:A \to B$ est bijective \emph{ssi} elle est inversible.

		\begin{proof}[d\'emo]
			~

			$\Leftarrow$:

			Supposons que $f$ est inversible.

			Alors, il existe un inverse $g:B \to A$.

			(inj): Soient $x_1,x_2 \in A$.

			On suppose que $f(x_1)=f(x_2)$.

			Alors, $g(f(x_1)) = g(f(x_2))$

			Donc, $x_1 = x_2$

			(surj): Soit $y \in B$.

			Posons $x=g(y) \in A$.

			Alors, $f(x)=f(g(y))=y$.

			$\Rightarrow$:

			Supposons $f$ est injective et surjective.

			\begin{lem}
				Pour \emph{chaque} $y \in B$, il existe un \emph{unique} $x \in A$ t.q. $f(x)=y$.

				\renewcommand{\qedsymbol}{$\blacksquare$}
				\begin{proof}[d\'emo]
					~

					\underline{Existance}:
					Comme $f$ est surjective, $x$ existe.

					\underline{Unicit\'e}:
					Supposons $x_1,x_2 \in A$ t.q. $f(x_1)=f(x_2)$, alors $x_1=x_2$.
				\end{proof}
				\renewcommand{\qedsymbol}{$\square$}
			\end{lem}

			On d\'efinit $g:B \to A$ par $g(y)=x$ o\`u $x$ est l'unique \'el\'ement du lemme.

			On v\'erifie:

			Soit $x \in A$, alors $g(\underbrace{f(x)}_y)=x$, par d\'efinition de $g$.

			Soit $y \in B$, alors $f(\underbrace{g(y)}_{\text{l'unique $x$ t.q. $f(x)=y$}})=y$.
		\end{proof}
	\end{prop}

	\begin{defin}
		Une \emph{op\'eration} (interne, binaire) sur un ensemble $E$ est un fonction $m:E \times E \to E$.
	\end{defin}

	\begin{exem}
		$E=\entiers$,

		$\begin{array}{rrcl}
			m:&\entiers \times \entiers &\longrightarrow& \entiers\\
			&(n,m)&\longmapsto&n + m
		\end{array}$

		$\begin{array}{rrcl}
			m:&\entiers \times \entiers &\longrightarrow& \entiers\\
			&(n,m)&\longmapsto&n \cdot m
		\end{array}$

		~

		$\begin{array}{rrcl}
			d:&\rationels \times \rationels&\longrightarrow&\rationels\\
			&(x,y)&\longmapsto&\frac{x}{y}
		\end{array}$

		n'est pas une op\'eration, car $(1,0) \mapsto \frac{1}{0}$ qui n'est pas d\'efini. ($d$ n'est pas une fonction.)

		Cependant,

		$\begin{array}{rrcl}
			d:&\rationels_* \times \rationels_*&\longrightarrow&\rationels_*\\
			&(x,y)&\longmapsto&\frac{x}{y}
		\end{array}$

		est une op\'eration.

		~

		$A$ un ensemble

		$E=\{f:A \to A\}$, o\`u $f$ est une fonction.

		$\begin{array}{rrcl}
			c:&E \times E&\longrightarrow&E\\
			&(f,g)&\longmapsto&f \circ g
		\end{array}$

		La composition est une op\'eration.
	\end{exem}

	\begin{nota}
		On note la plupart du temps une op\'eration par un symbole entre les entr\'ees.
	\end{nota}

	\begin{exem}
		$m(x,y) \coloneq x*y$, ou $x+y$, ou $x \circ y$, ou $xy$
	\end{exem}

	\begin{defin}~

		Un \emph{\'el\'ement neutre} pour une op\'eration $*$ est un \'el\'ement $e \in E$ t.q. pour tout $x \in E$, $e*x=x$ et $x*e=x$.
	\end{defin}

	\cours
	\begin{rappel}
		~

		\begin{ulist}[noitemsep]
			\item $f:E \to F$ est bijective $\Leftrightarrow$ $f$ est inversible.
			\item L'inverse est unique $(g=f^{-1})$
			\item Op\'eration: $m:E \times E \to E$, ou
			$\begin{array}{rrcl}
				*:&E \times E&\to&E\\
				&(x,y)&\mapsto&z
			\end{array}$
			\item \'El\'ement neutre: $e \in E$ t.q. $e*x=x$ et $x*e=x$.
			\item $f$ est injective si tout $y \in F$ a au plus un ant\'ec\'edent
			\item $f$ est surjective si tout $y \in F$ a au moins un ant\'ec\'edent
			\item $f$ est bijective si tout $y \in F$ a exactement un ant\'ec\'edent
			\item $x$ est ant\'ec\'edent de $y$ si $f(x)=y$
		\end{ulist}
	\end{rappel}

	\begin{exem}
		Sur $\naturels$,

		\begin{ulist}
			\item $0$ est neutre pour $+$.
			\begin{align*}
				0+n&=n\\
				n+0&=n
			\end{align*}
			\item $1$ est neutre pour $\times$.
			\begin{align*}
				1 \times n&=n\\
				n \times 1&=n
			\end{align*}
		\end{ulist}

		Sur $\entiers$, $-$ est une op\'eration mais elle n'a pas d\'el\'ement neutre.
		\renewcommand{\qedsymbol}{\lightning}
		\begin{proof}[En effet,]~

			Supposons que $e \in \entiers$ est neutre, alors $e-n=n$ pour tout $n$.

			Pour $n=0$, $e-0=0$, donc $e=0$.

			Pour $n=1$, $e-1=1$, donc $-1=1$.
		\end{proof}
		\renewcommand{\qedsymbol}{$\square$}
		\begin{ulist}
			\item Sur l'ensemble $E=\left\lbrace \left( \begin{array}{cc}
				a&b\\c&d
			\end{array}\right) \middle| a,b,c,d \in \reels\right\rbrace$, la multiplication matricielle $\times$ est une op\'eration.

			La matrice $I=\left( \begin{array}{cc}
				1&0\\0&1
			\end{array}\right)$ est neutre pour $\times$.
			\item Sur $E=\{f:A \to A\}$, la fonction $\mathds{1}_A$ est neutre pour la composition de fonctions.
			\begin{proof}[d\'emonstration]~

				On doit montrer $\mathds{1}_A \circ f=f$ et $f \circ \mathds{1}_A=f$ pour tout $f \in E$.
				\begin{nlist}
					\item Soit $x \in A$, alors
					\begin{align*}
						(\mathds{1}_A \circ f)(x)&= \mathds{1}_A(f(x))\\
						&= f(x)
					\end{align*}

					Donc, $\mathds{1}_A \circ f=f$.
					\item Soit $x \in A$, alors
					\begin{align*}
						(f \circ \mathds{1}_A)(x)&= f(\mathds{1}_A(x))\\
						&=f(x)
					\end{align*}

					Donc,$f \circ \mathds{1}_A=f$.
				\end{nlist}
			\end{proof}
		\end{ulist}
	\end{exem}

	On peut d\'ecrire une op\'eration sur un ensemble fini avec sa table ``de multiplication''.
	\begin{exem}
		$A=\{0,1\}$

		$f_1:\begin{array}{rcl}
			0&\mapsto&0\\
			1&\mapsto&0
		\end{array},f_2:\begin{array}{rcl}
			0&\mapsto&0\\
			1&\mapsto&1
		\end{array},f_3:\begin{array}{rcl}
			0&\mapsto&1\\
			1&\mapsto&0
		\end{array},f_4:\begin{array}{rcl}
			0&\mapsto&1\\
			1&\mapsto&1
		\end{array}$
		On a $f_2=\mathds{1}_A$.

		\[
		\begin{array}{c||c|c|c|c}
			\circ&f_1&f_2&f_3&f_4\\
			\hline\hline
			f_1&f_1&f_1&f_1&f_1\\
			\hline
			f_2&f_1&f_2&f_3&f_4\\
			\hline
			f_3&f_4&f_3&f_2&f_1\\
			\hline
			f_4&f_4&f_4&f_4&f_4
		\end{array}
		\]
	\end{exem}
	\begin{defin}
		~

		Une op\'eration $*$ sur $E$ est \emph{associative} si pour tout $x,y,z \in E$, on a $(x*y)*z=x*(y*z)$.
	\end{defin}
	\begin{prop}
		~

		Si $*$ admet un \'el\'ement neutre, alors celui-ci est \emph{unique}.
		\begin{proof}[d\'emonstration]~

			Supposons que $e$ et $e'$ sont neutres pour $*$.

			On a
			\begin{align*}
				e*e'&=e'&&\text{car $e$ est neutre}\\
				e*e'&=e&&\text{car $e'$ est neutre}
			\end{align*}

			Donc, $e=e'$.
		\end{proof}
	\end{prop}
	\begin{defin}
		~

		Soit $E$ un ensemble, $*$ une op\'eration sur $E$ et $e \in E$ un neutre pour $*$. On dit que $a,b \in E$ sont \emph{inverses} si $a*b=e$ et $b*a=e$.

		Dans ce cas, on dit que $a$ et $b$ sont inversibles.
	\end{defin}
	\begin{exem}
		~

		Dans $\entiers$ avec $+$, $3$ et $-3$ sont inverses. En effet, on a $3+(-3)=0$ et $(-3)=3=0$ avec $0$ l'\'el\'ement neutre de $+$.
	\end{exem}
	\begin{exem}
		~

		Dans $\entiers$ avec $\times$, le neutre est 1, mais seuls $1$ et $-1$ sont inversibles. En effet, on a $1 \times 1=1$ et $(-1) \times (-1)=1$.
	\end{exem}
	\begin{rema}
		~

		L'\'el\'ement neutre est son propre inverse. En effet, $e*e=e$, pour tout $*$ qui admet $e$ comme \'el\'ement neutre.
	\end{rema}
	\begin{prop}
		~

		Si $*$ est associative et admet un \'el\'ement neutre $e$, alors les inverses sont uniques s'ils existent.
		\begin{proof}[d\'emonstration]~

			Soit $a \in E$.

			Supposons $b,b'$ sont inverses de $a$.

			Alors,
			\begin{align*}
				b&=b*e\\
				&=b*(a*b')&&\text{car $b'$ est inverse de $a$}\\
				&=(b*a)*b'&&\text{associativit\'e}\\
				&=e*b'&&\text{car $b$ est inverse de $a$}\\
				&=b'
			\end{align*}
		\end{proof}
		\begin{nota}
			~

			Comme l'inverse de $a$ est unique, on le note $a^{-1}$.
		\end{nota}
	\end{prop}
	\begin{exem}
		~

		Dans $E=\{f:A \to A\}$, avec l'op\'eration $\circ$, les fonctions bijectives sont exactement celles qui sont inversibles pour $\circ$.
	\end{exem}
	\begin{prop}
		~

		La composition de fonctions est associative.
		\begin{proof}[d\'emonstration]~

			Soient $f:A \to B$, $g:B \to C$ et $h:C \to D$.

			Soit $a \in A$.
			\begin{align*}
				((h \circ g) \circ f)(a)&= (h \circ g)(f(a))\\
				&= h(g(f(a)))\\
				&= h((g \circ f)(a))\\
				&= (h \circ (g \circ f))(a)\\
				(h \circ g) \circ f&= h \circ (g \circ f)
			\end{align*}
		\end{proof}
	\end{prop}
	\setcounter{chapter}{5}
	\chapter{Groupes}
	\begin{defin}
		~

		Un \emph{groupe} est un ensemble $G$ muni d'une op\'eration $*$ t.q.
		\begin{itemize}
			\item[(A)] $*$ est associative
			\item[(N)] $*$ admet un neutre
			\item[(I)] tout $g \in G$ admet un inverse
		\end{itemize}
	\end{defin}
	\begin{exem}
		~

		\begin{nlist}
			\item $(\entiers,+)$ est un groupe.

			Neutre: $0$

			Inverse de $n$: $-n$
			\item $(\rationels,+)$ et $(\reels,+)$ sont des groupes.
			\item $(\entiers,\times)$ n'est pas un groupe, car, par exemple, $2$ n'est pas inversible.
			\item $(\rationels,\times)$ n'est pas un groupe, car $0$ n'est pas inversible.
			\item $(\rationels_*,\times)$ et $(\reels,\times)$ sont des groupes.

			Neutre: $1$

			Inverse de $x$: $\frac{1}{x}$
		\end{nlist}
		\begin{rema}
			~

			$(1)$, $(2)$ et $(5)$ sont \emph{commutatifs}.
		\end{rema}
		\begin{rema}
			~

			$(\naturels,+)$ n'est pas un groupe.
		\end{rema}
		\begin{defin}
			~

			Si l'op\'eration d'un groupe est commutative, on note le groupe comme \emph{ab\'elien} (ou commutatif).
		\end{defin}
		\begin{nlist}[resume]
			\item $GL(n,\reels)$ est un groupe pour la multiplication matricielle.

			$GL(n,\reels)=\{M|M \text{ est une matrice $n\times n$ r\'eelle inversible}\}$.

			$GL$: g\'en\'eral lin\'eaire

			Neutre: $\left( \begin{array}{ccc}
				1&\dotsb&0\\
				\vdots&\ddots&\vdots\\
				0&\dotsb&1
			\end{array} \right)$.

			$M^{-1}$ la matrice inverse est l'inverse.

			Pour $n \geq 2$, $GL(n,\reels)$  n'est pas ab\'elien.
			\item $A$ un ensemble quelconque

			$S(A)=\{f:A \to A|f \text{ est bijective}\}$ est un groupe pour $\circ$.

			Neutre: $\mathds{1}_A$

			Inverse de $f$: $f^{-1}$
		\end{nlist}
	\end{exem}
	\begin{rema}
		~

		Pour $A=\{0,1\}$

		$f_1:\begin{array}{rcl}
			0&\mapsto&0\\
			1&\mapsto&0
		\end{array},f_2:\begin{array}{rcl}
			0&\mapsto&0\\
			1&\mapsto&1
		\end{array},f_3:\begin{array}{rcl}
			0&\mapsto&1\\
			1&\mapsto&0
		\end{array},f_4:\begin{array}{rcl}
			0&\mapsto&1\\
			1&\mapsto&1
		\end{array}$, $S(A)=\{f_2,f_3\}$.
	\end{rema}

	\cours
	\begin{rappel}
		~

		\begin{ulist}[noitemsep]
			\item Groupe: $(G,*)$
			\begin{itemize}
				\item[$G$] ensemble
				\item[$*$] op\'eration sur $G$
				\item[$(A)$] $*$ est associative

				$\forall a,b,c \in G, (a*b)*c=a*(b*c)$
				\item[$(N)$] $*$ admet un \'el\'ement neutre dans $G$

				$\exists e \in G$ t.q. $\forall a \in G, e*a=a=a*e$
				\item[$(I)$] tout \'el\'ement de $G$ est inversible

				$\forall a \in G, \exists b \in G$ t.q. $a*b=e=b*a$
			\end{itemize}
			\item Le neutre et l'inverse sont uniques
		\end{ulist}
	\end{rappel}

	\begin{rema}
		~

		``Le groupe $\reels$'' implique l'op\'eration $+$ et ``le groupe $\reels_*$'' implique l'op\'eration $\times$.
	\end{rema}

	\subsection{Propri\'et\'es \'el\'ementaires des groupes}
	\begin{nlist}
		\item $\forall a,b \in G$, $(a*b)^{-1} = b^{-1}*a^{-1}$.
		\item $\forall a \in G$, $(a^{-1})^{-1}=a$
		\item Si $a*b=a*c$, alors $b=c$
		\item Si $b*a=c*a$, alors $b=c$
	\end{nlist}
	\begin{proof}[d\'emonstration]~

		\begin{nlist}
			\item On calcule
			\begin{align*}
				\begin{split}
					(a*b)*(b^{-1}*a^{-1})&= a*(b*(b^{-1}*a^{-1}))\\
					&= a*((b*b^{-1})*a^{-1})\\
					&= a*(e*a^{-1})\\
					&= a*a^{-1}\\
					&= e
				\end{split}
				&
				\begin{split}
					(b^{-1}*a^{-1})*(a*b)&= b^{-1}*(a^{-1}*(a*b))\\
					&= b^{-1}*((a^{-1}*a)*b)\\
					&= b^{-1}*(e*b)\\
					&= b^{-1}*b\\
					&= e
				\end{split}
			\end{align*}

			Donc, $(a*b)^{-1} = b^{-1}*a^{-1}$.
			\item Comme $a^{-1}*a=e=a*a^{-1}$, $a$ est l'inverse de $a^{-1}$, donc $\left( a^{-1} \right)^{-1}=a$.
			\item Supposons $a*b=a*c$. Alors
			\begin{align*}
				a^{-1}*(a*b)&= a^{-1}*(a*c)\\
				(a^{-1}*a)*b&= (a^{-1}*a)*c\\
				e*b&= e*c\\
				b&= c
			\end{align*}
			\item Supposons $b*a=c*a$. Alors
			\begin{align*}
				(b*a)*a^{-1}&= (c*a)*a^{-1}\\
				b*(a*a^{-1})&= c*(a*a^{-1})\\
				b*e&= c*e\\
				b&= c
			\end{align*}
		\end{nlist}
	\end{proof}
	\begin{exem}
		~

		$\left( \entiers_3,+ \right)$.

		$\entiers_3=\left\lbrace \overline{0}, \overline{1}, \overline{2} \right\rbrace$

		\renewcommand{\arraystretch}{1.5}
		\[
		\begin{array}{c||c|c|c}
			+&\overline0&\overline1&\overline2\\
			\hline\hline
			\overline0&\overline0&\overline1&\overline2\\
			\hline
			\overline1&\overline1&\overline2&\overline3\\
			\hline
			\overline2&\overline2&\overline0&\overline1
		\end{array}
		\]
		\renewcommand{\arraystretch}{1}

		$+$ est associative.

		$\overline{0}$ est l'\'el\'ement neutre.

		$(\overline{1})^{-1}=\overline{2}$.

		$(\overline{2})^{-1}=\overline{1}$.

		$\left( \entiers_3,+ \right)$ est un groupe ab\'elien.
		\begin{rema}
			La sym\'etrie de la table par rapport \`a la diagonale implique la commutativit\'e.
		\end{rema}
	\end{exem}
	\begin{exem}
		~

		$\left( \mathbb{D}_3, \circ \right)$ - groupe dih\'edral d'ordre 3.

		Groupe des sym\'etries d'un triangle \'equilat\'eral.

		$\mathbb{D}_3=\left\lbrace \underset{\text{identit\'e}}{\varepsilon}, \underset{\text{r\'eflexion par rapport \`a la verticale}}{\alpha}, \underset{\text{r\'eflexion par rapport \`a /}}{\beta}, \underset{\text{r\'eflexion par rapport \`a \textbackslash}}{\gamma}, \underset{\text{rotation de 120\textdegree}}{\rho}, \underset{\text{rotation de -120\textdegree}}{\sigma} \right\rbrace$.

		\[
		\begin{array}{c||c|c|c|c|c|c}
			\circ&\varepsilon&\alpha&\beta&\gamma&\rho&\sigma\\
			\hline\hline
			\varepsilon&\varepsilon&\alpha&\beta&\gamma&\rho&\sigma\\
			\hline
			\alpha&\alpha&\varepsilon&\rho&\sigma&\beta&\gamma\\
			\hline
			\beta&\beta&\sigma&\varepsilon&\rho&\gamma&\alpha\\
			\hline
			\gamma&\gamma&\rho&\sigma&\varepsilon&\alpha&\beta\\
			\hline
			\rho&\rho&\gamma&\alpha&\beta&\sigma&\varepsilon\\
			\hline
			\sigma&\sigma&\beta&\gamma&\alpha&\varepsilon&\rho
		\end{array}
		\]

		$\left( \mathbb{D}_3, \circ \right)$ n'est pas un groupe ab\'elien.
	\end{exem}

	\cours
	\begin{rappel}~

		\begin{ulist}[noitemsep]
			\item Groupe: $(G, *)$ avec $A, N, I$.

			Ab\'elien: $C$.
			\item
			\begin{align*}
				a*b&=a*c&\Rightarrow b&=c\\
				b*a&=c*a&\Rightarrow b&=c\\
				(a^{-1})^{-1}&= a\\
				(a*b)^{-1}&= b^{-1}*a^{-1}
			\end{align*}
			\item
			\begin{exem}~

				$(\entiers, +), (\rationels, +), (\reels, +), (\rationels_*, \times), (\reels_*, \times)$ ab\'eliens, $\entiers_3, \mathbb{D}_3, GL(n, \reels)$.

				$S(E)=\{f:E \to E \mid f\text{ est bijective}\}$.
				\begin{rema}
					$E$ n'est pas l'ensemble utilis\'e dans la d\'efinition du groupe.
				\end{rema}
			\end{exem}
		\end{ulist}
	\end{rappel}
	\subsection{Produit cart\'esien de groupes}
	$(G, *)$ et $(H, \diamond)$ deux groupes.

	\begin{prop}
		~

		$G \times H$ est un groupe lorsque muni de l'op\'eration $(a,b) \bullet (a',b') = (a*a',b \diamond b')$, avec $a,a' \in G$ et $b,b' \in H$.
		\begin{proof}[d\'emonstration]~

			\begin{itemize}
				\item[(N)] $e \in G$ le neutre et $e' \in H$ le neutre, alors $(e,e') \in G \times H$

				\begin{align*}
					(a,b) \bullet (e,e')&= (a*e, b \diamond e')\\
					&= (a,b)\\
					(e,e') \bullet (a,b)&= (e*a, e' \diamond b)\\
					&= (a,b)
				\end{align*}

				$(e,e')$ est bien neutre.
				\item[(I)] $(a,b) \in G \times H$, alors $(a^{-1},b^{-1})$ est inverse de $(a,b)$.

				exercice
				\item[(A)] exercice
			\end{itemize}
		\end{proof}
	\end{prop}
	\begin{exem}
		\begin{ulist}
			\item $\reels \times \reels = \reels^2$

			$(x,y)+(x',y')=(x+x',y+y')$.
			\item $(\entiers_2,+)$
			\renewcommand{\arraystretch}{1.5}
			\[
			\begin{array}{c||c|c}
				+&\overline0&\overline1\\
				\hline\hline
				\overline0&\overline0&\overline1\\
				\hline
				\overline1&\overline1&\overline0
			\end{array}
			\]

			$\entiers_2 \times \entiers_2$
			\[
			\begin{array}{c||c|c|c|c}
				+&(\overline0,\overline0)& (\overline0,\overline1)& (\overline1,\overline0)& (\overline1,\overline1)\\
				\hline\hline
				(\overline0,\overline0)& (\overline0,\overline0)& (\overline0,\overline1)& (\overline1,\overline0)& (\overline1,\overline1)\\
				\hline
				(\overline0,\overline1)& (\overline0,\overline1)& (\overline0,\overline0)& (\overline1,\overline1)& (\overline1,\overline0)\\
				\hline
				(\overline1,\overline0)& (\overline1,\overline0)& (\overline1,\overline1)& (\overline0,\overline0)& (\overline0,\overline1)\\
				\hline
				(\overline1,\overline1)& (\overline1,\overline1)& (\overline1,\overline0)& (\overline0,\overline1)& (\overline0,\overline0)
			\end{array}
			\]
			\renewcommand{\arraystretch}{1}
		\end{ulist}
	\end{exem}
	\subsection{Isomorphismes de groupes}
	\begin{defin}
		$(G,*)$ et $(H,\diamond)$ deux groupes.

		Un \emph{isomorphisme} de $G$ vers $H$ est une application $f:G \to H$ t.q.
		\begin{nlist}[noitemsep]
			\item $\forall a,b \in G$, $f(a*b)=f(a)\diamond f(b)$.

			Pr\'eservation des op\'erations
			\item $f$ est bijective.
		\end{nlist}
	\end{defin}
	\begin{exem}~

		\begin{ulist}
			\item $(\reels,+)$ et $(\reels^+_*,\times)$

			$\begin{array}{rcl}
				f:\reels&\to&\reels^+_*\\
				x&\mapsto&e^x
			\end{array}$ est un isomorphisme de groupes.
			\begin{nlist}
				\item Soient $x,y \in \reels$.
				\begin{align*}
					f(x+y)&= e^{x+y}\\
					&= e^x \times e^y\\
					&= f(x) \times f(y)
				\end{align*}
				\item $\ln:\reels^+_* \to \reels$ est inverse de $f$: $\ln e^x=x \forall x \in \reels$ et $e^{\ln x}=x \forall x \in \reels^+_*$.
			\end{nlist}
		\end{ulist}
	\end{exem}
	\begin{prop}
		Si $f:G \to H$ est un isomorphisme de groupes, alors $f(e_G)=e_H$, o\`u $e_G$ est l'\'el\'ement neutre de $G$ et $e_H$ est l'\'el\'ement neutre de $H$.
		\begin{proof}[d\'emonstration]
			Strat\'egie: montrer que $f(e_G)$ est neutre pour $H$ et utiliser l'unicit\'e.

			Soit $b \in H$.

			Comme $f$ est bijective, $\exists a \in G$ t.q. $f(a)=b$
			\begin{align*}
				\begin{split}
					f(e_G) \diamond b&= f(e_G) \diamond f(a)\\
					&= f(e_G*a)\\
					&= f(a)\\
					&=b
				\end{split}
				&
				\begin{split}
					b \diamond f(e_G)&= f(a) \diamond f(e_G)\\
					&= f(a*e_G)\\
					&= f(a)\\
					&= b
				\end{split}
			\end{align*}

			On a donc que $f(e_G) \in H$ est neutre pour $\diamond$, mais comme l'\'el\'ement neutre est unique, $f(e_G)=e_H$.
		\end{proof}
	\end{prop}
	\begin{exem}
		Pour $\begin{array}{rcl}
			f:\reels&\to&\reels^+_*\\
			x&\mapsto&e^x
		\end{array}$, $f(0)=e^0=1$.
	\end{exem}
	\begin{prop}
		Si $f:G \to H$ est un isomorphisme de groupes, alors $f(a^{-1})=f(a)^{-1}$, pour tout $a \in G$.
		\begin{proof}[d\'emonstration]
			Strat\'egie: montrer que $f(a^{-1})$ est inverse de $f(a)$ et utiliser l'unicit\'e.

			\begin{align*}
				\begin{split}
					f(a^{-1}) \diamond f(a)&= f(a^{-1}*a)\\
					&= f(e_G)\\
					&= e_H
				\end{split}
				&
				\begin{split}
					f(a) \diamond f(a^{-1})&= f(a*a^{-1})\\
					&= f(e_G)\\
					&= e_H
				\end{split}
			\end{align*}

			On a donc que $f(a^{-1})$ est inverse de $f(a)$, mais comme l'inverse est unique, $f(a^{-1})=f(a)^{-1}$.
		\end{proof}
	\end{prop}
	\begin{exem}
		Pour $\begin{array}{rcl}
			f:\reels&\to&\reels^+_*\\
			x&\mapsto&e^x
		\end{array}$, $f(-x)=e^{-x}=(e^x)^{-1}=f(x)^{-1}=\frac{1}{f(x)}$, o\`u $-x$ est l'inverse de $x$ pour $+$ et $\frac{1}{f(x)}$ est l'inverse de $f(x)$ pour $\times$.
	\end{exem}
	\begin{rema}
		Si $G,H$ sont des groupes finis et $f$ est un isomorphisme, alors $f$ ``envoie la table de $G$ \`a celle de $H$''.

		\[
		G:
		\begin{array}{c||c|c|c|c}
			*&e_G&a_1&a_2&\dotsb\\
			\hline\hline
			e_G&&&&\\
			\hline
			a_1&&&a_1*a_2&\\
			\hline
			a_2&&&&\\
			\hline
			\vdots&&&&
		\end{array}
		\xrightarrow{~~~f~~~}
		\begin{array}{c||c|c|c|c}
			*&e_H&f(a_1)&f(a_2)&\dotsb\\
			\hline\hline
			f(e_G)&&&&\\
			\hline
			f(a_1)&&&f(a_1) \diamond f(a_2)&\\
			\hline
			f(a_2)&&&&\\
			\hline
			\vdots&&&&
		\end{array}
		:H
		\]
		Avec $f(a_1*a_2)=f(a_1) \diamond f(a_2)$.
		\begin{exem}
			\[
			\entiers_2:
			\renewcommand{\arraystretch}{1.5}
			\begin{array}{c||c|c}
				+&\overline0&\overline1\\
				\hline\hline
				\overline0&\overline0&\overline1\\
				\hline
				\overline1&\overline1&\overline0
			\end{array}
			\renewcommand{\arraystretch}{1}
			\qquad
			H:
			\begin{array}{c||c|c}
				\circ&\varepsilon&\alpha\\
				\hline\hline
				\varepsilon&\varepsilon&\alpha\\
				\hline
				\alpha&\alpha&\varepsilon
			\end{array}
			\qquad
			C_2:
			\begin{array}{c||c|c}
				\times&1&-1\\
				\hline\hline
				1&1&-1\\
				\hline
				-1&-1&1
			\end{array}
			\]
			$\entiers_2$, $H$ et $C_2$ sont isomorphes.

			Il existe un isomorphisme entre chaque paire.
		\end{exem}
	\end{rema}
	\begin{prop}
		Si $f:G \to H$ est un isomorphisme, alors $f^{-1}:H \to G$ est un isomorphisme.
		\begin{proof}[d\'emonstration]~

			\begin{nlist}
				\item Soient $b_1,b_2 \in H$.
				\begin{align*}
					f^{-1}(b_1 \diamond b_2)&= f^{-1}(f(f^{-1}(b_1)) \diamond f(f^{-1}(b_2)))\\
					&= f^{-1}(f(f^{-1}(b_1) * f^{-1}(b_2)))\\
					&= f^{-1}(b_1) * f^{-1}(b_2)
				\end{align*}
				\item $f^{-1}$ est bijective, car elle est inversible d'inverse $f$.
				\begin{align*}
					f \circ f^{-1}&= \mathds{1}_H\\
					f^{-1} \circ f&= \mathds{1}_G
				\end{align*}
			\end{nlist}
		\end{proof}
	\end{prop}
	\begin{prop}[Transitivit\'e]~

		Si $f:G \to H$ et $g:H \to K$ sont des isomorphismes, alors $g \circ f:G \to K$ est un isomorphisme.
		\begin{proof}[d\'emonstration]~

			\begin{nlist}
				\item Soient $a,b \in G$
				\begin{align*}
					(g \circ f)(a*b)&= g(f(a*b))\\
					&= g(f(a) \diamond f(b))\\
					&= g(f(a)) \oplus g(f(b))\\
					&= (g \circ f)(a) \oplus (g \circ f)(b)
				\end{align*}
				\item $g \circ f$ est inversible d'inverse $f^{-1} \circ g^{-1}$.
			\end{nlist}
		\end{proof}
	\end{prop}

	\subsection{Puissances d'\'el\'ements de groupes}
	\begin{defin}[par r\'ecurrence]~

		$a \in G$, $n \in \naturels$
		\begin{nlist}
			\item $a^0 \coloneq e_G$
			\item $a^n=a*a^{n-1}$, $\forall n \geq 1$
		\end{nlist}
	\end{defin}
	\begin{exem}~

		\begin{ulist}
			\item \begin{align*}
				a^4&= a*a^3\\
				&= a*a*a*2\\
				&= a*a*a*a^1\\
				&= a*a*a*a*a^0\\
				&= a*a*a*a*e\\
				&= a*a*a*a
			\end{align*}
			\item Dans $(\entiers,+)$, $2^3=3 \cdot 2=2+2+2$.
		\end{ulist}
	\end{exem}
	\begin{prop}
		$a^{n+m}=a^n*a^m$, $\forall n,m \in \naturels$.
		\begin{proof}[d\'emonstration par r\'ecurrence sur $n$]~

			\begin{nlist}
				\item $n=0$:
				\begin{align*}
					a^{0+m}&= a^m\\
					&= e*a^m\\
					&= a^0*a^m
				\end{align*}
				\item supposons que $a^{n+m}=a^n*a^m$ pour un $n \geq 0$.
				\begin{align*}
					a^{(n+1)+m}&= a^{n+m+1}\\
					&= a*a^{n+m}\\
					\text{hyp rec}&= a*(a^n*a^m)\\
					&= (a*a^n)*a^m\\
					&= a^{n+1}*a^m
				\end{align*}
			\end{nlist}
		\end{proof}
	\end{prop}
	\begin{defin}
		Pour $n \in \entiers$.

		Si $n \geq 0$, on a d\'ej\`a d\'efini $a^n$.

		Si $n<0$, on d\'efinit $a^n=(a^{-1})^{-n}$.
	\end{defin}
	\begin{exem}
		$a^{-3}=a^{-1}*a^{-1}*a^{-1}$.
	\end{exem}
	\begin{prop}
		$a^{n+m}=a^n*a^m$, $\forall n,m \in \entiers$.
	\end{prop}
	\begin{prop}
		$(a^m)^n=a^{mn}$, $\forall m,n \in \naturels$. Vraie aussi pour $m,n \in \entiers$.
		\begin{proof}[d\'emonstration par r\'ecurrence sur $m$]~

		\begin{nlist}
			\item $m=0$:
			\begin{align*}
				(a^n)^0&= e\\
				a^{n \cdot 0}&= a^0=e
			\end{align*}
			\item supposons que $(a^n)^m=a^{nm}$ pour un certain $m \in \naturels$.
			\begin{align*}
				(a^n)^{m+1}&= (a^n)(a^n)^m\\
				\text{hyp rec}&= (a^n)a^{nm}\\
				&= a^{n+nm}\\
				&= a^{n(m+1)}
			\end{align*}
		\end{nlist}
		\end{proof}
	\end{prop}

	\cours
	\begin{rappel}
		~

		\begin{ulist}[noitemsep]
			\item Isomorphisme: $f:G \to H$ t.q.
			\begin{nlist}
				\item $f(ab)=f(a)f(b)$

				avec $a *b$ et $f(a) \diamond f(b)$ implicitement.
				\item $f$ est bijective
			\end{nlist}
			``m\^eme table''
			\item $f,g$ isomorphismes $\Rightarrow$ $f^{-1}, g \circ f$ isomorphismes.

			$\mathds{1}_G:G \to G$ est trivialement un isomorphisme.
			\item $G$ est isomorphe \`a $H$ s'il existe un isomorphisme $f:G \to H$.
			\item Puissances:

			Soit $a \in G$ avec $G$ un groupe.
			\begin{ulist}
				\item $a^0=e$
				\item $a^{n+1}=aa^n$
				\item $a^{-n}=(a^{-1})^n$
				\item $a^{n+m}=a^na^m$
				\item $(a^n)^m=a^{n \cdot m}$
			\end{ulist}
			\item $f$ isomorphisme
			\begin{ulist}
				\item $f(e_G)=e_H$
				\item $f(a^{-1})=f(a)^{-1}$
			\end{ulist}
		\end{ulist}
	\end{rappel}
	\begin{prop}
		$f$ isomorphisme $f:G \to H$.

		$a \in G$. Alors, $f(a^n)=f(a)^n$, $\forall n \in \entiers$.
		\begin{proof}[d\'emonstration par r\'ecurrence sur $n$]~

			\begin{ulist}
				\item[$n\geq 0$]
				\begin{nlist}
					\item $n=0$
					\begin{align*}
						f(a^0)&= f(e_G)\\
						&= e_H\\
						&= f(a)^0
					\end{align*}
					\item supposons que $f(a^n)=f(a)^n$ pour un certain $n \in \entiers$.
					\begin{align*}
						f(a^{n+1})&= f(a \cdot a^n)\\
						&= f(a)f(a^n)\\
						\text{hyp rec}&= f(a)f(a)^n\\
						&= f(a)^{n+1}
					\end{align*}
				\end{nlist}
				\item[$n<0$] alors, $-n>0$ et
				\begin{align*}
					f(a^n)&= f((a^{-1})^{-n})\\
					&= f(a^{-1})^{-n}\\
					&= (f(a)^{-1})^{-n}\\
					&= f(a)^n
				\end{align*}
			\end{ulist}
		\end{proof}
	\end{prop}
	\begin{exem}
		\(
		H= \left\lbrace \begin{pmatrix}
			1&x\\0&1
		\end{pmatrix} \in GL(2, \reels) \middle| x \in \reels \right\rbrace
		\)
		, avec la multiplication de matrices.

		Soient $\begin{pmatrix}
			1&x\\0&1
		\end{pmatrix} \begin{pmatrix}
			1&y\\0&1
		\end{pmatrix} = \begin{pmatrix}
			1&x+y\\0&1
		\end{pmatrix} \in H$
		\begin{itemize}
			\item[$(A)$:] associatif, car la multiplication de matrices est associative.
			\item[$(N)$:] $\begin{pmatrix}
				1&0\\0&1
			\end{pmatrix} \in H$ est neutre
			\item[$(I)$:] l'inverse de $\begin{pmatrix}
				1&x\\0&1
			\end{pmatrix}$ est $\begin{pmatrix}
				1&-x\\0&1
			\end{pmatrix}$
		\end{itemize}
		Ainsi, $H$ est un groupe pour la multiplication matricielle.

		On d\'efinit
		$\begin{array}{rcl}
			f:\reels &\to& H\\
			x&\mapsto&\begin{pmatrix}
				1&x\\0&1
			\end{pmatrix}
		\end{array}$

		Soient $x,y \in \reels$.
		\begin{nlist}
			\item $f(x+y) = \begin{pmatrix}
				1&x+y\\0&1
			\end{pmatrix} = \begin{pmatrix}
			1&x\\0&1
			\end{pmatrix} \begin{pmatrix}
			1&y\\0&1
			\end{pmatrix} = f(x) \cdot f(y)$
			\item
			\begin{proof}[montrons que]
				$f$ est bijective.

				\begin{ulist}
					\item $f$ est injective

					Soient $x,y \in \reels$.

					Supposons $f(x)=f(y)$
					\begin{align*}
						\begin{pmatrix}
							1&x\\0&1
						\end{pmatrix}&= \begin{pmatrix}
							1&y\\0&1
						\end{pmatrix}\\
						x&= y
					\end{align*}
					\item $f$ est surjective

					Soit $Y = \begin{pmatrix}
						1&y\\0&1
					\end{pmatrix} \in H$, avec $y \in \reels$.

					$Y = f(y)$.
				\end{ulist}
			\end{proof}
		\end{nlist}
	\end{exem}

	\subsection{Sous-groupes}
	\begin{defin}
		$H \subseteq (G,*)$ est un \emph{sous-groupe} de $G$ si $H$ est un groupe pour la m\^eme op\'eration que $G$.
	\end{defin}
	\begin{exem}
		~

		\begin{ulist}
			\item $\{e\} \subseteq G$ est un sous-groupe.
			\item $G \subseteq G$ est un sous-groupe.
			\item $\left\lbrace \dotsc, -4, -2, 0, 2, 4, \dotsc \right\rbrace = 2\entiers \subseteq (\entiers,+)$
			\item Dans $\entiers_4 = \{\overline0, \overline1, \overline2, \overline3\}$, $\{\overline0, \overline2\}$ est un sous-groupe.
			\renewcommand{\arraystretch}{1.5}
			\[
			\begin{array}{c||c|c}
				+&\overline0&\overline2\\
				\hline\hline
				\overline0&\overline0&\overline2\\
				\hline
				\overline2&\overline2&\overline0
			\end{array}
			\]
			\renewcommand{\arraystretch}{1}

			Ce groupe est isomorphe \`a $\entiers_2$ et \`a $C_2=\left( \{-1,1\}, \times \right)$.
			\item $(\entiers,+) \subseteq (\rationels,+) \subseteq (\reels,+)$.
			\item $C_2 \subseteq \rationels_* \subseteq \reels_*$.
			\item $\mathbb{D}_3 = \{\varepsilon, \alpha, \beta, \gamma, \rho, \sigma\}$.

			$\{\varepsilon, \alpha\}$ et $\{\varepsilon, \rho, \sigma\}$ sont des sous-groupes de $\mathbb{D}_3$.
		\end{ulist}
		\begin{nota}
			On note l'ensemble $m\entiers = \{m \cdot n \mid n \in \entiers\}$.
		\end{nota}
	\end{exem}

	\cours
	\begin{rappel}
		~

		\begin{ulist}[noitemsep]
			\item $a \in G$.
			\begin{ulist}
				\item $a^n=\underbrace{a*a*\dotsb*a}_{\text{$n$ fois}}$
				\item $a^n=a*a^{n-1}$
				\item $a^0=e$
				\item $a^{-n}=(a^{-1})^n$
			\end{ulist}
			\item Sous-groupe de $(G,*):H \subseteq G$ qui est un groupe pour $*$.
			\begin{exem}
				$\entiers \subseteq \rationels \subseteq \reels$ pour $+$.
			\end{exem}
			\begin{exem}
				\[
				\left\lbrace \begin{pmatrix}
					1&0\\0&1
				\end{pmatrix}, \begin{pmatrix}
					0&1\\1&0
				\end{pmatrix} \right\rbrace
				\]
				est un sous-groupe de $GL(2,\reels)$.
			\end{exem}
		\end{ulist}
	\end{rappel}
	\begin{prop}
		$H \subseteq G$ un sous-groupe.
		\begin{nlist}[itemsep=1pt]
			\item Si $G$ est ab\'elien, alors $H$ est ab\'elien;
			\item Le neutre de $H$ est le neutre de $G$;
			\item Si $a \in H$, son inverse $a^{-1}\in H$ est l'inverse de $a$ dans $G$.
		\end{nlist}
		\begin{proof}[d\'emonstration]~

			\begin{nlist}[itemsep=1pt]
				\item $G$ est ab\'elien, alors $\forall a,b \in G$, $ab=ba$.

				En particulier, $\forall a,b \in H$, $ab=ba$.
				\item Le neutre de $G$ $e_G$ a la propri\'et\'e que $\forall a \in G$, $e_Ga=a=ae_G$.

				Comme $H \subseteq G$, cette propri\'et\'e est vraie pour $H$ aussi.

				Donc, $ae_G=a=e_Ga$.

				Ainsi, $e_G$ est le neutre de $H$, par l'unicit\'e de l'\'el\'ement neutre.
				\item $a \in H$, il existe un inverse $b \in G$ pour $a$ t.q. $ab=ba=e$.

				Comme $H$ est un groupe, $\exists! a^{-1} \in H$.

				De $ab=e$, on a $a^{-1}ab=a^{-1}e$, donc $b=a^{-1}$.
			\end{nlist}
		\end{proof}
	\end{prop}
	\begin{thm}~

		Un sous-ensemble non-vide $H \subseteq G$ est un sous-groupe ssi pour tous $a,b \in H$, $ab^{-1} \in H$.
		\begin{proof}[d\'emonstration]~

			\begin{itemize}
				\item[$(\Rightarrow)$] Supposons que $H$ est un sous-groupe, donc $a,b \in H$, alors $b^{-1} \in H$.

				De plus, $H$ est ferm\'e pour la multiplication, donc $ab^{-1} \in H$.
				\item[$(\Leftarrow)$]
				\begin{itemize}
					\item[$(N)$] $H$ est non-vide, donc $\exists a \in H$.

					Par hypoth\`ese, $aa^{-1}=e \in H$.
					\item[$(I)$] On vient de montrer que $e \in H$.

					Soit $b \in H$ quelconque. Par hypoth\`ese, $eb^{-1}=b^{-1} \in H$.
					\item[$(A)$] On sait que $\forall a,b,c \in G$, $(ab)c=a(bc)$.

					En particulier, $\forall a,b,c \in H$, $(ab)c=a(bc)$.
				\end{itemize}

				Finalement, $H$ est ferm\'e pour l'op\'eration de $G$, car $\forall a,b \in H$, $b^{-1} \in H$.

				Donc, par hypoth\`ese, $a(b^{-1})^{-1}=ab \in H$.
			\end{itemize}
		\end{proof}
	\end{thm}
	\begin{exem}~

		Soit $m \in \entiers$.

		Posons $H = m\entiers = \{mn \mid n \in \entiers\} = \{\dotsb,-2m,-m,0,m,2m,\dotsb\}$, muni de l'addition.
		\begin{proof}[m.q.]
			$H$ est un sous-groupe de $(\entiers,+)$.

			$H$ est non-vide, car $m0=0 \in H$.

			Soient $a,b \in H$.

			Par d\'efinition de $H$, $\exists a',b' \in \entiers$ t.q. $a=ma'$ et $b=mb'$.

			Dans $\entiers$, $b^{-1} = -mb'$.

			On a $a+(-b) = ma' + (-mb') = m(a'-b') \in H$.

			Donc, $H$ est un sous-groupe de $(\entiers,+)$.
		\end{proof}

		R\'eciproquement, soit $H \subseteq H$ un sous-groupe de $\entiers$ quelconque. Alors $\exists m \in \entiers$ t.q. $H=m\entiers$.
		\begin{proof}[d\'emonstration]~

			On sait que $0\ in H$.

			Si $H=\{0\}$, alors $H=0\entiers$ et l'\'enonc\'e est vrai.

			Sinon, $H$ contient un autre \'el\'ement $a \in H$, donc $-a \in H$.

			En particulier, $H$ contient au moins un entier positif.

			Soit $m$ le \emph{plus petit} \'el\'ement positif de $H$.

			Soit $h \in H$ quelconque. On divise $h$ par $m$, donc $h=qm+r$, o\`u $q,r \in \entiers$ et $0 \leq r < m$.

			Si $r=o$, $h=qm \in m\entiers$.

			Sinon, comme $h \in H$ et $m \in H$, $h-qm \in H$, mais $h-qm=r$, donc $r \in H$.

			Comme $0<r<m$, il y a une contradiction \`a la d\'efinition de $m$.
			\begin{flushright}
				\lightning
			\end{flushright}

			Ainsi, $H \subseteq m\entiers$. Mais clairement, $m\entiers \subseteq H$, car $m \in H$ et $mn \in H$, donc $H=m\entiers$.
		\end{proof}
	\end{exem}
	\begin{prop}~

		Soit $f:G \to H$ un isomorphisme. Alors, $K \subseteq G$ est un sous-groupe de $G$ ssi $f(K)$ est un sous-groupe de $H$.
		\begin{nota}
			$f(K) = \{f(k) \mid k \in K\}$.
		\end{nota}

		\begin{itemize}
			\item[$(\Rightarrow)$] Supposons que $K$ est un sous-groupe de $G$.

			On sait que $e \in K$, alors $f(e)=e \in f(K)$, donc $f(K)$ est non-vide.

			Soient $a,b \in f(K)$. On veut montrer que $ab^{-1} \in f(K)$.

			Alors, $a=f(k)$ et $b=f(k')$, avec $k,k' \in K$.

			Donc, $ab^{-1} = f(k)f(k')^{-1} = f(k)f(k'^{-1}) = f(kk'^{-1})$.

			Comme $kk'^{-1} \in K$, $ab^{-1} \in f(K)$.
			\item[$(\Leftarrow)$] On effectue la m\^eme preuve avec $f^{-1}$ qui est un isomorphisme en remarquant que $f^{-1}(f(k))=k$.
		\end{itemize}
	\end{prop}
	\begin{nota}
		~

		$G\xrightarrow{f}H$ avec $f$ un isomorphisme est \'equivalent \`a $G\xrightarrow{\sim}H$.
	\end{nota}
	\begin{nota}
		~

		$H \subseteq G$ un sous-groupe est \'equivalent \`a $H \leq G$.
	\end{nota}
	\begin{prop}
		~

		Soit $\left\lbrace H_i \right\rbrace_{i \in I}$ une collection de sous-groupes de $G$. Alors $\bigcap\limits_{i \in I}H_i \leq G$.
		\begin{proof}[d\'emonstration]~

			$H_i \leq G, \forall i \in I$.

			Alors, $e \in H_i, \forall i$. Donc, $e \in \bigcap\limits_{i \in I}H_i$. En particulier, $\bigcap\limits_{i \in I}H_i \neq \emptyset$.

			Soient $a,b\in \bigcap\limits_{i \in I}H_i$. Alors, $a,b \in H_i, \forall i$.

			Comme $H_i \leq G$, $ab^{-1} \in H_i, \forall i$, donc $ab^{-1} \in \bigcap\limits_{i \in I}H_i$.
		\end{proof}
	\end{prop}
	\begin{rema}
		Si $H_1,H_2 \leq G$, $H_1 \cup H_2$ n'est pas n\'ecessairement un sous-groupe de $G$.
		\begin{exem}
			$2\entiers \leq \entiers$ et $3\entiers \leq \entiers$. $2\entiers \cup 3\entiers$ n'est pas un sous-groupe de $\entiers$.

			Plus pr\'ecis\'ement, $2,3 \in 2\entiers \cup 3\entiers$, mais $2+3=5 \not\in 2\entiers \cup 3\entiers$.
		\end{exem}
	\end{rema}
	\begin{exem}
		$2\entiers \cap 3\entiers = 6\entiers \leq \entiers$.
	\end{exem}

	\setcounter{chapter}{1}
	\chapter{Applications et \'equivalences}
	\setcounter{section}{3}
	\section{Relations d'\'equivalence}
	\begin{defin}
		Une \emph{relation d\'equivalence} sur un ensemble $E$ est un sous-ensemble $R \subseteq E \times E$ satisfaisant
		\begin{nlist}
			\item r\'eflexivit\'e

			$x \sim x, \forall x \in E$.
			\item sym\'etrie

			$x \sim y \Rightarrow y \sim x$.
			\item transitivit\'e

			$x \sim y$ et $y \sim z$, alors $x \sim z$.
		\end{nlist}
		\begin{nota}
			On note $x \sim y$ ssi $(x,y) \in R$.
		\end{nota}
	\end{defin}
	\begin{exem}~

		\begin{nlist}
			\item $E = \reels$

			$x \sim y$ ssi $\abs{x} = \abs{y}$
			\begin{itemize}
				\item[(refl)] $\abs{x} = \abs{x}$, donc $x \sim x$.
				\item[(sym)] Supposons $x \sim y$. Alors, $\abs{x} = \abs{y}$. Donc, $y \sim x$.
				\item[(trans)] Supposons $x \sim y$ et $y \sim z$. Alors, $\abs{x} = \abs{y}$ et $\abs{y} = \abs{z}$. Donc $\abs{x} = \abs{z}$. Ainsi, $x \sim z$.
			\end{itemize}
			\item $C$ est l'ensemble des \'el\`eves dans la classe.

			$x \sim y$ ssi $x$ et $y$ ont le m\^eme \^age est une relation d'\'equivalence.
		\end{nlist}
	\end{exem}
	\begin{defin}
		Si $E$ est un ensemble et $\sim$ est une relation d'\'equivalence sur $E$, la \emph{classe d'\'euivalence} de $x \in E$ est $\overline{x} = \{y \in E \mid y \sim x\} \subseteq E$.
	\end{defin}
	\begin{lem}
		$\overline{x} = \overline{y}$ ssi $x \sim y$.
		\begin{proof}[d\'emonstration]~

			\begin{itemize}
				\item[$(\Rightarrow)$] Supposons $\overline{x} = \overline{y}$.

				Par (refl), $x \sim x$, donc $x \in \overline{x}$. Alors, $x \in \overline{y}$. Ainsi, $x \sim y$.
				\item[$(\Leftarrow)$] Supposons $x \sim y$.
				\begin{itemize}
					\item[$(\subseteq)$] Soit $z \in \overline{x}$. Alors $z \sim x$. Comme $x \sim y$, par (trans), $z \sim y$. Donc, $z \in \overline{y}$.
					\item[$(\supseteq)$] idem
				\end{itemize}

				Ainsi, $\overline{x} = \overline{y}$.
			\end{itemize}
		\end{proof}
	\end{lem}

	\cours
	\begin{rappel}~

		\begin{ulist}[noitemsep]
			\item $H \leq G$, $H$ un sous-groupe de $G$, ssi
			\begin{ulist}
				\item $H \neq \emptyset$;
				\item $\forall a,b \in H, ab^{-1} \in H$.
			\end{ulist}
			\item Si $H \leq G$, $H$ a le m\^eme neutre que $G$, m\^emes inverses.
			\item $G$ ab\'elien $\Rightarrow H \leq G$ ab\'elien.
			\item Si $f:G \to H$ isomorphisme, alors $K \leq G \Leftrightarrow f(K) \leq H$.
			\item Relations d'\'equivalence $\sim$ sur $E$:
			\begin{itemize}
				\item[(Refl)] $a \sim a, \forall a \in E$;
				\item[(Sym)] $a \sim b \Rightarrow b \sim a$;
				\item[(Trans)] $a \sim b, b \sim c \Rightarrow a \sim c$.
			\end{itemize}
			\item Classe d'\'equivalence: $\overline{a} = \{b \in E \mid b \sim a\}$.
			\item $a \sim b \Longleftrightarrow \overline{a} = \overline{b}$.
		\end{ulist}
	\end{rappel}
	\begin{exem}
		$E=\entiers$.

		\'Equivalence modulo $m$:

		$a \sim b$ ssi $a-b=km$, avec $k \in \entiers$.
		\begin{nota}
			$m \mid a-b$, $m$ divise $a-b$: $\exists k \in \entiers$ t.q. $a-b=km$.
		\end{nota}

		$a \sim b \Longleftrightarrow m \mid a-b$.
		\begin{proof}[d\'emonstration]
			que c'est bel et bien une \'equivalence
			\begin{itemize}
				\item[(Refl)] Soit $a \in \entiers$.

				$a-a=0m \Rightarrow a \sim a$.
				\item[(Sym)] Supposons que $a \sim b$, avec $a,b \in \entiers$.

				Alors, $a-b=km$, avec $k \in \entiers$.

				Or, $-(a-b)=-km \Rightarrow b-a=(-k)m$, donc $b \sim a$.
				\item[(Trans)] Supposons que $a \sim b$ et $b \sim c$, avec $a,b,c \in \entiers$.

				Alors, $a-b=k_1m$ et $b-c=k_2m$, avec $k_1,k_2 \in \entiers$.

				En additionant les deux \'equations, on obtient
				\begin{align*}
					a-c&= k_1m+k_2m\\
					&= (k_1+k_2)m
				\end{align*}

				Donc, $a \sim c$.
			\end{itemize}
		\end{proof}

		Si $m=2$, les classes d'\'equivalence sont
		\begin{align*}
			\begin{split}
				\overline{0}&= \{a \in \entiers \mid a \sim 0\}\\
				&= \{a \in \entiers \mid a-0=2k\}\\
				&= \{a \in \entiers \mid a=2k\}\\
				&= 2\entiers
			\end{split}
			&
			\begin{split}
				\overline{1}&= \{a \in \entiers \mid a \sim 1\}\\
				&= \{a \in \entiers \mid a-1=2k\}\\
				&= \{a \in \entiers \mid a=2k+1\}\\
				&= \entiers\setminus2\entiers
			\end{split}
		\end{align*}
		\begin{rema}
			\begin{align*}
				\begin{split}
					\overline{0} = \overline{2} = \overline{4} = \overline{-2} = \dotsb
				\end{split}
				&
				\begin{split}
					\overline{1} = \overline{3} = \overline{5} = \overline{-1} = \dotsb
				\end{split}
			\end{align*}
		\end{rema}

		Plus g\'en\'eralement, pour $m$, on a $m$ classes d'\'equivalence.
		\[
		\overline{0}, \overline1, \overline2, \dotsb, \overline{m-1}
		\]
		\begin{nota}
			L'ensemble des classes d'\'equivalence est not\'e $\entiers_m = \{\overline{0}, \overline1, \overline2, \dotsb, \overline{m-1}\}$ pour la relation de congruence modulo $m$.
		\end{nota}
	\end{exem}
	\begin{defin}
		Une \emph{partition} d'un ensemble $E$ est une collection $\mathcal{P} = \{E_i\}$, avec $i \in I$ de sous-ensembles de $E$ t.q.
		\begin{nlist}
			\item $\bigcup\limits_{i \in I} E_i = E$;
			\item $E_i \cap E_j = \emptyset$, si $i \neq j$.
		\end{nlist}
		\begin{rema}
			Chaque $x \in E$ est \'el\'ement d'exactement un $E_i$.
		\end{rema}
	\end{defin}
	\begin{prop}
		Si $\sim$ est une relation d'\'equivalence sur $E$, alors $\mathcal{P} = \{\overline{a} \mid a \in E\}$ est une partition de $E$.
		\begin{exem}
			$E = \entiers$, $\sim$ equivalence modulo 3.

			$\mathcal{P} = \{\overline{0}, \overline{1}, \overline{2}\}$ est une partition de $\entiers$.
		\end{exem}
		\begin{proof}[d\'emonstration]~

			\begin{nlist}
				\item Clairement, $\bigcup\limits_{a \in E}\overline{a} \subseteq E$.

				On veut montrer que $E \subseteq \bigcup\limits_{a \in E}\overline{a}$.

				Soit $x \in E$, alors $x \sim x$ par r\'eflexivit\'e, donc$x \in \overline{x}$ et $x \in \bigcup\limits_{a \in E}\overline{a}$.
				\item Supposons que $x \in \overline{a}$ et $x \in \overline{b}$, avec $\overline{a} \neq \overline{b}$.

				Alors, $x \sim a$ et $x \sim b$. Donc, par sym\'etrie, $a \sim x$ et $x \sim b$. Donc, par transitivit\'e, $a \sim b$. Donc $\overline{a} = \overline{b}$. Ceci est une contradiction, donc $\overline{a} \cap \overline{b} = \emptyset, \forall \overline{a}, \overline{b} \in \mathcal{P}$.
			\end{nlist}
		\end{proof}
	\end{prop}

	On d\'efinit une op\'eration sur $\entiers_m$ pour la congruence modulo $m$.

	\[
	\begin{array}{rcl}
		+:\entiers_m \times \entiers_m& \to& \entiers_m\\
		(\overline{a}, \overline{b})& \mapsto& \overline{a+b}
	\end{array}
	\]

	Autrement dit, $\overline{a} + \overline{b} = \overline{a+b}$.
	\begin{rema}
		L'\'ecriture d'un \'el\'ement $\overline{a}$ n'est pas unique ($\overline{a} = \overline{a'}$ si $a \sim a'$).
	\end{rema}

	Il faut v\'erifier que l'op\'eration $+$ est correctement d\'efinie (d\'efinie sans abigu\"\i t\'e).

	Autrement dit, si $\overline{a} = \overline{a'}$ et $\overline{b} = \overline{b'}$, on veut montrer que $\overline{a+b} = \overline{a'+b'}$.

	\cours
	\begin{rappel}
		~

		\begin{ulist}[noitemsep]
			\item Relation d'\'equivalence ($\sim$) $\rightarrow$ partition en classes d'\'equivalence ($\overline{a}\{b \mid b \sim a\}$).
			\item \'Equivalence (congruence) $\mod m$ (sur $\entiers$):
			\begin{align*}
				a \sim b&\Leftrightarrow m \mid a-b\\
				&\Leftrightarrow \exists k \in \entiers \text{ t.q. } a-b=km
			\end{align*}

			On note l'ensemble des classes d'\'equivalence $\mod m$ par $\entiers_m = \{\overline{a} \mid a \in \entiers\} = \{\overline{0}, \overline{1}, \dotsb, \overline{m-1}\}$.

			On veut d\'efinir une op\'eration $+$ sur $\entiers_m$ par $\overline{a} + \overline{b} = \overline{a+b}$.

			On doit v\'erifier que cette d\'efinition n'est pas ambigu\"e (ne d\'epend pas des repr\'esentants).

			Supposons $\overline{a_1} = \overline{a_2}$ et $\overline{b_1} = \overline{b_2}$. On doit v\'erifier que $\overline{a_1+b_1} = \overline{a_2+b_2}$, c'est-\`a-dire que $a_1+b_1 \sim a_2+b_2$.

			Les hypoth\`eses donnent: $a_1-a_2=k_am$ et $b_1-b_2=k_bm$. En additionnant ces deux \'equations, on obtient $(a_1-a_2)+(b_1-b_2) = k_am+k_bm$. Alors, $(a_1+b_1) - (a_2+b_2) = (k_a+k_b)m$. Donc, $a_1+b_1 \sim a_2+b_2$.
			\begin{flushright}
				$\square$
			\end{flushright}
			\begin{rema}
				On doit faire ce genre de preuve pour chaque d\'efinition de fonction/op\'eration qui ont comme domaine des classes d'\'equivalence.
			\end{rema}
		\end{ulist}
	\end{rappel}
	\begin{exem}
		$m=5$.

		\begin{center}
			\begin{tikzpicture}
				\node at (0,0) {0};
				\node at (-.5,-.5) {-5};
				\node at (.5,-.5) {5};
				\node at (-.5,-1) {-10};
				\node at (.5,-1) {10};
				\node at (0,-1.5) {$\vdots$};
				\draw (0,-.75) ellipse (1 and 1.5);
				\node at (1,.5) {$\overline{0}$};
			\end{tikzpicture}
			\qquad
			\begin{tikzpicture}
				\node at (0,0) {1};
				\node at (-.5,-.5) {-4};
				\node at (.5,-.5) {6};
				\node at (-.5,-1) {-9};
				\node at (.5,-1) {11};
				\node at (0,-1.5) {$\vdots$};
				\draw (0,-.75) ellipse (1 and 1.5);
				\node at (1,.5) {$\overline{1}$};
			\end{tikzpicture}
			\qquad
			\begin{tikzpicture}
				\node at (0,0) {2};
				\node at (-.5,-.5) {-3};
				\node at (.5,-.5) {7};
				\node at (-.5,-1) {-8};
				\node at (.5,-1) {12};
				\node at (0,-1.5) {$\vdots$};
				\draw (0,-.75) ellipse (1 and 1.5);
				\node at (1,.5) {$\overline{2}$};
			\end{tikzpicture}
			\qquad
			\begin{tikzpicture}
				\node at (0,0) {3};
				\node at (-.5,-.5) {-2};
				\node at (.5,-.5) {8};
				\node at (-.5,-1) {-7};
				\node at (.5,-1) {13};
				\node at (0,-1.5) {$\vdots$};
				\draw (0,-.75) ellipse (1 and 1.5);
				\node at (1,.5) {$\overline{3}$};
			\end{tikzpicture}
			\qquad
			\begin{tikzpicture}
				\node at (0,0) {4};
				\node at (-.5,-.5) {-1};
				\node at (.5,-.5) {9};
				\node at (-.5,-1) {-6};
				\node at (.5,-1) {14};
				\node at (0,-1.5) {$\vdots$};
				\draw (0,-.75) ellipse (1 and 1.5	);
				\node at (1,.5) {$\overline{4}$};
			\end{tikzpicture}
		\end{center}

		Pour faire $\overline{2}+\overline{1}$, on peut prendre $\overline{2+1} = \overline{2}$, ou bien $\overline{17+(-4)} = \overline{13}$.
	\end{exem}

	\begin{prop}
		$(\entiers_m,+)$ est un groupe ab\'elien.
		\begin{proof}[d\'emonstration]~

			\begin{itemize}
				\item[(A)] Soient $a,b,c \in \entiers_m$.
				\begin{align*}
					\left( \overline{a} + \overline{b} \right) + \overline{c}&= \overline{a+b} + \overline{c}\\
					&= \overline{(a+b)+c}\\
					&= \overline{a+(b+c)}\\
					&= \overline{a} + \overline{b+c}\\
					&= \overline{a} + \left( \overline{b} + \overline{c} \right)
				\end{align*}
				\item[(C)] $\overline{a} + \overline{b} = \overline{a+b} = \overline{b+a} = \overline{b} + \overline{a}$.
				\item[(N)] $\overline{0} + \overline{a} = \overline{0+a} = \overline{a}$, donc $\overline{0}$ est neutre. Par commutativit\'e, la propri\'et\'e est satisfaite.
				\item[(I)] $\overline{a} + \overline{-a} = \overline{a-a} = \overline{0}$, donc $\overline{-a}$ est l'inverse de $\overline{a}$. Par commutativit\'e, la propri\'et\'e est satisfaite.

				On peut donc \'ecrire $-\overline{a} = \overline{-a}$.
			\end{itemize}
		\end{proof}
	\end{prop}

	\subsection{Ordre et groupes cycliques}
	\begin{defin}
		Soient $G$ un groupe et $a \in G$.

		L'\emph{ordre} de $a$, not\'e $o(a)$ est la plus petite positive $k$ t.q. $a^k=e$, si elle existe. Sinon, on note $o(a) = \infty$.
	\end{defin}
	\begin{exem}~

		\begin{ulist}
			\item $o(e)=1$
			\item Dans $\mathbb{D}_3$
			\begin{ulist}
				\item $o(\alpha) = 2$
				\item $o(\rho) = 3$
			\end{ulist}
			\item Dans $\entiers$
			\begin{ulist}
				\item $o(0) = 1$
				\item $o(n) = \infty$, $\forall n\neq0$
			\end{ulist}
			\item Dans $\entiers_6$
			\begin{ulist}
				\item $o(\overline2) = 3$
			\end{ulist}
		\end{ulist}
	\end{exem}
	\begin{prop}
		Soit $m \in \entiers$. $a^m = e$, ssi $o(a) \mid m$.
		\begin{proof}[d\'emonstration]~

			\begin{itemize}
				\item[$(\Rightarrow)$] Supposons $a^m=e$.

				On divise $m$ par $o(a)$: $m = q \cdot o(a) + r$, avec $0 \leq r < o(a)$.

				Si $r=0$, $o(a) \mid m$ et on a termin\'e.

				Supposons que $0<r<o(a)$, alors
				\begin{align*}
					r&= m-q \cdot o(a)\\
					a^r&= a^{m-q \cdot o(a)}\\
					&= a^m \cdot a^{-q \cdot o(a)}\\
					&= e \cdot (a^{o(a)})^{-q}\\
					&= e \cdot e^{-q}\\
					&= e
				\end{align*}
				mais $0<r<o(a)$ contredit la minimalit\'e de $o(a)$.
			\end{itemize}
		\end{proof}
	\end{prop}
\end{document}

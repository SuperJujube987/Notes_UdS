\documentclass{report}
\usepackage[letterpaper,portrait,margin=2cm]{geometry}
\usepackage{titlesec}
\usepackage{amsmath,amssymb,amsthm}
\usepackage{tabularx}
\usepackage{enumitem}
\usepackage{indentfirst}
\usepackage{dsfont}
\usepackage{tikz}
\usetikzlibrary{tikzmark, calc, positioning, arrows.meta, shapes.geometric, 3d}
\usepackage{hyperref}
\usepackage{mathtools}
\usepackage{mathrsfs}
\usepackage{enumitem}
\usepackage{wasysym}
\usepackage{fancyhdr}
\usepackage[french]{babel}
\usepackage[makeroom]{cancel}
\usepackage{nicematrix}
\usepackage[dvipsnames]{xcolor}
\usepackage{xfrac}
\usepackage{tikz-cd}
\usepackage{fontawesome5}
\usepackage{multicol}
\usepackage{gensymb}

\allowdisplaybreaks

\hypersetup{
	colorlinks=false
}

\definecolor{Col1}{HTML}{117733}
\definecolor{Col2}{HTML}{44AA99}
\definecolor{Col3}{HTML}{88CCEE}
\definecolor{Col4}{HTML}{DDCC77}
\definecolor{Col5}{HTML}{CC6677}
\definecolor{Col6}{HTML}{AA4499}
\colorlet{col1}{Col1!50}
\colorlet{col2}{Col2!50}
\colorlet{col3}{Col3!50}
\colorlet{col4}{Col4!50}
\colorlet{col5}{Col5!50}
\colorlet{col6}{Col6!50}

\fancyfoot{}
\fancyhead{}
\fancyhead[l]{MAT141 - \'El\'ements d'alg\`ebre}
\fancyhead[r]{\leftmark}
\fancyfoot[c]{\thepage}
\renewcommand{\headrulewidth}{.4pt}
\renewcommand{\footrulewidth}{0pt}

\pagestyle{fancy}

\newlist{nlist}{enumerate}{3}
\setlist[nlist, 1]{label=(\arabic*)}
\setlist[nlist, 2]{label=(\alph*)}
\setlist[nlist, 3]{label=(\roman*)}
\newlist{ulist}{itemize}{3}
\setlist[ulist, 1]{label=\textbullet}
\setlist[ulist, 2]{label=|}
\setlist[ulist, 3]{label=\textperiodcentered}

\title{MAT141 - \'El\'ements d'alg\`ebre

Donn\'e par Jean-Philippe Burelle}
\author{Julien Houle}
\date{Automne 2025}

\newcounter{cours}
\setcounter{cours}{1}
\newcommand*{\cours}{\section*{Cours \thecours}\stepcounter{cours}}

\newcommand*{\abs}[1]{\left| #1 \right|}
\newcommand*{\card}[1]{\left| #1 \right|}

\newcommand*{\lte}{\leqslant}
\newcommand*{\gte}{\geqslant}

\newcommand*{\reels}{\mathbb{R}}
\newcommand*{\entiers}{\mathbb{Z}}
\newcommand*{\rationels}{\mathbb{Q}}
\newcommand*{\naturels}{\mathbb{N}}

\newcommand*{\identite}{\mathds{1}}

\newcommand*{\image}{\operatorname{Im}}
\newcommand*{\signe}{\operatorname{sgn}}
\newcommand*{\stab}{\operatorname{stab}}

\renewcommand{\thesection}{ \arabic{chapter}.\arabic{section}}
\renewcommand{\thesubsection}{}
\renewcommand{\thesubsubsection}{}

\titleformat{\chapter}[hang]{\bfseries\huge\centering}{Chapitre \arabic{chapter}}{1em}{}[]
\titleformat{\section}[hang]{\bfseries\large}{Section\thesection}{1em}{}[]
\titleformat{\subsection}[hang]{\bfseries\normalsize}{}{0pt}{}[]
\titleformat{\subsubsection}[hang]{\slshape\normalsize}{}{0pt}{}[]

\newtheorem*{thm}{Th\'eor\`eme}
\newtheorem*{lem}{Lemme}
\newtheorem*{prop}{Proposition}
\newtheorem*{coro}{Corollaire}
\theoremstyle{definition}
\newtheorem*{defin}{D\'efinition}
\theoremstyle{remark}
\newtheorem*{exem}{Exemple}
\newtheorem*{exer}{Exercice}
\newtheorem*{nota}{Notation}
\newtheorem*{rema}{Remarque}
\newtheorem*{rappel}{Rappel}

\newenvironment*{subproof}[1]
{
\begin{proof}[#1]

}
{
	\renewcommand{\qedsymbol}{$\blacksquare$}
\end{proof}
\renewcommand{\qedsymbol}{$\square$}
}
\newenvironment{subsubproof}[1]
{
\begin{proof}[#1]
}
{
	\renewcommand{\qedsymbol}{$\#$}
\end{proof}
\renewcommand{\qedsymbol}{$\blacksquare$}
}

\begin{document}
	\maketitle
	\tableofcontents
	\pagenumbering{roman}
	\newpage
	\pagenumbering{arabic}

	\chapter{Ensembles}
	\cours
	Id\'ee: ensemble = patate.

	\begin{tikzpicture}
		\path[smooth cycle,very thick,draw=black] plot coordinates {(0,-.3) (-1,-1) (-1.5,-3) (-.8,-3.9) (0,-4) (.8,-3.8) (1.4,-3) (.8,-1.5)};
		\node[font=\Large] () at (0,-1) {$a$};
		\node[font=\Large] () at (-.5,-2) {$b$};
		\node[font=\Large] () at (.5,-3) {$c$};
		\node[anchor=east] () at (-1.5,-2) {$E=$};
		\node[anchor=west] () at (1.5,-2) {$=\{a,b,c\}$};
	\end{tikzpicture}

	\begin{nota}
		$E \subseteq F \Leftarrow \forall x \in E, x \in F$.
		\begin{rema}
			$E \subseteq E$.
		\end{rema}
	\end{nota}

	\begin{nota}
		La cardinalit\'e d'un ensemble, $\card{E}$, est le nombre d'\'el\'ements d'un ensemble.
	\end{nota}

	\begin{defin}
		D\'efinition d'un ensemble par \emph{compr\'ehension}
		\begin{exem}
			$E=\left\lbrace n \in \entiers \middle| 1 \lte n \lte 20 \right\rbrace$.
		\end{exem}
	\end{defin}

	\begin{nota}
		$E=F \Leftrightarrow E \subseteq F$ et $F \subseteq E$.
	\end{nota}

	\begin{defin}
		Produit cart\'esien: $E \times F=\left\lbrace (x,y) \middle| x \in E, y \in F \right\rbrace$.
	\end{defin}

	\begin{defin}
		Fonction/Application

		$f: A \to B$, $A$ et $B$ des ensembles, associe \`a \emph{chaque} $x \in A$ un \emph{unique} \'el\'ement $f(x) \in B$.
	\end{defin}

	\cours
	\begin{rappel}
		~

		\begin{tabularx}{.9\textwidth}{cl>{\raggedright\arraybackslash}X}
			\textbullet&Ensemble&collection d'objets\\
			\textbullet&$\in$&``\'el\'ement'' d'un ensemble\\
			\textbullet&\emph{sous-ensemble} $(\subseteq)$&$E \subseteq F$ si $x \in E$ \emph{implique} $x \in F$\\
			\textbullet&$E=F$&si, et seulement si, $E \subseteq F$ et $F \subseteq E$\\
			\textbullet&$\cup$&union\\
			&$\cap$&intersection\\
			\textbullet&$E \times F$&produit cart\'esien (paires $(x,y)$)\\
			\textbullet&$f:E \to F$&\emph{fonction} ou \emph{application}, associe \`a \underline{chaque $x \in E$} un unique $\underline{f(x) \in F}$, image de $x$ par $f$\\
			\textbullet&$\identite$&$\identite_E:E \to E$ est d\'efinie comme $\identite_E(x)=x$
		\end{tabularx}
	\end{rappel}


	\subsection{Mani\`eres de d\'efinir une fonction}
	\begin{ulist}[noitemsep]
		\item \'enum\'erer $f(x)$ pour chaque $x \in E$
		\item donner une formule

		une formule ne d\'efinit pas toujours une fonction, elle doit \^etre valide pour chaque $x$ de l'ensemble de d\'epart.
		\item en mots (d\'ecrire la valeur pour chaque $x \in E$)
		\item m\'elange de formule et mots
	\end{ulist}

	\begin{defin}
		Une fonction $f:E \to F$ est \emph{inversible} s'il existe une fonction $\underbrace{g:F \to E}_{*}$ telle que $\underbrace{g(f(x))=x}_{**}$ pour tout $x \in E$ et $\underbrace{f(g(y))=y}_{***}$ pour tout $y \in F$.
	\end{defin}

	\begin{exem}
		$f:\entiers \to \entiers$, $f(x)=x+1$ est inversible d'inverse $g(y)=y-1$

		\begin{proof}
			~

			On v\'erifie que
			\begin{align*}
				g(f(x))&= x& g(f(x))&= g(x+1)\\
				&&&= (x+1)-1\\
				&&&= x\\
				f(g(y))&= y& f(g(y))&= f(y-1)\\
				&&&= (y-1)+1\\
				&&&= y
			\end{align*}
		\end{proof}
	\end{exem}

	\begin{prop}
		Si $f$ admet un inverse, celui-ci est unique.

		\begin{proof}
			~

			Supposons que $g_1$ et $g_2$ sont tous deux inverses de $f$ et montrons qu'elles sont \'gales.

			(Pour d\'emontrer que deux fonctions sont \'egales, il suffit de montrer que $g_1(y)=g_2(y)$ pour tout $y \in F$)

			Soit $y \in F$.

			On a
			\begin{align*}
				g_1(y)&\overbrace{=}^{***} g_1(f(\underbrace{g_2(y)}_{*}))\\
				&\overbrace{=}^{**} g_2(y)
			\end{align*}
		\end{proof}
	\end{prop}

	\begin{defin}
		Si $f:E \to F$ et $g:F \to G$, alors la \emph{compos\'ee} de $f$ et $g$ est la fonction $g \circ f:E \to G$ d\'efinie par la formule $g \circ f(x)=g(f(x))$.
	\end{defin}

	\begin{defin}[Red\'efinition de l'inverse]~

		$g \circ f=\identite_E$

		$f \circ g=\identite_F$
	\end{defin}

	\begin{exem}~

		$A=\{a,b,c\}$

		$B=\{d,e,f\}$

		$f:A \to B$, $a \mapsto d, b \mapsto e, c \mapsto f$

		$g:B \to A$, $d \mapsto a, e \mapsto b, f \mapsto c$

		$g \circ f:A \to A$, $g \circ f(x)=x$, $g \circ f=\identite_A$.

		De la m\^eme mani\`ere, $f \circ g=\identite_B$.

		~

		Ainsi, $g$ est l'inverse de $f$.
	\end{exem}

	\begin{nota}
		On note $g=f^{-1}$ l'inverse de $f$.
	\end{nota}


	\begin{rappel}
		Pour trouver l'inverse d'une fonction $f:\reels \to \reels$ donn\'ee par une formule $f(x)=y$, on isole $x$ en fonction de $y$.
	\end{rappel}


	\begin{exem}
		\begin{align*}
			f(x)&= 3x-8\\
			y&= 3x-8\\
			y+8&= 3x\\
			\dfrac{y+8}{3}&= x\\
			g(y)&= \dfrac{y+8}{3}
		\end{align*}

		Dans un devoir, on commence par la formule de l'inverse et on v\'erifie $g(f(x))=x$ et $f(g(y))=y$.
	\end{exem}

	\begin{defin}
		On dit que $f:E \to F$ est une fonction \emph{injective} si $f(x_1)=f(x_2)$ implique $x_1=x_2$.
	\end{defin}

	\begin{defin}
		On dit que $f:E \to F$ est une fonction \emph{surjective} si pour tout $y \in F,~\exists~x \in E$ t.q. $f(x)=y$.
	\end{defin}

	\begin{defin}
		On dit que $f:E \to F$ est une fonction \emph{bijective} si elle est injective \textbf{et} surjective.
	\end{defin}

	\begin{exem}~

		\begin{ulist}
			\item
			\[
			f:\reels \to \reels^{\gte0}, f(x)=\abs{x}
			\]
			$f$ n'est pas injective, car $f(1)=\abs{1}=1$ et $f(-1)=\abs{-1}=1$, mais $1 \neq -1$.

			$f$ est surjective, car soit $y \in \reels^{\gte0}$, alors pour $x=y$, on a $f(x)=f(y)=\abs{y}=y$.

			\item
			\[
			\begin{array}{rrcl}
				f:&\naturels&\to&\naturels\\
				&x&\mapsto&x+1
			\end{array}
			\]

			$f$ est injective:

			Soient $x_1,x_2 \in \naturels$.

			On suppose $f(x_1)=f(x_2)$.

			\begin{align*}
				x_1+1&= x_2+1\\
				x_1&= x_2
			\end{align*}

			$f$ n'est pas surjective

			$y=0\in\naturels$ n'est pas \'egal \`a $f(x)$ pour $x\in\naturels$. Si il existait $x$ avec $f(x)=0$, $x+1=0$, $x=-1$, $x \not\in \naturels$.

			\item
			\[
			\begin{array}{rrcl}
				f:&\reels&\to&\reels\\
				&x&\mapsto&2x+3
			\end{array}
			\]

			$f$ est injective:

			Soient $x_1,x_2 \in \reels$.

			supposons $f(x_1)=f(x_2)$, $2x_1+3=2x_2+3$, $2x_1=2x_2$, $x_1=x_2$.

			$f$ est surjective:

			Soit $y \in \reels$.

			On cherche $x$ t.q. $f(x)=y$.

			Posons $x=\dfrac{y-3}{2} \in \reels$.

			Alors, $f(x) = f\left( \dfrac{y-3}{2} \right) = 2 \cdot  \dfrac{y-3}{2} + 3 = y-3+3 = y$.

			Ainsi, $f$ est bijective.

			\item
			\[
			f:A \to B; A=\{1,48,57\}, B=\{a,b,c\}
			\]

			\[
			1 \mapsto a, 48 \mapsto a, 57 \mapsto b
			\]

			$f$ n'est pas injective, car $1 \mapsto a$ et $48 \mapsto a$  avec $1 \neq 48$.

			$f$ n'est pas surjective, car aucun \'el\'ement de $x \in A \mapsto c$.
		\end{ulist}


		~



		~



		~


	\end{exem}

	\begin{rema}
		La fonction $f':A \to B'$ avec $B'=\{a,b\}$ est surjective.
	\end{rema}


	\cours
	\begin{rappel}
		$A, B$ deux ensembles

		\begin{ulist}[noitemsep]
			\item $f:A \to B$ une fonction, associe \`a chaque $x \in A$ un unique $f(x) \in B$. $x \mapsto f(x)$.
			\item $f$ est \emph{inversible} s'il existe $g:B \to A$ t.q. $g(f(a))=a$ pour tout $a \in A$ et $f(g(b))=b$ pour tout $b \in B$.
			\item l'inverse est \emph{unique}.
			\item La composition de $f:A \to B$ avec $g:B \to C$ est $g \circ f:A \to C$ avec $(g \circ f)(a)=g(f(a))$.
			\item $f$ est injective si $f(x_1)=f(x_2) \Rightarrow x_1=x_2$.
			\item $f$ est surjective si pour tout $b \in B$ il existe $a \in A$ t.q. $f(a)=b$.
			\item $f$ est bijective si elle est injective et surjective.
		\end{ulist}
	\end{rappel}


	\begin{prop}
		$f:A \to B$ est bijective \emph{si, et seulement si,} elle est inversible.

		\begin{proof}~

			\begin{itemize}
				\item[$\Leftarrow$:] ~

				Supposons que $f$ est inversible.

				Alors, il existe un inverse $g:B \to A$.

				\begin{itemize}
					\item[(inj):] ~

					Soient $x_1,x_2 \in A$.

					On suppose que $f(x_1)=f(x_2)$.

					Alors, $g(f(x_1)) = g(f(x_2))$.

					Donc, $x_1 = x_2$.
					\item[(surj):] ~

					Soit $y \in B$.

					Posons $x=g(y) \in A$.

					Alors, $f(x)=f(g(y))=y$.
				\end{itemize}

				Ainsi, $f$ est bijective.
				\item[$\Rightarrow$:] ~

				Supposons $f$ est injective et surjective.

				\begin{lem}
					Pour \emph{chaque} $y \in B$, il existe un \emph{unique} $x \in A$ t.q. $f(x)=y$.

					\renewcommand{\qedsymbol}{$\blacksquare$}
					\begin{proof}~

						\underline{Existance}:
						Comme $f$ est surjective, $x$ existe.

						\underline{Unicit\'e}:
						Supposons $x_1,x_2 \in A$ t.q. $f(x_1)=f(x_2)$, alors $x_1=x_2$ puisque $f$ est injective.
					\end{proof}
					\renewcommand{\qedsymbol}{$\square$}
				\end{lem}

				On d\'efinit $g:B \to A$ par $g(y)=x$ o\`u $x$ est l'unique \'el\'ement du lemme.

				On v\'erifie:

				Soit $x \in A$, alors $g(\underbrace{f(x)}_y)=x$, par d\'efinition de $g$.

				Soit $y \in B$, alors $f(\underbrace{g(y)}_{\text{l'unique $x$ t.q. $f(x)=y$}})=y$.
			\end{itemize}
		\end{proof}
	\end{prop}

	\begin{defin}
		Une \emph{op\'eration} (interne, binaire) sur un ensemble $E$ est un fonction $m:E \times E \to E$.
	\end{defin}

	\begin{exem}
		$E=\entiers$,

		$\begin{array}{rrcl}
			+:&\entiers \times \entiers &\longrightarrow& \entiers\\
			&(n,m)&\longmapsto&n + m
		\end{array}$

		$\begin{array}{rrcl}
			\cdot:&\entiers \times \entiers &\longrightarrow& \entiers\\
			&(n,m)&\longmapsto&n \cdot m
		\end{array}$

		~

		$\begin{array}{rrcl}
			d:&\rationels \times \rationels&\longrightarrow&\rationels\\
			&(x,y)&\longmapsto&\frac{x}{y}
		\end{array}$

		n'est pas une op\'eration, car $(1,0) \mapsto \frac{1}{0}$ qui n'est pas d\'efini. ($d$ n'est pas une fonction.)

		Cependant,

		$\begin{array}{rrcl}
			d:&\rationels_* \times \rationels_*&\longrightarrow&\rationels_*\\
			&(x,y)&\longmapsto&\frac{x}{y}
		\end{array}$

		est une op\'eration.

		~

		$A$ un ensemble

		$E=\{f:A \to A\}$, o\`u $f$ est une fonction.

		$\begin{array}{rrcl}
			c:&E \times E&\longrightarrow&E\\
			&(f,g)&\longmapsto&f \circ g
		\end{array}$

		La composition est une op\'eration.
	\end{exem}

	\begin{nota}
		On note la plupart du temps une op\'eration par un symbole entre les entr\'ees.
	\end{nota}

	\begin{exem}
		$m(x,y) \coloneq x*y$, ou $x+y$, ou $x \circ y$, ou $xy$.
	\end{exem}

	\begin{defin}~

		Un \emph{\'el\'ement neutre} pour une op\'eration $*$ est un \'el\'ement $e \in E$ t.q. pour tout $x \in E$, $e*x=x$ et $x*e=x$.
	\end{defin}

	\cours
	\begin{rappel}
		~

		\begin{ulist}[noitemsep]
			\item $f:E \to F$ est bijective $\Leftrightarrow$ $f$ est inversible.
			\item L'inverse est unique $(g=f^{-1})$
			\item Op\'eration: $m:E \times E \to E$, ou
			$\begin{array}{rrcl}
				*:&E \times E&\to&E\\
				&(x,y)&\mapsto&z
			\end{array}$
			\item \'El\'ement neutre: $e \in E$ t.q. $e*x=x$ et $x*e=x$.
			\item $f$ est injective si tout $y \in F$ a au plus un ant\'ec\'edent
			\item $f$ est surjective si tout $y \in F$ a au moins un ant\'ec\'edent
			\item $f$ est bijective si tout $y \in F$ a exactement un ant\'ec\'edent
			\item $x$ est ant\'ec\'edent de $y$ si $f(x)=y$
		\end{ulist}
	\end{rappel}

	\begin{exem}~

		Sur $\naturels$,

		\begin{ulist}
			\item $0$ est neutre pour $+$.
			\begin{align*}
				0+n&=n\\
				n+0&=n
			\end{align*}
			\item $1$ est neutre pour $\times$.
			\begin{align*}
				1 \times n&=n\\
				n \times 1&=n
			\end{align*}
		\end{ulist}

		Sur $\entiers$, $-$ est une op\'eration mais elle n'a pas d\'el\'ement neutre.
		\renewcommand{\qedsymbol}{\lightning}
		\begin{proof}[En effet,]~

			Supposons que $e \in \entiers$ est neutre, alors $e-n=n$ pour tout $n$.

			Pour $n=0$, $e-0=0$, donc $e=0$.

			Pour $n=1$, $e-1=1$, donc $-1=1$.
		\end{proof}
		\renewcommand{\qedsymbol}{$\square$}
		\begin{ulist}
			\item Sur l'ensemble $E=\left\lbrace \left( \begin{array}{cc}
				a&b\\c&d
			\end{array}\right) \middle| a,b,c,d \in \reels\right\rbrace$, la multiplication matricielle $\times$ est une op\'eration.

			La matrice $I=\left( \begin{array}{cc}
				1&0\\0&1
			\end{array}\right)$ est neutre pour $\times$.
			\item Sur $E=\{f:A \to A\}$, la fonction $\identite_A$ est neutre pour la composition de fonctions.
			\begin{proof}~

				On doit montrer $\identite_A \circ f=f$ et $f \circ \identite_A=f$ pour tout $f \in E$.
				\begin{nlist}
					\item Soit $x \in A$, alors
					\begin{align*}
						(\identite_A \circ f)(x)&= \identite_A(f(x))\\
						&= f(x)
					\end{align*}

					Donc, $\identite_A \circ f=f$.
					\item Soit $x \in A$, alors
					\begin{align*}
						(f \circ \identite_A)(x)&= f(\identite_A(x))\\
						&=f(x)
					\end{align*}

					Donc,$f \circ \identite_A=f$.
				\end{nlist}
			\end{proof}
		\end{ulist}
	\end{exem}

	On peut d\'ecrire une op\'eration sur un ensemble fini avec sa table ``de multiplication''.
	\begin{exem}
		$A=\{0,1\}$

		$f_1:\begin{array}{rcl}
			0&\mapsto&0\\
			1&\mapsto&0
		\end{array},f_2:\begin{array}{rcl}
			0&\mapsto&0\\
			1&\mapsto&1
		\end{array},f_3:\begin{array}{rcl}
			0&\mapsto&1\\
			1&\mapsto&0
		\end{array},f_4:\begin{array}{rcl}
			0&\mapsto&1\\
			1&\mapsto&1
		\end{array}$
		On a $f_2=\identite_A$.

		\[
		\begin{array}{c||c|c|c|c}
			\circ&f_1&f_2&f_3&f_4\\
			\hline\hline
			f_1&f_1&f_1&f_1&f_1\\
			\hline
			f_2&f_1&f_2&f_3&f_4\\
			\hline
			f_3&f_4&f_3&f_2&f_1\\
			\hline
			f_4&f_4&f_4&f_4&f_4
		\end{array}
		\]
	\end{exem}
	\begin{defin}
		~

		Une op\'eration $*$ sur $E$ est \emph{associative} si pour tout $x,y,z \in E$, on a $(x*y)*z=x*(y*z)$.
	\end{defin}
	\begin{prop}
		~

		Si $*$ admet un \'el\'ement neutre, alors celui-ci est \emph{unique}.
		\begin{proof}~

			Supposons que $e$ et $e'$ sont neutres pour $*$.

			On a
			\[
			\begin{array}{rclll}
				e*e'&=&e'&\quad&\text{\footnotesize car $e$ est neutre}\\
				e*e'&=&e&&\text{\footnotesize car $e'$ est neutre}
			\end{array}
			\]

			Donc, $e=e'$.
		\end{proof}
	\end{prop}
	\begin{defin}
		~

		Soit $E$ un ensemble, $*$ une op\'eration sur $E$ et $e \in E$ un neutre pour $*$. On dit que $a,b \in E$ sont \emph{inverses} si $a*b=e$ et $b*a=e$.

		Dans ce cas, on dit que $a$ et $b$ sont inversibles.
	\end{defin}
	\begin{exem}
		~

		Dans $\entiers$ avec $+$, $3$ et $-3$ sont inverses. En effet, on a $3+(-3)=0$ et $(-3)+3=0$ avec $0$ l'\'el\'ement neutre de $+$.
	\end{exem}
	\begin{exem}
		~

		Dans $\entiers$ avec $\times$, le neutre est 1, mais seuls $1$ et $-1$ sont inversibles. En effet, on a $1 \times 1=1$ et $(-1) \times (-1)=1$.
	\end{exem}
	\begin{rema}
		~

		L'\'el\'ement neutre est son propre inverse. En effet, $e*e=e$, pour tout $*$ qui admet $e$ comme \'el\'ement neutre.
	\end{rema}
	\begin{prop}
		~

		Si $*$ est associative et admet un \'el\'ement neutre $e$, alors les inverses sont uniques s'ils existent.
		\begin{proof}~

			Soit $a \in E$.

			Supposons $b,b'$ sont inverses de $a$.

			Alors,
			\begin{align*}
				b&=b*e\\
				\text{\footnotesize car $b'$ est inverse de $a$}&=b*(a*b')\\
				\text{\footnotesize associativit\'e}&=(b*a)*b'\\
				\text{\footnotesize car $b$ est inverse de $a$}&=e*b'\\
				&=b'
			\end{align*}
		\end{proof}
		\begin{nota}
			~

			Comme l'inverse de $a$ est unique, on le note $a^{-1}$.
		\end{nota}
	\end{prop}
	\begin{exem}
		~

		Dans $E=\{f:A \to A\}$, avec l'op\'eration $\circ$, les fonctions bijectives sont exactement celles qui sont inversibles pour $\circ$.
	\end{exem}
	\begin{prop}
		~

		La composition de fonctions est associative.
		\begin{proof}~

			Soient $f:A \to B$, $g:B \to C$ et $h:C \to D$.

			Soit $a \in A$.
			\begin{align*}
				((h \circ g) \circ f)(a)&= (h \circ g)(f(a))\\
				&= h(g(f(a)))\\
				&= h((g \circ f)(a))\\
				&= (h \circ (g \circ f))(a)\\
				\Rightarrow (h \circ g) \circ f&= h \circ (g \circ f)
			\end{align*}
		\end{proof}
	\end{prop}
	\setcounter{chapter}{5}
	\chapter{Groupes}
	\begin{defin}
		~

		Un \emph{groupe} est un ensemble $G$ muni d'une op\'eration $*$ t.q.
		\begin{itemize}
			\item[(A)] $*$ est associative
			\item[(N)] $*$ admet un neutre
			\item[(I)] tout $g \in G$ admet un inverse
		\end{itemize}
	\end{defin}
	\begin{exem}
		~

		\begin{nlist}
			\item $(\entiers,+)$ est un groupe.

			Neutre: $0$

			Inverse de $n$: $-n$
			\item $(\rationels,+)$ et $(\reels,+)$ sont des groupes.
			\item $(\entiers,\times)$ n'est pas un groupe, car, par exemple, $2$ n'est pas inversible.
			\item $(\rationels,\times)$ n'est pas un groupe, car $0$ n'est pas inversible.
			\item $(\rationels_*,\times)$ et $(\reels_*,\times)$ sont des groupes.

			Neutre: $1$

			Inverse de $x$: $\frac{1}{x}$
		\end{nlist}
		\begin{rema}
			$(1)$, $(2)$ et $(5)$ sont \emph{commutatifs}.
		\end{rema}
		\begin{rema}
			$(\naturels,+)$ n'est pas un groupe.
		\end{rema}
		\begin{defin}
			Si l'op\'eration d'un groupe est commutative, on note le groupe comme \emph{ab\'elien} (ou commutatif).
		\end{defin}
		\begin{nlist}[resume]
			\item $GL(n,\reels)$ est un groupe pour la multiplication matricielle.

			$GL(n,\reels)=\{M|M \text{ est une matrice $n\times n$ r\'eelle inversible}\}$.

			$GL$: g\'en\'eral lin\'eaire

			Neutre: $\left( \begin{array}{ccc}
				1&\dotsb&0\\
				\vdots&\ddots&\vdots\\
				0&\dotsb&1
			\end{array} \right)$.

			$M^{-1}$ la matrice inverse est l'inverse.

			Pour $n \gte 2$, $GL(n,\reels)$  n'est pas ab\'elien.
			\item $A$ un ensemble quelconque

			$S(A)=\{f:A \to A|f \text{ est bijective}\}$ est un groupe pour $\circ$.

			Neutre: $\identite_A$

			Inverse de $f$: $f^{-1}$
		\end{nlist}
	\end{exem}
	\begin{rema}
		~

		Pour $A=\{0,1\}$

		$f_1:\begin{array}{rcl}
			0&\mapsto&0\\
			1&\mapsto&0
		\end{array},f_2:\begin{array}{rcl}
			0&\mapsto&0\\
			1&\mapsto&1
		\end{array},f_3:\begin{array}{rcl}
			0&\mapsto&1\\
			1&\mapsto&0
		\end{array},f_4:\begin{array}{rcl}
			0&\mapsto&1\\
			1&\mapsto&1
		\end{array}$, $S(A)=\{f_2,f_3\}$.
	\end{rema}

	\cours
	\begin{rappel}
		~

		\begin{ulist}[noitemsep]
			\item Groupe: $(G,*)$
			\begin{itemize}
				\item[$G$] ensemble
				\item[$*$] op\'eration sur $G$
				\item[$(A)$] $*$ est associative

				$\forall a,b,c \in G, (a*b)*c=a*(b*c)$
				\item[$(N)$] $*$ admet un \'el\'ement neutre dans $G$

				$\exists e \in G$ t.q. $\forall a \in G, e*a=a=a*e$
				\item[$(I)$] tout \'el\'ement de $G$ est inversible

				$\forall a \in G, \exists b \in G$ t.q. $a*b=e=b*a$
			\end{itemize}
			\item Le neutre et l'inverse sont uniques
		\end{ulist}
	\end{rappel}

	\begin{rema}
		~

		``Le groupe $\reels$'' implique l'op\'eration $+$ et ``le groupe $\reels_*$'' implique l'op\'eration $\times$.
	\end{rema}

	\subsection{Propri\'et\'es \'el\'ementaires des groupes}
	\begin{nlist}
		\item $\forall a,b \in G$, $(a*b)^{-1} = b^{-1}*a^{-1}$.
		\item $\forall a \in G$, $(a^{-1})^{-1}=a$
		\item Si $a*b=a*c$, alors $b=c$
		\item Si $b*a=c*a$, alors $b=c$
	\end{nlist}
	\begin{proof}~

		\begin{nlist}
			\item On calcule
			\begin{align*}
				\begin{split}
					(a*b)*(b^{-1}*a^{-1})&= a*(b*(b^{-1}*a^{-1}))\\
					&= a*((b*b^{-1})*a^{-1})\\
					&= a*(e*a^{-1})\\
					&= a*a^{-1}\\
					&= e
				\end{split}
				&
				\begin{split}
					(b^{-1}*a^{-1})*(a*b)&= b^{-1}*(a^{-1}*(a*b))\\
					&= b^{-1}*((a^{-1}*a)*b)\\
					&= b^{-1}*(e*b)\\
					&= b^{-1}*b\\
					&= e
				\end{split}
			\end{align*}

			Donc, $(a*b)^{-1} = b^{-1}*a^{-1}$.
			\item Comme $a^{-1}*a=e=a*a^{-1}$, $a$ est l'inverse de $a^{-1}$, donc $\left( a^{-1} \right)^{-1}=a$.
			\item Supposons $a*b=a*c$. Alors
			\begin{align*}
				a^{-1}*(a*b)&= a^{-1}*(a*c)\\
				(a^{-1}*a)*b&= (a^{-1}*a)*c\\
				e*b&= e*c\\
				b&= c
			\end{align*}
			\item Supposons $b*a=c*a$. Alors
			\begin{align*}
				(b*a)*a^{-1}&= (c*a)*a^{-1}\\
				b*(a*a^{-1})&= c*(a*a^{-1})\\
				b*e&= c*e\\
				b&= c
			\end{align*}
		\end{nlist}
	\end{proof}
	\begin{exem}
		~

		$\left( \entiers_3,+ \right)$.

		$\entiers_3=\left\lbrace \overline{0}, \overline{1}, \overline{2} \right\rbrace$

		\renewcommand{\arraystretch}{1.5}
		\[
		\begin{array}{c||c|c|c}
			+&\overline0&\overline1&\overline2\\
			\hline\hline
			\overline0&\overline0&\overline1&\overline2\\
			\hline
			\overline1&\overline1&\overline2&\overline0\\
			\hline
			\overline2&\overline2&\overline0&\overline1
		\end{array}
		\]
		\renewcommand{\arraystretch}{1}

		$+$ est associative.

		$\overline{0}$ est l'\'el\'ement neutre.

		$(\overline{1})^{-1}=\overline{2}$.

		$(\overline{2})^{-1}=\overline{1}$.

		$\left( \entiers_3,+ \right)$ est un groupe ab\'elien.
		\begin{rema}
			La sym\'etrie de la table par rapport \`a la diagonale implique la commutativit\'e.
		\end{rema}
	\end{exem}
	\begin{exem}
		~\phantomsection

		$\left( \mathbb{D}_3, \circ \right)$ - groupe dih\'edral d'ordre 3.\label{D3}

		Groupe des sym\'etries d'un triangle \'equilat\'eral.

		$\mathbb{D}_3=\left\lbrace \underset{\text{identit\'e}}{\varepsilon}, \underset{\text{r\'eflexion}}{\alpha, \beta, \gamma}, \underset{\text{rotation}}{\rho, \sigma} \right\rbrace$.
		\begin{center}
			\begin{tikzpicture}[>=to]
				\coordinate (A) at (0,0);
				\coordinate (B) at (0:2cm);
				\coordinate (C) at (60:2cm);
				\draw (A) -- (B) -- (C) -- cycle;

				\coordinate (a) at ($(A)!.5!(B)$);
				\coordinate (b) at ($(B)!.5!(C)$);
				\coordinate (c) at ($(A)!.5!(C)$);
				\draw (A) -- (b);
				\draw (C) -- (a);
				\draw (B) -- (c);
				\node (alpha) at (a) [below=2mm] {$\alpha$};
				\node (beta) at (b) [above right=0mm and 1mm] {$\beta$};
				\node (gamma) at (c) [above left=0mm and 1mm] {$\gamma$};

				\draw[<->] (a)+(215:7pt) -- +(325:7pt);
				\draw[<->] (b)+(335:7pt) -- +(85:7pt);
				\draw[<->] (c)+(95:7pt) -- +(205:7pt);

				\coordinate (r1) at (77.02:1.65cm);
				\coordinate (r2) at (108.57:.64cm);
				\coordinate (s1) at (44.55:2.29);
				\coordinate (s2) at (15.45:2.29);
				\path[->,bend angle=45] (r1) edge[bend right] (r2) (s1) edge[bend left] (s2);

				\node (sigma) at (30:2.6cm) {$\sigma$};
				\node (rho) at (100.96:1.32cm) {$\rho$};
			\end{tikzpicture}
			\hspace{2cm}
			$\begin{array}{c||c|c|c|c|c|c}
				\circ&\varepsilon&\alpha&\beta&\gamma&\rho&\sigma\\
				\hline\hline
				\varepsilon&\varepsilon&\alpha&\beta&\gamma&\rho&\sigma\\
				\hline
				\alpha&\alpha&\varepsilon&\rho&\sigma&\beta&\gamma\\
				\hline
				\beta&\beta&\sigma&\varepsilon&\rho&\gamma&\alpha\\
				\hline
				\gamma&\gamma&\rho&\sigma&\varepsilon&\alpha&\beta\\
				\hline
				\rho&\rho&\gamma&\alpha&\beta&\sigma&\varepsilon\\
				\hline
				\sigma&\sigma&\beta&\gamma&\alpha&\varepsilon&\rho
			\end{array}$
		\end{center}

		$\left( \mathbb{D}_3, \circ \right)$ n'est pas un groupe ab\'elien.
	\end{exem}

	\cours
	\begin{rappel}~

		\begin{ulist}[noitemsep]
			\item Groupe: $(G, *)$ avec $A, N, I$.

			Ab\'elien: $C$.
			\item
			\[
			\begin{array}{rclcrcl}
				a*b&=&a*c&\Rightarrow&b&=&c\\
				b*a&=&c*a&\Rightarrow&b&=&c\\
				(a^{-1})^{-1}&=&a\\
				(a*b)^{-1}&=&b^{-1}*a^{-1}
			\end{array}
			\]
			\item
			\begin{exem}~

				$(\entiers, +), (\rationels, +), (\reels, +), (\rationels_*, \times), (\reels_*, \times)$ sont des groupes ab\'eliens; $\entiers_3, \mathbb{D}_3, GL(n, \reels)$ sont des groupes non ab\'eliens.

				$S(E)=\{f:E \to E \mid f\text{ est bijective}\}$.
				\begin{rema}
					$E$ n'est pas l'ensemble utilis\'e dans la d\'efinition du groupe.
				\end{rema}
			\end{exem}
		\end{ulist}
	\end{rappel}
	\subsection{Produit cart\'esien de groupes}
	$(G, *)$ et $(H, \diamond)$ deux groupes.

	\begin{prop}
		~

		$G \times H$ est un groupe lorsque muni de l'op\'eration $(a,b) \bullet (a',b') = (a*a',b \diamond b')$, avec $a,a' \in G$ et $b,b' \in H$.
		\begin{proof}~

			\begin{itemize}
				\item[(N)] $e \in G$ le neutre et $e' \in H$ le neutre, alors $(e,e') \in G \times H$

				\begin{align*}
					(a,b) \bullet (e,e')&= (a*e, b \diamond e')\\
					&= (a,b)\\
					(e,e') \bullet (a,b)&= (e*a, e' \diamond b)\\
					&= (a,b)
				\end{align*}

				$(e,e')$ est bien neutre.
				\item[(I)] $(a,b) \in G \times H$, alors $(a^{-1},b^{-1})$ est inverse de $(a,b)$.

				En effet,
				\begin{align*}
					(a,b) \bullet (a^{-1},b^{-1})&= (a*a^{-1},b \diamond b^{-1})\\
					&= (e,e')\\
					(a^{-1},b^{-1}) \bullet (a,b)&= (a^{-1}*a,b^{-1} \diamond b)\\
					&= (e,e')
				\end{align*}
				\item[(A)] Soient $(a,b), (c,d), (e,f) \in G \times H$. On a
				\begin{align*}
					((a,b) \bullet (c,d)) \bullet (e,f)&= (a*c,b \diamond d) \bullet (e,f)\\
					&= ((a*c)*e,(b \diamond d) \diamond f)\\
					&= (a*(c*e),b \diamond (d \diamond f))\\
					&= (a,b) \bullet (c*e,d \diamond f)\\
					&= (a,b) \bullet ((c,d) \bullet (e,f))
				\end{align*}
			\end{itemize}
		\end{proof}
	\end{prop}
	\begin{exem}
		\begin{ulist}~

			\item $\reels \times \reels = \reels^2$

			$(x,y)+(x',y')=(x+x',y+y')$.
			\item $(\entiers_2,+)$
			\renewcommand{\arraystretch}{1.5}
			\[
			\begin{array}{c||c|c}
				+&\overline0&\overline1\\
				\hline\hline
				\overline0&\overline0&\overline1\\
				\hline
				\overline1&\overline1&\overline0
			\end{array}
			\]

			$\entiers_2 \times \entiers_2$
			\[
			\begin{array}{c||c|c|c|c}
				+&(\overline0,\overline0)& (\overline0,\overline1)& (\overline1,\overline0)& (\overline1,\overline1)\\
				\hline\hline
				(\overline0,\overline0)& (\overline0,\overline0)& (\overline0,\overline1)& (\overline1,\overline0)& (\overline1,\overline1)\\
				\hline
				(\overline0,\overline1)& (\overline0,\overline1)& (\overline0,\overline0)& (\overline1,\overline1)& (\overline1,\overline0)\\
				\hline
				(\overline1,\overline0)& (\overline1,\overline0)& (\overline1,\overline1)& (\overline0,\overline0)& (\overline0,\overline1)\\
				\hline
				(\overline1,\overline1)& (\overline1,\overline1)& (\overline1,\overline0)& (\overline0,\overline1)& (\overline0,\overline0)
			\end{array}
			\]
			\renewcommand{\arraystretch}{1}
		\end{ulist}
	\end{exem}
	\subsection{Isomorphismes de groupes}
	\begin{defin}
		$(G,*)$ et $(H,\diamond)$ deux groupes.

		Un \emph{isomorphisme} de $G$ vers $H$ est une application $f:G \to H$ t.q.
		\begin{nlist}[noitemsep]
			\item $\forall a,b \in G$, $f(a*b)=f(a)\diamond f(b)$.

			Pr\'eservation des op\'erations
			\item $f$ est bijective.
		\end{nlist}
	\end{defin}
	\begin{exem}~

		\begin{ulist}
			\item $(\reels,+)$ et $(\reels^+_*,\times)$

			$\begin{array}{rcl}
				f:\reels&\to&\reels^+_*\\
				x&\mapsto&e^x
			\end{array}$ est un isomorphisme de groupes.
			\begin{nlist}
				\item Soient $x,y \in \reels$.
				\begin{align*}
					f(x+y)&= e^{x+y}\\
					&= e^x \times e^y\\
					&= f(x) \times f(y)
				\end{align*}
				\item $\ln:\reels^+_* \to \reels$ est inverse de $f$: $\ln e^x=x, \forall x \in \reels$ et $e^{\ln x}=x, \forall x \in \reels^+_*$.
			\end{nlist}
		\end{ulist}
	\end{exem}
	\begin{prop}
		Si $f:G \to H$ est un isomorphisme de groupes, alors $f(e_G)=e_H$, o\`u $e_G$ est l'\'el\'ement neutre de $G$ et $e_H$ est l'\'el\'ement neutre de $H$.
		\begin{proof}
			Strat\'egie: montrer que $f(e_G)$ est neutre pour $H$ et utiliser l'unicit\'e.

			Soit $b \in H$.

			Comme $f$ est bijective, $\exists a \in G$ t.q. $f(a)=b$
			\begin{align*}
				\begin{split}
					f(e_G) \diamond b&= f(e_G) \diamond f(a)\\
					&= f(e_G*a)\\
					&= f(a)\\
					&=b
				\end{split}
				&
				\begin{split}
					b \diamond f(e_G)&= f(a) \diamond f(e_G)\\
					&= f(a*e_G)\\
					&= f(a)\\
					&= b
				\end{split}
			\end{align*}

			On a donc que $f(e_G) \in H$ est neutre pour $\diamond$, mais comme l'\'el\'ement neutre est unique, $f(e_G)=e_H$.
		\end{proof}
	\end{prop}
	\begin{exem}
		Pour $\begin{array}{rcl}
			f:\reels&\to&\reels^+_*\\
			x&\mapsto&e^x
		\end{array}$, $f(0)=e^0=1$.
	\end{exem}
	\begin{prop}
		Si $f:G \to H$ est un isomorphisme de groupes, alors $f(a^{-1})=f(a)^{-1}$, pour tout $a \in G$.
		\begin{proof}
			Strat\'egie: montrer que $f(a^{-1})$ est inverse de $f(a)$ et utiliser l'unicit\'e.

			\begin{align*}
				\begin{split}
					f(a^{-1}) \diamond f(a)&= f(a^{-1}*a)\\
					&= f(e_G)\\
					&= e_H
				\end{split}
				&
				\begin{split}
					f(a) \diamond f(a^{-1})&= f(a*a^{-1})\\
					&= f(e_G)\\
					&= e_H
				\end{split}
			\end{align*}

			On a donc que $f(a^{-1})$ est inverse de $f(a)$, mais comme l'inverse est unique, $f(a^{-1})=f(a)^{-1}$.
		\end{proof}
	\end{prop}
	\begin{exem}
		Pour $\begin{array}{rcl}
			f:\reels&\to&\reels^+_*\\
			x&\mapsto&e^x
		\end{array}$, $f(-x)=e^{-x}=(e^x)^{-1}=f(x)^{-1}=\frac{1}{f(x)}$, o\`u $-x$ est l'inverse de $x$ pour $+$ et $\frac{1}{f(x)}$ est l'inverse de $f(x)$ pour $\times$.
	\end{exem}
	\begin{rema}
		Si $G,H$ sont des groupes finis et $f$ est un isomorphisme, alors $f$ ``envoie la table de $G$ \`a celle de $H$''.

		\[
		G:
		\begin{array}{c||c|c|c|c}
			*&e_G&a_1&a_2&\dotsb\\
			\hline\hline
			e_G&&&&\\
			\hline
			a_1&&&a_1*a_2&\\
			\hline
			a_2&&&&\\
			\hline
			\vdots&&&&
		\end{array}
		\xrightarrow{~~~f~~~}
		\begin{array}{c||c|c|c|c}
			\diamond&e_H&f(a_1)&f(a_2)&\dotsb\\
			\hline\hline
			f(e_G)&&&&\\
			\hline
			f(a_1)&&&f(a_1) \diamond f(a_2)&\\
			\hline
			f(a_2)&&&&\\
			\hline
			\vdots&&&&
		\end{array}
		:H
		\]
		Avec $f(a_1*a_2)=f(a_1) \diamond f(a_2)$.
		\begin{exem}
			\[
			\entiers_2:
			\renewcommand{\arraystretch}{1.5}
			\begin{array}{c||c|c}
				+&\overline0&\overline1\\
				\hline\hline
				\overline0&\overline0&\overline1\\
				\hline
				\overline1&\overline1&\overline0
			\end{array}
			\renewcommand{\arraystretch}{1}
			\qquad
			H:
			\begin{array}{c||c|c}
				\circ&\varepsilon&\alpha\\
				\hline\hline
				\varepsilon&\varepsilon&\alpha\\
				\hline
				\alpha&\alpha&\varepsilon
			\end{array}
			\qquad
			C_2:
			\begin{array}{c||c|c}
				\times&1&-1\\
				\hline\hline
				1&1&-1\\
				\hline
				-1&-1&1
			\end{array}
			\]
			$\entiers_2$, $H$ et $C_2$ sont isomorphes.

			Il existe un isomorphisme entre chaque paire.
		\end{exem}
	\end{rema}
	\begin{prop}
		Si $f:G \to H$ est un isomorphisme, alors $f^{-1}:H \to G$ est un isomorphisme.
		\begin{proof}~

			\begin{nlist}
				\item Soient $b_1,b_2 \in H$.
				\begin{align*}
					f^{-1}(b_1 \diamond b_2)&= f^{-1}\left( f\left[ f^{-1}\left( b_1 \right) \right] \diamond f\left[ f^{-1}\left( b_2 \right) \right] \right)\\
					&= f^{-1}\left( f\left[ f^{-1}\left( b_1 \right) * f^{-1}\left( b_2 \right) \right] \right)\\
					&= f^{-1}(b_1) * f^{-1}(b_2)
				\end{align*}
				\item $f^{-1}$ est bijective, car elle est inversible d'inverse $f$.
				\begin{align*}
					f \circ f^{-1}&= \identite_H\\
					f^{-1} \circ f&= \identite_G
				\end{align*}
			\end{nlist}
		\end{proof}
	\end{prop}
	\begin{prop}[Transitivit\'e]~

		Si $f:G \to H$ et $g:H \to K$ sont des isomorphismes, alors $g \circ f:G \to K$ est un isomorphisme.
		\begin{proof}~

			\begin{nlist}
				\item Soient $a,b \in G$
				\begin{align*}
					(g \circ f)(a*b)&= g(f(a*b))\\
					&= g(f(a) \diamond f(b))\\
					&= g(f(a)) \oplus g(f(b))\\
					&= (g \circ f)(a) \oplus (g \circ f)(b)
				\end{align*}
				\item $g \circ f$ est inversible d'inverse $f^{-1} \circ g^{-1}$.
			\end{nlist}
		\end{proof}
	\end{prop}

	\subsection{Puissances d'\'el\'ements de groupes}
	\begin{defin}[par r\'ecurrence]~

		$a \in G$, $n \in \naturels$
		\begin{nlist}
			\item $a^0 \coloneq e_G$
			\item $a^n=a*a^{n-1}$, $\forall n \gte 1$
		\end{nlist}
	\end{defin}
	\begin{exem}~

		\begin{ulist}
			\item \begin{align*}
				a^4&= a*a^3\\
				&= a*a*a^2\\
				&= a*a*a*a^1\\
				&= a*a*a*a*a^0\\
				&= a*a*a*a*e\\
				&= a*a*a*a
			\end{align*}
			\item Dans $(\entiers,+)$, $2^3=3 \cdot 2=2+2+2$.
		\end{ulist}
	\end{exem}
	\begin{prop}
		$a^{n+m}=a^n*a^m$, $\forall n,m \in \naturels$.
		\begin{proof}
			par r\'ecurrence sur $n$.

			\begin{nlist}
				\item $n=0$:
				\begin{align*}
					a^{0+m}&= a^m\\
					&= e*a^m\\
					&= a^0*a^m
				\end{align*}
				\item supposons que $a^{n+m}=a^n*a^m$ pour un $n \gte 0$.
				\begin{align*}
					a^{(n+1)+m}&= a^{n+m+1}\\
					&= a*a^{n+m}\\
					\text{\footnotesize hyp rec}&= a*(a^n*a^m)\\
					&= (a*a^n)*a^m\\
					&= a^{n+1}*a^m
				\end{align*}
			\end{nlist}
		\end{proof}
	\end{prop}
	\begin{defin}
		Pour $n \in \entiers$.

		Si $n \gte 0$, on a d\'ej\`a d\'efini $a^n$.

		Si $n<0$, on d\'efinit $a^n=(a^{-1})^{-n}$.
	\end{defin}
	\begin{exem}
		$a^{-3}=(a^{-1})^3=a^{-1}*a^{-1}*a^{-1}$.
	\end{exem}
	\begin{prop}
		$a^{n+m}=a^n*a^m$, $\forall n,m \in \entiers$.
	\end{prop}
	\begin{prop}
		$(a^n)^m=a^{nm}$, $\forall m,n \in \naturels$. Vraie aussi pour $m,n \in \entiers$.
		\begin{proof}
			par r\'ecurrence sur $m$.

		\begin{nlist}
			\item $m=0$:
			\begin{align*}
				(a^n)^0&= e\\
				a^{n \cdot 0}&= a^0=e
			\end{align*}
			\item supposons que $(a^n)^m=a^{nm}$ pour un certain $m \in \naturels$.
			\begin{align*}
				(a^n)^{m+1}&= (a^n)(a^n)^m\\
				\text{\footnotesize hyp rec}&= (a^n)a^{nm}\\
				&= a^{n+nm}\\
				&= a^{n(m+1)}
			\end{align*}
		\end{nlist}
		\end{proof}
	\end{prop}

	\cours
	\begin{rappel}
		~

		\begin{ulist}[noitemsep]
			\item Isomorphisme: $f:G \to H$ t.q.
			\begin{nlist}
				\item $f(ab)=f(a)f(b)$

				avec $a *b$ et $f(a) \diamond f(b)$ implicitement.
				\item $f$ est bijective
			\end{nlist}
			``m\^eme table''
			\item $f,g$ isomorphismes $\Rightarrow$ $f^{-1}, g \circ f$ isomorphismes.

			$\identite_G:G \to G$ est trivialement un isomorphisme.
			\item $G$ est isomorphe \`a $H$ s'il existe un isomorphisme $f:G \to H$.
			\item Puissances:

			Soit $a \in G$ avec $G$ un groupe.
			\begin{ulist}
				\item $a^0=e$
				\item $a^{n+1}=aa^n$
				\item $a^{-n}=(a^{-1})^n$
				\item $a^{n+m}=a^na^m$
				\item $(a^n)^m=a^{n \cdot m}$
			\end{ulist}
			\item $f$ isomorphisme
			\begin{ulist}
				\item $f(e_G)=e_H$
				\item $f(a^{-1})=f(a)^{-1}$
			\end{ulist}
		\end{ulist}
	\end{rappel}
	\begin{prop}
		$f$ isomorphisme $f:G \to H$.

		$a \in G$. Alors, $f(a^n)=f(a)^n$, $\forall n \in \entiers$.
		\begin{proof}
			par r\'ecurrence sur $n$.

			\begin{ulist}
				\item[$n\gte 0$]
				\begin{nlist}
					\item $n=0$
					\begin{align*}
						f(a^0)&= f(e_G)\\
						&= e_H\\
						&= f(a)^0
					\end{align*}
					\item supposons que $f(a^n)=f(a)^n$ pour un certain $n \in \entiers$.
					\begin{align*}
						f(a^{n+1})&= f(a \cdot a^n)\\
						&= f(a)f(a^n)\\
						\text{\footnotesize hyp rec}&= f(a)f(a)^n\\
						&= f(a)^{n+1}
					\end{align*}
				\end{nlist}
				\item[$n<0$] alors, $-n>0$ et
				\begin{align*}
					f(a^n)&= f((a^{-1})^{-n})\\
					&= f(a^{-1})^{-n}\\
					&= (f(a)^{-1})^{-n}\\
					&= f(a)^n
				\end{align*}
			\end{ulist}
		\end{proof}
	\end{prop}
	\begin{exem}
		\(
		H= \left\lbrace \begin{pmatrix}
			1&x\\0&1
		\end{pmatrix} \in GL(2, \reels) \middle| x \in \reels \right\rbrace
		\)
		, avec la multiplication de matrices.

		Soient $\begin{pmatrix}
			1&x\\0&1
		\end{pmatrix} \begin{pmatrix}
			1&y\\0&1
		\end{pmatrix} = \begin{pmatrix}
			1&x+y\\0&1
		\end{pmatrix} \in H$
		\begin{itemize}
			\item[$(A)$:] associatif, car la multiplication de matrices est associative.
			\item[$(N)$:] $\begin{pmatrix}
				1&0\\0&1
			\end{pmatrix} \in H$ est neutre
			\item[$(I)$:] l'inverse de $\begin{pmatrix}
				1&x\\0&1
			\end{pmatrix}$ est $\begin{pmatrix}
				1&-x\\0&1
			\end{pmatrix}$
		\end{itemize}
		Ainsi, $H$ est un groupe pour la multiplication matricielle.

		On d\'efinit
		$\begin{array}{rcl}
			f:\reels &\to& H\\
			x&\mapsto&\begin{pmatrix}
				1&x\\0&1
			\end{pmatrix}
		\end{array}$

		Soient $x,y \in \reels$.
		\begin{nlist}
			\item $f(x+y) = \begin{pmatrix}
				1&x+y\\0&1
			\end{pmatrix} = \begin{pmatrix}
			1&x\\0&1
			\end{pmatrix} \begin{pmatrix}
			1&y\\0&1
			\end{pmatrix} = f(x) \cdot f(y)$
			\item
			\begin{proof}[montrons que]
				$f$ est bijective.

				\begin{ulist}
					\item $f$ est injective

					Soient $x,y \in \reels$.

					Supposons $f(x)=f(y)$
					\begin{align*}
						\begin{pmatrix}
							1&x\\0&1
						\end{pmatrix}&= \begin{pmatrix}
							1&y\\0&1
						\end{pmatrix}\\
						x&= y
					\end{align*}
					\item $f$ est surjective

					Soit $Y = \begin{pmatrix}
						1&y\\0&1
					\end{pmatrix} \in H$, avec $y \in \reels$.

					$Y = f(y)$.
				\end{ulist}
			\end{proof}
		\end{nlist}
	\end{exem}

	\subsection{Sous-groupes}
	\begin{defin}
		$H \subseteq G$ est un \emph{sous-groupe} de $G$ si $H$ est un groupe pour la m\^eme op\'eration que $G$.
	\end{defin}
	\begin{exem}
		~

		\begin{ulist}
			\item $\{e\} \subseteq G$ est un sous-groupe.
			\item $G \subseteq G$ est un sous-groupe.
			\item $\left\lbrace \dotsc, -4, -2, 0, 2, 4, \dotsc \right\rbrace = 2\entiers \subseteq (\entiers,+)$
			\item Dans $\entiers_4 = \{\overline0, \overline1, \overline2, \overline3\}$, $\{\overline0, \overline2\}$ est un sous-groupe.
			\renewcommand{\arraystretch}{1.5}
			\[
			\begin{array}{c||c|c}
				+&\overline0&\overline2\\
				\hline\hline
				\overline0&\overline0&\overline2\\
				\hline
				\overline2&\overline2&\overline0
			\end{array}
			\]
			\renewcommand{\arraystretch}{1}

			Ce groupe est isomorphe \`a $\entiers_2$ et \`a $C_2=\left( \{-1,1\}, \times \right)$.
			\item $(\entiers,+) \subseteq (\rationels,+) \subseteq (\reels,+)$.
			\item $C_2 \subseteq \rationels_* \subseteq \reels_*$.
			\item $\mathbb{D}_3 = \{\varepsilon, \alpha, \beta, \gamma, \rho, \sigma\}$.

			$\{\varepsilon, \alpha\}$ et $\{\varepsilon, \rho, \sigma\}$ sont des sous-groupes de $\mathbb{D}_3$.
		\end{ulist}
		\begin{nota}
			On note l'ensemble $m\entiers = \{m \cdot n \mid n \in \entiers\}$.
		\end{nota}
	\end{exem}

	\cours
	\begin{rappel}
		~

		\begin{ulist}[noitemsep]
			\item $a \in G$.
			\begin{ulist}
				\item $a^n=\underbrace{a*a*\dotsb*a}_{\text{$n$ fois}}$
				\item $a^n=a*a^{n-1}$
				\item $a^0=e$
				\item $a^{-n}=(a^{-1})^n$
			\end{ulist}
			\item Sous-groupe de $(G,*):H \subseteq G$ qui est un groupe pour $*$.
			\begin{exem}
				$\entiers \subseteq \rationels \subseteq \reels$ pour $+$.
			\end{exem}
			\begin{exem}
				\[
				\left\lbrace \begin{pmatrix}
					1&0\\0&1
				\end{pmatrix}, \begin{pmatrix}
					0&1\\1&0
				\end{pmatrix} \right\rbrace
				\]
				est un sous-groupe de $GL(2,\reels)$.
			\end{exem}
		\end{ulist}
	\end{rappel}
	\begin{prop}
		$H \subseteq G$ un sous-groupe.
		\begin{nlist}[itemsep=1pt]
			\item Si $G$ est ab\'elien, alors $H$ est ab\'elien;
			\item Le neutre de $H$ est le neutre de $G$;
			\item Si $a \in H$, son inverse $a^{-1}\in H$ est l'inverse de $a$ dans $G$.
		\end{nlist}
		\begin{proof}~

			\begin{nlist}[itemsep=1pt]
				\item $G$ est ab\'elien, alors $\forall a,b \in G$, $ab=ba$.

				En particulier, $\forall a,b \in H$, $ab=ba$.
				\item Le neutre de $G$ $e_G$ a la propri\'et\'e que $\forall a \in G$, $e_Ga=a=ae_G$.

				Comme $H \subseteq G$, cette propri\'et\'e est vraie pour $H$ aussi.

				Donc, $ae_G=a=e_Ga$.

				Ainsi, $e_G$ est le neutre de $H$, par l'unicit\'e de l'\'el\'ement neutre.
				\item $a \in H$, il existe un inverse $b \in G$ pour $a$ t.q. $ab=ba=e$.

				Comme $H$ est un groupe, $\exists! a^{-1} \in H$.

				De $ab=e$, on a $a^{-1}ab=a^{-1}e$, donc $b=a^{-1}$.
			\end{nlist}
		\end{proof}
	\end{prop}
	\begin{thm}~

		Un sous-ensemble non-vide $H \subseteq G$ est un sous-groupe si, et seulement si, pour tous $a,b \in H$, $ab^{-1} \in H$.
		\begin{proof}~

			\begin{itemize}
				\item[$(\Rightarrow)$] Supposons que $H$ est un sous-groupe, donc $a,b \in H$, alors $b^{-1} \in H$.

				De plus, $H$ est ferm\'e pour la multiplication, donc $ab^{-1} \in H$.
				\item[$(\Leftarrow)$]
				\begin{itemize}
					\item[$(N)$] $H$ est non-vide, donc $\exists a \in H$.

					Par hypoth\`ese, $aa^{-1}=e \in H$.
					\item[$(I)$] On vient de montrer que $e \in H$.

					Soit $b \in H$ quelconque. Par hypoth\`ese, $eb^{-1}=b^{-1} \in H$.
					\item[$(A)$] On sait que $\forall a,b,c \in G$, $(ab)c=a(bc)$.

					En particulier, $\forall a,b,c \in H$, $(ab)c=a(bc)$.
				\end{itemize}

				Finalement, $H$ est ferm\'e pour l'op\'eration de $G$, car $\forall a,b \in H$, $b^{-1} \in H$.

				Donc, par hypoth\`ese, $a(b^{-1})^{-1}=ab \in H$.
			\end{itemize}
		\end{proof}
	\end{thm}
	\begin{exem}~

		Soit $m \in \entiers$.

		Posons $H = m\entiers = \{mn \mid n \in \entiers\} = \{\dotsb,-2m,-m,0,m,2m,\dotsb\}$, muni de l'addition.
		\begin{proof}[montrons que]
			$H$ est un sous-groupe de $(\entiers,+)$.

			$H$ est non-vide, car $m0=0 \in H$.

			Soient $a,b \in H$.

			Par d\'efinition de $H$, $\exists a',b' \in \entiers$ t.q. $a=ma'$ et $b=mb'$.

			Dans $\entiers$, $b^{-1} = -mb'$.

			On a $a+(-b) = ma' + (-mb') = m(a'-b') \in H$.

			Donc, $H$ est un sous-groupe de $(\entiers,+)$.
		\end{proof}

		R\'eciproquement, soit $H \subseteq \entiers$ un sous-groupe de $\entiers$ quelconque. Alors $\exists m \in \entiers$ t.q. $H=m\entiers$.
		\begin{proof}~

			On sait que $0 \in H$.

			Si $H=\{0\}$, alors $H=0\entiers$ et l'\'enonc\'e est vrai.

			Sinon, $H$ contient un autre \'el\'ement $a \in H$, donc $-a \in H$.

			En particulier, $H$ contient au moins un entier positif.

			Soit $m$ le \emph{plus petit} \'el\'ement positif de $H$.

			Soit $h \in H$ quelconque. On divise $h$ par $m$, donc $h=qm+r$, o\`u $q,r \in \entiers$ et $0 \lte r < m$.

			Si $r=0$, $h=qm \in m\entiers$.

			Sinon, comme $h \in H$ et $m \in H$, $h-qm \in H$, mais $h-qm=r$, donc $r \in H$.

			Comme $0<r<m$, il y a une contradiction \`a la d\'efinition de $m$.
			\begin{flushright}
				\lightning
			\end{flushright}

			Ainsi, $H \subseteq m\entiers$. Mais clairement, $m\entiers \subseteq H$, car $m \in H$ et $mn \in H$, donc $H=m\entiers$.
		\end{proof}
	\end{exem}
	\begin{prop}~

		Soit $f:G \to H$ un isomorphisme. Alors, $K \subseteq G$ est un sous-groupe de $G$ si, et seulement si, $f(K)$ est un sous-groupe de $H$.
		\begin{nota}
			$f(K) = \{f(k) \mid k \in K\}$.
		\end{nota}

		\begin{itemize}
			\item[$(\Rightarrow)$] Supposons que $K$ est un sous-groupe de $G$.

			On sait que $e \in K$, alors $f(e)=e \in f(K)$, donc $f(K)$ est non-vide.

			Soient $a,b \in f(K)$. On veut montrer que $ab^{-1} \in f(K)$.

			Alors, $a=f(k)$ et $b=f(k')$, avec $k,k' \in K$.

			Donc, $ab^{-1} = f(k)f(k')^{-1} = f(k)f(k'^{-1}) = f(kk'^{-1})$.

			Comme $kk'^{-1} \in K$, $ab^{-1} \in f(K)$.
			\item[$(\Leftarrow)$] On effectue la m\^eme preuve avec $f^{-1}$ qui est un isomorphisme en remarquant que $f^{-1}(f(k))=k$.
		\end{itemize}
	\end{prop}
	\begin{nota}
		~

		$G\xrightarrow{f}H$ avec $f$ un isomorphisme est \'equivalent \`a $G\xrightarrow{\sim}H$.
	\end{nota}
	\begin{nota}
		~

		$H \subseteq G$ un sous-groupe est \'equivalent \`a $H \lte G$.
	\end{nota}
	\begin{prop}
		~

		Soit $\left\lbrace H_i \right\rbrace_{i \in I}$ une collection de sous-groupes de $G$. Alors $\bigcap\limits_{i \in I}H_i \lte G$.
		\begin{proof}~

			$H_i \lte G, \forall i \in I$.

			Alors, $e \in H_i, \forall i$. Donc, $e \in \bigcap\limits_{i \in I}H_i$. En particulier, $\bigcap\limits_{i \in I}H_i \neq \emptyset$.

			Soient $a,b\in \bigcap\limits_{i \in I}H_i$. Alors, $a,b \in H_i, \forall i$.

			Comme $H_i \lte G$, $ab^{-1} \in H_i, \forall i$, donc $ab^{-1} \in \bigcap\limits_{i \in I}H_i$.
		\end{proof}
	\end{prop}
	\begin{rema}
		Si $H_1,H_2 \lte G$, $H_1 \cup H_2$ n'est pas n\'ecessairement un sous-groupe de $G$.
		\begin{exem}
			$2\entiers \lte \entiers$ et $3\entiers \lte \entiers$. $2\entiers \cup 3\entiers$ n'est pas un sous-groupe de $\entiers$.

			Plus pr\'ecis\'ement, $2,3 \in 2\entiers \cup 3\entiers$, mais $2+3=5 \not\in 2\entiers \cup 3\entiers$.
		\end{exem}
	\end{rema}
	\begin{exem}
		$2\entiers \cap 3\entiers = 6\entiers \lte \entiers$.
	\end{exem}

	\setcounter{chapter}{1}
	\chapter{Applications et \'equivalences}
	\setcounter{section}{3}
	\section{Relations d'\'equivalence}
	\begin{defin}
		Une \emph{relation d\'equivalence} sur un ensemble $E$ est un sous-ensemble $R \subseteq E \times E$ satisfaisant
		\begin{nlist}
			\item r\'eflexivit\'e

			$x \sim x, \forall x \in E$.
			\item sym\'etrie

			$x \sim y \Rightarrow y \sim x$.
			\item transitivit\'e

			$x \sim y$ et $y \sim z$, alors $x \sim z$.
		\end{nlist}
		\begin{nota}
			On note $x \sim y$ si, et seulement si, $(x,y) \in R$.
		\end{nota}
	\end{defin}
	\begin{exem}~

		\begin{nlist}
			\item $E = \reels$

			$x \sim y$ si, et seulement si, $\abs{x} = \abs{y}$
			\begin{itemize}
				\item[(refl)] $\abs{x} = \abs{x}$, donc $x \sim x$.
				\item[(sym)] Supposons $x \sim y$. Alors, $\abs{x} = \abs{y}$. Donc, $y \sim x$.
				\item[(trans)] Supposons $x \sim y$ et $y \sim z$. Alors, $\abs{x} = \abs{y}$ et $\abs{y} = \abs{z}$. Donc $\abs{x} = \abs{z}$. Ainsi, $x \sim z$.
			\end{itemize}
			\item $C$ est l'ensemble des \'el\`eves dans la classe.

			$x \sim y$ si, et seulement si, $x$ et $y$ ont le m\^eme \^age est une relation d'\'equivalence.
		\end{nlist}
	\end{exem}
	\begin{defin}
		Si $E$ est un ensemble et $\sim$ est une relation d'\'equivalence sur $E$, la \emph{classe d'\'euivalence} de $x \in E$ est $\overline{x} = \{y \in E \mid y \sim x\} \subseteq E$.
	\end{defin}
	\begin{lem}
		$\overline{x} = \overline{y}$ si, et seulement si, $x \sim y$.
		\begin{proof}~

			\begin{itemize}
				\item[$(\Rightarrow)$] Supposons $\overline{x} = \overline{y}$.

				Par (refl), $x \sim x$, donc $x \in \overline{x}$. Alors, $x \in \overline{y}$. Ainsi, $x \sim y$.
				\item[$(\Leftarrow)$] Supposons $x \sim y$.
				\begin{itemize}
					\item[$(\subseteq)$] Soit $z \in \overline{x}$. Alors $z \sim x$. Comme $x \sim y$, par (trans), $z \sim y$. Donc, $z \in \overline{y}$.
					\item[$(\supseteq)$] Soit $z \in \overline{y}$. Alors, $z \sim y$ et, par (sym), $y \sim z$. Comme $x \sim y$, par (trans), $x \sim z$ et, par (sym), $z \sim x$. Donc, $z \in \overline{x}$.
				\end{itemize}

				Ainsi, $\overline{x} = \overline{y}$.
			\end{itemize}
		\end{proof}
	\end{lem}

	\cours
	\begin{rappel}~

		\begin{ulist}[noitemsep]
			\item $H \lte G$, $H$ un sous-groupe de $G$, si, et seulement si,
			\begin{ulist}
				\item $H \neq \emptyset$;
				\item $\forall a,b \in H, ab^{-1} \in H$.
			\end{ulist}
			\item Si $H \lte G$, $H$ a le m\^eme neutre que $G$, m\^emes inverses.
			\item $G$ ab\'elien $\Rightarrow H \lte G$ ab\'elien.
			\item Si $f:G \to H$ isomorphisme, alors $K \lte G \Leftrightarrow f(K) \lte H$.
			\item Relations d'\'equivalence $\sim$ sur $E$:
			\begin{itemize}
				\item[(Refl)] $a \sim a, \forall a \in E$;
				\item[(Sym)] $a \sim b \Rightarrow b \sim a$;
				\item[(Trans)] $a \sim b, b \sim c \Rightarrow a \sim c$.
			\end{itemize}
			\item Classe d'\'equivalence: $\overline{a} = \{b \in E \mid b \sim a\}$.
			\item $a \sim b \Longleftrightarrow \overline{a} = \overline{b}$.
		\end{ulist}
	\end{rappel}
	\begin{exem}
		$E=\entiers$.

		\'Equivalence modulo $m$:

		$a \sim b$ si, et seulement si, $a-b=km$, avec $k \in \entiers$.
		\begin{nota}
			$m \mid a-b$, $m$ divise $a-b$: $\exists k \in \entiers$ t.q. $a-b=km$.
		\end{nota}

		$a \sim b \Longleftrightarrow m \mid a-b$.
		\begin{proof}
			que c'est bel et bien une \'equivalence
			\begin{itemize}
				\item[(Refl)] Soit $a \in \entiers$.

				$a-a=0m \Rightarrow a \sim a$.
				\item[(Sym)] Supposons que $a \sim b$, avec $a,b \in \entiers$.

				Alors, $a-b=km$, avec $k \in \entiers$.

				Or, $-(a-b)=-km \Rightarrow b-a=(-k)m$, donc $b \sim a$.
				\item[(Trans)] Supposons que $a \sim b$ et $b \sim c$, avec $a,b,c \in \entiers$.

				Alors, $a-b=k_1m$ et $b-c=k_2m$, avec $k_1,k_2 \in \entiers$.

				En additionant les deux \'equations, on obtient
				\begin{align*}
					a-c&= k_1m+k_2m\\
					&= (k_1+k_2)m
				\end{align*}

				Donc, $a \sim c$.
			\end{itemize}
		\end{proof}

		Si $m=2$, les classes d'\'equivalence sont
		\begin{align*}
			\begin{split}
				\overline{0}&= \{a \in \entiers \mid a \sim 0\}\\
				&= \{a \in \entiers \mid a-0=2k\}\\
				&= \{a \in \entiers \mid a=2k\}\\
				&= 2\entiers
			\end{split}
			&
			\begin{split}
				\overline{1}&= \{a \in \entiers \mid a \sim 1\}\\
				&= \{a \in \entiers \mid a-1=2k\}\\
				&= \{a \in \entiers \mid a=2k+1\}\\
				&= \entiers\setminus2\entiers
			\end{split}
		\end{align*}
		\begin{rema}
			\begin{align*}
				\begin{split}
					\overline{0} = \overline{2} = \overline{4} = \overline{-2} = \dotsb
				\end{split}
				&
				\begin{split}
					\overline{1} = \overline{3} = \overline{5} = \overline{-1} = \dotsb
				\end{split}
			\end{align*}
		\end{rema}

		Plus g\'en\'eralement, pour $m\entiers$, on a $m$ classes d'\'equivalence.
		\[
		\overline{0}, \overline1, \overline2, \dotsb, \overline{m-1}
		\]
		\begin{nota}
			L'ensemble des classes d'\'equivalence est not\'e $\entiers_m = \{\overline{0}, \overline1, \overline2, \dotsb, \overline{m-1}\}$ pour la relation de congruence modulo $m$.
		\end{nota}
	\end{exem}
	\begin{defin}
		Une \emph{partition} d'un ensemble $E$ est une collection $\mathcal{P} = \{E_i\}$, avec $i \in I$ de sous-ensembles de $E$ t.q.
		\begin{nlist}
			\item $\bigcup\limits_{i \in I} E_i = E$;
			\item $E_i \cap E_j = \emptyset$, si $i \neq j$.
		\end{nlist}
		\begin{rema}
			Chaque $x \in E$ est \'el\'ement d'exactement un $E_i$.
		\end{rema}
	\end{defin}
	\begin{prop}
		Si $\sim$ est une relation d'\'equivalence sur $E$, alors $\mathcal{P} = \{\overline{a} \mid a \in E\}$ est une partition de $E$.
		\begin{exem}
			$E = \entiers$, $\sim$ \'equivalence modulo 3.

			$\mathcal{P} = \{\overline{0}, \overline{1}, \overline{2}\}$ est une partition de $\entiers$.
		\end{exem}
		\begin{proof}~

			\begin{nlist}
				\item Clairement, $\bigcup\limits_{a \in E}\overline{a} \subseteq E$.

				On veut montrer que $E \subseteq \bigcup\limits_{a \in E}\overline{a}$.

				Soit $x \in E$, alors $x \sim x$ par r\'eflexivit\'e, donc $x \in \overline{x}$ et $x \in \bigcup\limits_{a \in E}\overline{a}$.
				\item Supposons que $x \in \overline{a}$ et $x \in \overline{b}$, avec $\overline{a} \neq \overline{b}$.

				Alors, $x \sim a$ et $x \sim b$. Donc, par sym\'etrie, $a \sim x$ et $x \sim b$. Donc, par transitivit\'e, $a \sim b$. Donc $\overline{a} = \overline{b}$. Ceci est une contradiction, donc $\overline{a} \cap \overline{b} = \emptyset, \forall \overline{a} \neq \overline{b} \in \mathcal{P}$.
			\end{nlist}
		\end{proof}
	\end{prop}

	On d\'efinit une op\'eration sur $\entiers_m$ pour la congruence modulo $m$.

	\[
	\begin{array}{rcl}
		+:\entiers_m \times \entiers_m& \to& \entiers_m\\
		(\overline{a}, \overline{b})& \mapsto& \overline{a+b}
	\end{array}
	\]

	Autrement dit, $\overline{a} + \overline{b} = \overline{a+b}$.
	\begin{rema}
		L'\'ecriture d'un \'el\'ement $\overline{a}$ n'est pas unique ($\overline{a} = \overline{a'}$ si $a \sim a'$).
	\end{rema}

	Il faut v\'erifier que l'op\'eration $+$ est correctement d\'efinie (d\'efinie sans abigu\"\i t\'e).

	Autrement dit, si $\overline{a} = \overline{a'}$ et $\overline{b} = \overline{b'}$, on veut montrer que $\overline{a+b} = \overline{a'+b'}$.

	\cours
	\begin{rappel}
		~

		\begin{ulist}[noitemsep]
			\item Relation d'\'equivalence ($\sim$) $\rightarrow$ partition en classes d'\'equivalence ($\overline{a}=\{b \mid b \sim a\}$).
			\item \'Equivalence (congruence)$\mod m$ (sur $\entiers$):
			\begin{align*}
				a \sim b&\Leftrightarrow m \mid a-b\\
				&\Leftrightarrow \exists k \in \entiers \text{ t.q. } a-b=km
			\end{align*}

			On note l'ensemble des classes d'\'equivalence$\mod m$ par $\entiers_m = \{\overline{a} \mid a \in \entiers\} = \{\overline{0}, \overline{1}, \dotsb, \overline{m-1}\}$.

			On veut d\'efinir une op\'eration $+$ sur $\entiers_m$ par $\overline{a} + \overline{b} = \overline{a+b}$.

			On doit v\'erifier que cette d\'efinition n'est pas ambigu\"e (ne d\'epend pas des repr\'esentants).

			Supposons $\overline{a_1} = \overline{a_2}$ et $\overline{b_1} = \overline{b_2}$. On doit v\'erifier que $\overline{a_1+b_1} = \overline{a_2+b_2}$, c'est-\`a-dire que $a_1+b_1 \sim a_2+b_2$.

			Les hypoth\`eses donnent: $a_1-a_2=k_am$ et $b_1-b_2=k_bm$. En additionnant ces deux \'equations, on obtient $(a_1-a_2)+(b_1-b_2) = k_am+k_bm$. Alors, $(a_1+b_1) - (a_2+b_2) = (k_a+k_b)m$. Donc, $a_1+b_1 \sim a_2+b_2$.
			\begin{flushright}
				$\square$
			\end{flushright}
			\begin{rema}
				On doit faire ce genre de preuve pour chaque d\'efinition de fonction/op\'eration qui ont comme domaine des classes d'\'equivalence.
			\end{rema}
		\end{ulist}
	\end{rappel}
	\begin{exem}
		$m=5$.

		\begin{center}
			\begin{tikzpicture}
				\node at (0,0) {0};
				\node at (-.5,-.5) {-5};
				\node at (.5,-.5) {5};
				\node at (-.5,-1) {-10};
				\node at (.5,-1) {10};
				\node at (0,-1.5) {$\vdots$};
				\draw (0,-.75) ellipse (1 and 1.5);
				\node at (1,.5) {$\overline{0}$};
			\end{tikzpicture}
			\qquad
			\begin{tikzpicture}
				\node at (0,0) {1};
				\node at (-.5,-.5) {-4};
				\node at (.5,-.5) {6};
				\node at (-.5,-1) {-9};
				\node at (.5,-1) {11};
				\node at (0,-1.5) {$\vdots$};
				\draw (0,-.75) ellipse (1 and 1.5);
				\node at (1,.5) {$\overline{1}$};
			\end{tikzpicture}
			\qquad
			\begin{tikzpicture}
				\node at (0,0) {2};
				\node at (-.5,-.5) {-3};
				\node at (.5,-.5) {7};
				\node at (-.5,-1) {-8};
				\node at (.5,-1) {12};
				\node at (0,-1.5) {$\vdots$};
				\draw (0,-.75) ellipse (1 and 1.5);
				\node at (1,.5) {$\overline{2}$};
			\end{tikzpicture}
			\qquad
			\begin{tikzpicture}
				\node at (0,0) {3};
				\node at (-.5,-.5) {-2};
				\node at (.5,-.5) {8};
				\node at (-.5,-1) {-7};
				\node at (.5,-1) {13};
				\node at (0,-1.5) {$\vdots$};
				\draw (0,-.75) ellipse (1 and 1.5);
				\node at (1,.5) {$\overline{3}$};
			\end{tikzpicture}
			\qquad
			\begin{tikzpicture}
				\node at (0,0) {4};
				\node at (-.5,-.5) {-1};
				\node at (.5,-.5) {9};
				\node at (-.5,-1) {-6};
				\node at (.5,-1) {14};
				\node at (0,-1.5) {$\vdots$};
				\draw (0,-.75) ellipse (1 and 1.5	);
				\node at (1,.5) {$\overline{4}$};
			\end{tikzpicture}
		\end{center}

		Pour faire $\overline{2}+\overline{1}$, on peut prendre $\overline{2+1} = \overline{2}$, ou bien $\overline{17+(-4)} = \overline{13}$.
	\end{exem}

	\begin{prop}
		$(\entiers_m,+)$ est un groupe ab\'elien.
		\begin{proof}~

			\begin{itemize}
				\item[(A)] Soient $a,b,c \in \entiers_m$.
				\begin{align*}
					\left( \overline{a} + \overline{b} \right) + \overline{c}&= \overline{a+b} + \overline{c}\\
					&= \overline{(a+b)+c}\\
					&= \overline{a+(b+c)}\\
					&= \overline{a} + \overline{b+c}\\
					&= \overline{a} + \left( \overline{b} + \overline{c} \right)
				\end{align*}
				\item[(C)] $\overline{a} + \overline{b} = \overline{a+b} = \overline{b+a} = \overline{b} + \overline{a}$.
				\item[(N)] $\overline{0} + \overline{a} = \overline{0+a} = \overline{a}$, donc $\overline{0}$ est neutre. Par commutativit\'e, la propri\'et\'e est satisfaite.
				\item[(I)] $\overline{a} + \overline{-a} = \overline{a-a} = \overline{0}$, donc $\overline{-a}$ est l'inverse de $\overline{a}$. Par commutativit\'e, la propri\'et\'e est satisfaite.

				On peut donc \'ecrire $-\overline{a} = \overline{-a}$.
			\end{itemize}
		\end{proof}
	\end{prop}

	\subsection{Ordre et groupes cycliques}
	\begin{defin}
		Soient $G$ un groupe et $a \in G$.

		L'\emph{ordre} de $a$, not\'e $o(a)$ est la plus petite quantit\'e positive $k \in \naturels_*$ t.q. $a^k=e$, si elle existe. Sinon, on note $o(a) = \infty$.
	\end{defin}
	\begin{exem}~

		\begin{ulist}
			\item $o(e)=1$
			\item Dans $\mathbb{D}_3$
			\begin{ulist}
				\item $o(\alpha) = 2$
				\item $o(\rho) = 3$
			\end{ulist}
			\item Dans $\entiers$
			\begin{ulist}
				\item $o(0) = 1$
				\item $o(n) = \infty$, $\forall n\neq0$
			\end{ulist}
			\item Dans $\entiers_6$
			\begin{ulist}
				\item $o(\overline2) = 3$
			\end{ulist}
		\end{ulist}
	\end{exem}
	\begin{prop}
		Soit $m \in \entiers$. $a^m = e$, si, et seulement si, $o(a) \mid m$.
		\begin{proof}~

			\begin{itemize}
				\item[$(\Rightarrow)$] Supposons $a^m=e$.

				On divise $m$ par $o(a)$: $m = q \cdot o(a) + r$, avec $0 \lte r < o(a)$.

				Si $r=0$, $o(a) \mid m$ et on a termin\'e.

				Supposons que $0<r<o(a)$, alors
				\begin{align*}
					r&= m-q \cdot o(a)\\
					a^r&= a^{m-q \cdot o(a)}\\
					&= a^m \cdot a^{-q \cdot o(a)}\\
					&= e \cdot (a^{o(a)})^{-q}\\
					&= e \cdot e^{-q}\\
					&= e
				\end{align*}
				mais $0<r<o(a)$ contredit la minimalit\'e de $o(a)$.
			\end{itemize}
		\end{proof}
	\end{prop}

	\cours
	\begin{rappel}
		~

		\begin{ulist}
			\item $a \sim b \Leftrightarrow m \mid a-b$ est une relation d'\'equivalence sur $\entiers$.
			\item $\entiers_m = \{\overline0, \overline1, \dotsb, \overline{m-1}\}$ sont les classes d'\'equivalence.
			\item $(\entiers_m, +)$ est un groupe, o\`u $\overline{a} + \overline{b} = \overline{a+b}$.
			\item $o(a) = \min\{k \in \naturels_* \mid a^k = e\}$ ordre de $a$. Si $a^k \neq e$, $\forall k>0$, $o(a)=\infty$.
			\item Si $a^k=e$, $o(a) \mid k$.
		\end{ulist}
	\end{rappel}
	\begin{exem}
		~

		$M=\begin{pmatrix}
			1&-1\\1&0
		\end{pmatrix} \in GL(2, \reels)$.

		\begin{align*}
			\begin{split}
				M^2&= \begin{pmatrix}
					0&-1\\1&-1
				\end{pmatrix}\\
				M^3&= \begin{pmatrix}
					-1&0\\0&-1
				\end{pmatrix}\\
				&= -\begin{pmatrix}
					1&0\\0&1
				\end{pmatrix}
			\end{split}
			&
			\begin{split}
				M^4&= -M\\
				M^5&= -\begin{pmatrix}
					0&-1\\1&-1
				\end{pmatrix}\\
				M^6&= \left( -\begin{pmatrix}
					1&0\\0&1
				\end{pmatrix} \right) \left( -\begin{pmatrix}
					1&0\\0&1
				\end{pmatrix} \right)\\
				&= \begin{pmatrix}
					1&0\\0&1
				\end{pmatrix}
			\end{split}
		\end{align*}

		Ainsi, $o(M) = 6$, puisque $M^6 = \begin{pmatrix}
			1&0\\0&1
		\end{pmatrix}$.
	\end{exem}

	\setcounter{chapter}{5}
	\chapter{Groupes (suite)}
	\setcounter{section}{12}
	\section{Groupes sym\'etriques $S_n$}
	\begin{rappel}
		$S(E) = \{f:E \to E \mid f \text{ est bijective}\}$.

		$S(E)$ est un groupe lorsque muni de l'op\'eration $\circ$.

		$\identite_E$ est l'\'el\'ement neutre.

		$f^{-1}$ est l'\'el\'ement inverse de $f$.

		$S(E)$ s'appelle le \emph{groupe sym\'etrique} de l'ensemble $E$.
	\end{rappel}
	\begin{rema}
		~

		$E = \{$rouge, orange, jaune, vert, bleu, indigo, violet$\}$.

		$S(E) \ni f:\left\lbrace \begin{array}{rcl}
			\text{rouge}&\mapsto&\text{jaune}\\
			\text{jaune}&\mapsto&\text{rouge}\\
			\text{autres}&\mapsto&\text{elles-m\^emes}
		\end{array} \right.$.

		Le groupe $S(E)$ est isomorphe au groupe $S(\{1,2,3,4,5,6,7\})$, en num\'erotant les \'el\'ements.
	\end{rema}
	\begin{defin}
		$S_n = S(\{1,2,\dotsb,n\})$ le groupe sym\'etrique de $n$ \'el\'ements.

		Les \'el\'ements de $S_n$ sont des bijections, aussi appel\'ees permutations.
	\end{defin}
	\begin{nota}
		On note une permutation $\sigma$ de la mani\`ere suivante:
		\[
		\sigma=\begin{pmatrix}
			1&2&3&\dotsb&n\\
			\sigma(1)&\sigma(2)&\sigma(3)&\dotsb&\sigma(n)
		\end{pmatrix}
		\]
	\end{nota}
	\begin{exem}
		\[
		\sigma=\begin{pmatrix}
			1&2&3&4\\
			2&1&4&3
		\end{pmatrix}
		\Longleftrightarrow
		\begin{array}{rcl}
			1&\mapsto&2\\
			2&\mapsto&1\\
			3&\mapsto&4\\
			4&\mapsto&3
		\end{array}
		\]
		\[
		\sigma_1=\begin{pmatrix}
			1&2&3&4\\
			2&1&3&4
		\end{pmatrix}
		,\quad
		\sigma_2=\begin{pmatrix}
			1&2&3&4\\
			1&2&4&3
		\end{pmatrix}
		\Longrightarrow
		\sigma_1 \circ \sigma_2=\begin{pmatrix}
			1&2&3&4\\
			2&1&4&3
		\end{pmatrix}=\sigma
		\]
		\[
		\sigma^{-1}=\begin{pmatrix}
			1&2&3&4\\
			2&1&4&3
		\end{pmatrix}=\sigma
		\]
		\[
		\eta=\begin{pmatrix}
			1&2&3&4\\
			2&3&4&1
		\end{pmatrix}
		,\quad \eta^{-1}=\begin{pmatrix}
			1&2&3&4\\
			4&1&2&3
		\end{pmatrix}
		\]

		L'\'el\'ement neutre est $e=\begin{pmatrix}
			1&2&3&\dotsb&n\\
			1&2&3&\dotsb&n
		\end{pmatrix}$.

		$S_2 = \left\lbrace \begin{pmatrix}
			1&2\\1&2
		\end{pmatrix}, \begin{pmatrix}
			1&2\\2&1
		\end{pmatrix} \right\rbrace \cong \entiers_2$.

		$S_3 = \left\lbrace \begin{pmatrix}
			1&2&3\\1&2&3
		\end{pmatrix}, \begin{pmatrix}
			1&2&3\\2&1&3
		\end{pmatrix}, \begin{pmatrix}
			1&2&3\\1&3&2
		\end{pmatrix}, \begin{pmatrix}
			1&2&3\\3&2&1
		\end{pmatrix}, \begin{pmatrix}
			1&2&3\\2&3&1
		\end{pmatrix}, \begin{pmatrix}
			1&2&3\\3&1&2
		\end{pmatrix} \right\rbrace \cong \mathbb{D}_3$.
	\end{exem}
	\begin{rema}
		Tous les \'el\'ement de $S_3$ donnent des isom\'etries du triangle, mais pas tous les \'el\'ements de $S_4$ donnent des isom\'etries du carr\'e.
	\end{rema}
	\begin{prop}
		$S_n$ contient $n!$ \'el\'ements.
		\begin{proof}[id\'ee]~

			\[
			\begin{matrix}
				1&2&3&&n\\
				\downarrow&\downarrow&\downarrow&\dotsb&\downarrow\\
				n\text{ choix}&n-1\text{ choix}&n-2\text{ choix}&&1\text{ choix}
			\end{matrix}
			\]

			Ainsi, $n(n-1)(n-2)\dotsb1 = n!$.
			\renewcommand{\qedsymbol}{\#}
		\end{proof}
		\renewcommand{\qedsymbol}{$\square$}
	\end{prop}
	\begin{defin}
		Un \emph{cycle} est une permutation de la forme $a_1 \to a_2 \to \dotsb \to a_l \to a_1$, o\`u $a_i \neq a_j$ quand $i \neq j$.
		\begin{center}
			\begin{tikzpicture}
			\node (1) at (0,0) {$a_1$};
			\node (2) [below right of=1] {$a_2$};
			\node (r) [below left of=2] {$\dotsb$};
			\node (l) [above left of=r] {$a_l$};

			\path[->,bend left]	(1) edge (2)
								(2) edge (r)
								(r) edge (l)
								(l) edge (1);
		\end{tikzpicture}
		\end{center}
	\end{defin}
	\begin{exem}
		$\begin{pmatrix}
			1&2&3\\2&3&1
		\end{pmatrix}$ est un cycle de longueur $l=3$.
		\begin{tikzpicture}
			\node (1) at (0,0) {1};
			\node (2) [below right of=1] {2};
			\node (3) [below left of=1] {3};

			\path[->,bend left]	(1) edge (2)
								(2) edge (3)
								(3) edge (1);
		\end{tikzpicture}

		$\begin{pmatrix}
			1&2&3&4&5\\
			1&4&3&5&2
		\end{pmatrix}$ est un cycle de longueur $l=3$.
		\begin{tikzpicture}
			\node (1) at (0,0) {1};
			\node (2) at (2.5,0) {2};
			\node (3) at (1,0) {3};
			\node (4) [below right of=2] {4};
			\node (5) [below left of=2] {5};

			\path[->,bend left]	(2) edge (4)
								(4) edge (5)
								(5) edge (2);
			\path[->,loop below]	(1) edge (1)
									(3) edge (3);
		\end{tikzpicture}
	\end{exem}
	\begin{nota}
		\'Ecriture raccourcie pour un cycle: $\begin{pmatrix}
			1&2&3\\2&3&1
		\end{pmatrix} = \begin{pmatrix}
			1&2&3
		\end{pmatrix}$, $\begin{pmatrix}
			1&2&3&4&5\\
			1&4&3&5&2
		\end{pmatrix} = \begin{pmatrix}
			2&4&5
		\end{pmatrix} \in S_5$.
	\end{nota}
	\begin{prop}
		L'ordre d'un cycle est \'egal \`a sa longueur.
		\begin{proof}~

			Si le cycle $\sigma$ permute les \'el\'ements $a_i$ comme $\sigma = a_1 \xrightarrow{\sigma} a_2 \xrightarrow{\sigma} \dotsb \xrightarrow{\sigma} a_l \xrightarrow{\sigma} a_1$.

			Alors, calculons $\sigma^l$
			\[
			\begin{pNiceMatrix}
				a_1&a_2&a_3&\dotsb&a_l\\
				a_2&a_3&a_4&\dotsb&a_1\\
				a_3&a_4&a_5&\dotsb&a_2\\
				\vdots&\vdots&\vdots&\ddots&\vdots\\
				a_l&a_1&a_2&\dotsb&a_{l-1}\\
				a_1&a_2&a_3&\dotsb&a_l
				\CodeAfter
				\tikz \draw (2.5-|1) -- (2.5-|last) (3.5-|1) -- (3.5-|last) (5.5-|1) -- (5.5-|last);
			\end{pNiceMatrix}
			\]

			Ainsi, $\sigma^l=e$.
		\end{proof}
	\end{prop}
	\begin{exem}
		$\sigma=\begin{pmatrix}
			1&2&3&4\\
			2&1&4&3
		\end{pmatrix}$ n'est pas un cycle.

		Cependant, $\sigma$ se d\'ecompose en cycles $\sigma=\sigma_1 \circ \sigma_2$, o\`u $\sigma_1 = \begin{pmatrix}
			1&2
		\end{pmatrix}, \sigma_2 = \begin{pmatrix}
			3&4
		\end{pmatrix}$.
	\end{exem}
	\begin{prop}
		Toute permutation $\sigma \in S_n$ s'\'ecrit de mani\`ere unique comme une composition de \emph{cycles disjoints}. (les cycles sont uniques, mais pas l'ordre de l'\'ecriture).
		\begin{nota}
			Des cycles disjoints sont des cycles de \emph{supports} disjoints, o\`u le support d'un cycle est $\mathrm{supp}(\sigma)\{i \mid \sigma(i) \neq i\}$.
		\end{nota}
		\begin{lem}
			Les cycles disjoints commutent entre eux.
			\begin{exem}
				\begin{align*}
					\begin{split}
						\begin{pmatrix}
							1&4&5
						\end{pmatrix} \circ \begin{pmatrix}
							2&3
						\end{pmatrix}&= \begin{pNiceMatrix}
							1&2&3&4&5\\
							1&3&2&4&5\\
							4&3&2&5&1
							\CodeAfter
							\tikz \draw (2.4-|1) -- (2.4-|last);
						\end{pNiceMatrix}
					\end{split}
					&=&
					\begin{split}
						\begin{pmatrix}
							2&3
						\end{pmatrix} \circ \begin{pmatrix}
						1&4&5
						\end{pmatrix}&= \begin{pNiceMatrix}
							1&2&3&4&5\\
							4&2&3&5&1\\
							4&3&2&5&1
							\CodeAfter
							\tikz \draw (2.4-|1) -- (2.4-|last);
						\end{pNiceMatrix}
					\end{split}
				\end{align*}
			\end{exem}
			\begin{proof}~

				Soient $\sigma=\begin{pmatrix}
					a_1&a_2&\dotsb&a_l
				\end{pmatrix}$ et $\eta=\begin{pmatrix}
				b_1&b_2&\dotsb&b_k
				\end{pmatrix}$ deux cycles disjoints.

				Soit $i \in \{1,2,\dotsb,n\}$.

				Il y a trois cas:
				\begin{nlist}
					\item $i \in \mathrm{supp}(\sigma)$, $i \notin \mathrm{supp}(\eta)$.
					\begin{align*}
						\sigma \circ \eta(i)&= \sigma(i)\\
						\eta \circ \sigma(i)&= \sigma(i)
					\end{align*}
					\item $i \notin \mathrm{supp}(\sigma)$, $i \in \mathrm{supp}(\eta)$.
					\begin{align*}
						\sigma \circ \eta(i)&= \eta(i)\\
						\eta \circ \sigma(i)&= \eta(i)
					\end{align*}
					\item $i \notin \mathrm{supp}(\sigma)$, $i \notin \mathrm{supp}(\eta)$.
					\begin{align*}
						\sigma \circ \eta(i)&= i\\
						\eta \circ \sigma(i)&=i
					\end{align*}
				\end{nlist}
			\end{proof}
		\end{lem}
		\begin{proof}[id\'ee de la d\'emonstration]
			Par r\'ecurrence.

			$n=2$, $\begin{pmatrix}
				1&2
			\end{pmatrix}$.

			Si vrai pour toutes les permutations de longueur $\lte n$.

			$\sigma$ permutation de $n+1$ \'el\'ements.

			On prend $i \in \mathrm{supp}(\sigma)$.
			\[
			\underbrace{i \to \sigma(i) \to \sigma^2(i) \to \sigma^3(i) \to \dotsb \to \sigma^n(i)}_{n+1\text{ \'el\'ements de }\{1,2,\dotsb,n\}}
			\]
			ces \'el\'ements ne peuvent pas tous \^etre distincts.

			$\exists m_1>m_2$ t.q. $\sigma^{m_1}(i) = \sigma^{m_2}(i)$, donc $\sigma^{m_1-m_2}(i) = i$.

			Alors, $i \to \sigma(i) \to \sigma^2(i) \to \dotsb \to \sigma^{m_1-m_2-1}(i) \to i$ est un cycle.

			Les \'el\'ements restants $\{1,2,\dotsb,n\} \setminus \{i, \sigma(i), \sigma^2(i), \dotsb, \sigma^{m_1-m_2-1}(i)\}$ sont permut\'es entre eux par $\sigma$.

			Utiliser l'hypoth\`ese de r\'ecurrence.
			\renewcommand{\qedsymbol}{\#}
		\end{proof}
		\renewcommand{\qedsymbol}{$\square$}
	\end{prop}
	\begin{exem}
		\begin{ulist}~

			\item \[
			\begin{pmatrix}
				1&2&3&4&5&6&7\\
				6&2&4&5&7&1&3
			\end{pmatrix} = \begin{pmatrix}
				1&6
			\end{pmatrix} \begin{pmatrix}
				3&4&5&7
			\end{pmatrix}
			\]
			\item \[
			\begin{pmatrix}
				1&2&3&4&5&6&7&8\\
				4&7&2&8&3&1&5&6
			\end{pmatrix} = \begin{pmatrix}
				1&4&8&6
			\end{pmatrix} \begin{pmatrix}
				2&7&5&3
			\end{pmatrix}
			\]
		\end{ulist}
	\end{exem}
	\begin{prop}
		L'ordre d'une permutation $\sigma$ est le \verb|ppcm| des longueurs des cycles dans sa d\'ecomposition.
		\begin{proof}
			$\sigma = \sigma_1 \circ \sigma_2 \circ \dotsb \circ \sigma_k$, o\`u $\sigma_i$ sont des cycles de longueur $l_i$.

			Puisque les cycles disjoints commutent, $\sigma^m = (\sigma_1 \sigma_2 \dotsb \sigma_k)^m = \sigma_1^m \sigma_2^m \dotsb \sigma_k^m$.

			$\sigma_1^m \sigma_2^m \dotsb \sigma_k^m = e$, si, et seulement si, $\sigma_1^m=e$, $\sigma_2^m=e$, $\dotsb$, $\sigma_k^m=e$.

			Alors, $o(\sigma_1) \mid m$, $o(\sigma_2) \mid m$, $\dotsb$, $o(\sigma_k) \mid m$.

			Donc, $l_i \mid m$, $\forall i$.

			L'ordre de $\sigma$ $m$ est le plus petit multiple des $l_i$.
		\end{proof}
	\end{prop}

	\cours
	\begin{rappel}
		~

		\begin{ulist}
			\item $S_n = S(\{1,2,\dots,n\}) = \{\sigma \mid \sigma \text{ est une bijection de \{1,2,\dots,n\}}\}$ est un groupe pour $\circ$.
			\item Notations: $\begin{pmatrix}
				1&2&3&\dotsb&n\\
				\sigma(1)&\sigma(2)&\sigma(3)&\dotsb&\sigma(n)
			\end{pmatrix}$

			Cycle: $\begin{pmatrix}
				a_1&a_2&\dotsb&a_l
			\end{pmatrix}$
			\item $o\begin{pmatrix}
				a_1&a_2&\dotsb&a_l
			\end{pmatrix} = l$
			\item Toute permutation de $S_n$ s'\'ecrit comme un produit (composition) de cycles disjoints uniques
			\item Les cycles disjoints commutent
			\item Si $\sigma = \sigma_1 \dotsb \sigma_k$ est la d\'ecomposition, $o(\sigma) = \verb|ppcm|(l_1, \dots, l_k)$, o\`u $l_k$ est la longueur de $\sigma_k$
		\end{ulist}
	\end{rappel}
	\begin{exem}
		\[
		\sigma = \begin{pmatrix}
			1&2&3&4&5&6&7&8\\
			2&1&5&4&3&7&8&6
		\end{pmatrix} = \begin{pmatrix}
			1&2
		\end{pmatrix} \circ \begin{pmatrix}
			3&5
		\end{pmatrix} \circ \begin{pmatrix}
			6&7&8
		\end{pmatrix}
		\]

		$o(\sigma) = \verb|ppcm|(2,2,3) = 6$.

		\[
		\sigma^2 = \begin{pmatrix}
			1&2
		\end{pmatrix}^2 \circ \begin{pmatrix}
			3&5
		\end{pmatrix}^2 \circ \begin{pmatrix}
			6&7&8
		\end{pmatrix}^2 = \begin{pmatrix}
			6&7&8
		\end{pmatrix}^2 = \begin{pmatrix}
			6&8&7
		\end{pmatrix}
		\]
	\end{exem}
	\begin{defin}
		Le \emph{signe} d'une permutation $\sigma \in S_n$ est le nombre
		\[
		\signe(\sigma) = \prod_{i < j}\dfrac{\sigma(i) - \sigma(j)}{i-j}
		\]
	\end{defin}
	\begin{exem}
		\[
		\sigma = \begin{pmatrix}
			1&2&3&4\\
			3&4&1&2
		\end{pmatrix}
		\]
		\begin{align*}
			\signe(\sigma)&= \left( \dfrac{\colorbox{col1}{$\cancel{3-4}$}}{\colorbox{col2}{$\cancel{1-2}$}} \right) \left( \dfrac{\colorbox{col3}{$\cancelto{-1}{3-1}$}}{\colorbox{col3}{$\cancel{1-3}$}} \right) \left( \dfrac{\colorbox{col4}{$\cancelto{-1}{3-2}$}}{\colorbox{col5}{$\cancel{1-4}$}} \right) \left( \dfrac{\colorbox{col5}{$\cancelto{-1}{4-1}$}}{\colorbox{col4}{$\cancel{2-3}$}} \right) \left( \dfrac{\colorbox{col6}{$\cancelto{-1}{4-2}$}}{\colorbox{col6}{$\cancel{2-4}$}} \right) \left( \dfrac{\colorbox{col2}{$\cancel{1-2}$}}{\colorbox{col1}{$\cancel{3-4}$}} \right)\\
			&= (-1)^4 = 1
		\end{align*}
	\end{exem}
	\begin{rema}
		Chaque terme $(i-j)$ appara\^it au d\'enominateur et au num\'erateur, \`a signe pr\`es, donc $\signe(\sigma) \in \{-1,1\}$.
	\end{rema}
	\begin{prop}
		Soient $\alpha, \beta \in S_n$, alors $\signe(\alpha \circ \beta) = \signe(\alpha) \cdot \signe(\beta)$.
		\begin{proof}~

			\begin{align*}
				\signe(\alpha \circ \beta)&= \prod_{i < j}\dfrac{\alpha(\beta(i)) - \alpha(\beta(j))}{i-j}\\
				&= \prod_{i < j}\left( \dfrac{\alpha(\beta(i)) - \alpha(\beta(j))}{i-j} \right) \left( \dfrac{\beta(i) - \beta(j)}{\beta(i) - \beta(j)} \right)\\
				&= \prod_{i < j}\dfrac{\alpha(\beta(i)) - \alpha(\beta(j))}{\beta(i) - \beta(j)} \cdot \prod_{i < j}\dfrac{\beta(i) - \beta(j)}{i-j}\\
				&= \left( \prod_{i < j}\dfrac{\alpha(\beta(i)) - \alpha(\beta(j))}{\beta(i) - \beta(j)} \right) \cdot \signe(\beta)\\
				&= \signe(\alpha) \cdot \signe(\beta)
			\end{align*}
		\end{proof}
	\end{prop}
	\begin{defin}
		Un cycle longueur 2 s'appelle une \emph{transposition}.
	\end{defin}
	\begin{rema}
		On peut d\'ecomposer un cycle en un produit de transposition:
		\begin{align*}
			\begin{pmatrix}
				a_1&a_2&\dotsb&a_l
			\end{pmatrix}&= \begin{pmatrix}
				a_1&a_l
			\end{pmatrix} \circ \begin{pmatrix}
				a_1&a_{l-1}
			\end{pmatrix} \circ \dotsb \circ \begin{pmatrix}
				a_1&a_2
			\end{pmatrix}
		\end{align*}
	\end{rema}
	\begin{exem}
		\[
		\sigma = \begin{pmatrix}
			2&3&5&6&8
		\end{pmatrix}
		\]
		\begin{align*}
			\begin{split}
				\begin{pmatrix}
					2&8
				\end{pmatrix} \circ \begin{pmatrix}
					2&6
				\end{pmatrix} \circ \begin{pmatrix}
					2&5
				\end{pmatrix} \circ \begin{pmatrix}
					2&3
				\end{pmatrix}(2)&= 3
			\end{split}\tag{2}
			\\[8pt]
			\begin{split}
				\begin{pmatrix}
					2&8
				\end{pmatrix} \circ \begin{pmatrix}
					2&6
				\end{pmatrix} \circ \begin{pmatrix}
					2&5
				\end{pmatrix} \circ \begin{pmatrix}
					2&3
				\end{pmatrix}(3)&= \begin{pmatrix}
					2&8
				\end{pmatrix} \circ \begin{pmatrix}
					2&6
				\end{pmatrix} \circ \begin{pmatrix}
					2&5
				\end{pmatrix}(2)\\
				&= 5
			\end{split}\tag{3}
			\\[8pt]
			\begin{split}
				\begin{pmatrix}
					2&8
				\end{pmatrix} \circ \begin{pmatrix}
					2&6
				\end{pmatrix} \circ \begin{pmatrix}
					2&5
				\end{pmatrix} \circ \begin{pmatrix}
					2&3
				\end{pmatrix}(5)&= \begin{pmatrix}
					2&8
				\end{pmatrix} \circ \begin{pmatrix}
					2&6
				\end{pmatrix} \circ \begin{pmatrix}
					2&5
				\end{pmatrix}(5)\\
				&= \begin{pmatrix}
					2&8
				\end{pmatrix} \circ \begin{pmatrix}
					2&6
				\end{pmatrix}(2)\\
				&= 6
			\end{split}\tag{5}
			\\[8pt]
			\begin{split}
				\begin{pmatrix}
					2&8
				\end{pmatrix} \circ \begin{pmatrix}
					2&6
				\end{pmatrix} \circ \begin{pmatrix}
					2&5
				\end{pmatrix} \circ \begin{pmatrix}
					2&3
				\end{pmatrix}(6)&= \begin{pmatrix}
					2&8
				\end{pmatrix} \circ \begin{pmatrix}
					2&6
				\end{pmatrix} \circ \begin{pmatrix}
					2&5
				\end{pmatrix}(6)\\
				&= \begin{pmatrix}
					2&8
				\end{pmatrix} \circ \begin{pmatrix}
					2&6
				\end{pmatrix}(6)\\
				&= \begin{pmatrix}
					2&8
				\end{pmatrix}(2)\\
				&= 8
			\end{split}\tag{6}
			\\[8pt]
			\begin{split}
				\begin{pmatrix}
					2&8
				\end{pmatrix} \circ \begin{pmatrix}
					2&6
				\end{pmatrix} \circ \begin{pmatrix}
					2&5
				\end{pmatrix} \circ \begin{pmatrix}
					2&3
				\end{pmatrix}(8)&= \begin{pmatrix}
					2&8
				\end{pmatrix} \circ \begin{pmatrix}
					2&6
				\end{pmatrix} \circ \begin{pmatrix}
					2&5
				\end{pmatrix}(8)\\
				&= \begin{pmatrix}
					2&8
				\end{pmatrix} \circ \begin{pmatrix}
					2&6
				\end{pmatrix}(8)\\
				&= \begin{pmatrix}
					2&8
				\end{pmatrix}(8)\\
				&= 2
			\end{split}\tag{8}
		\end{align*}
		\begin{align*}
		\begin{pmatrix}
			2&8
		\end{pmatrix} \circ \begin{pmatrix}
			2&6
		\end{pmatrix} \circ \begin{pmatrix}
			2&5
		\end{pmatrix} \circ \begin{pmatrix}
			2&3
		\end{pmatrix}&= \begin{pmatrix}
			2&3&5&6&8
		\end{pmatrix}
		\end{align*}

		Alors, $\signe(\sigma) = \signe\left( \begin{pmatrix}
			2&8
		\end{pmatrix} \circ \begin{pmatrix}
			2&6
		\end{pmatrix} \circ \begin{pmatrix}
			2&5
		\end{pmatrix} \circ \begin{pmatrix}
			2&3
		\end{pmatrix} \right) = (-1)^4 = 1$.
	\end{exem}
	\begin{prop}
		Le signe d'une transposition est $-1$.
		\begin{proof}
			par exemple

			$\begin{pmatrix}
				2&4
			\end{pmatrix} \in S_4$.
			\begin{align*}
				\signe(\sigma)&= \left( \dfrac{\colorbox{col3}{$\cancel{1-4}$}}{\colorbox{col1}{$\cancel{1-2}$}} \right) \left( \dfrac{\colorbox{col2}{$\cancel{1-3}$}}{\colorbox{col2}{$\cancel{1-3}$}} \right) \left( \dfrac{\colorbox{col1}{$\cancel{1-2}$}}{\colorbox{col3}{$\cancel{1-4}$}} \right) \left( \dfrac{\colorbox{col6}{$\cancelto{-1}{4-3}$}}{\colorbox{col4}{$\cancel{2-3}$}} \right) \left( \dfrac{\colorbox{col5}{$\cancelto{-1}{4-2}$}}{\colorbox{col5}{$\cancel{2-4}$}} \right) \left( \dfrac{\colorbox{col4}{$\cancelto{-1}{3-2}$}}{\colorbox{col6}{$\cancel{3-4}$}} \right)\\
				&= (-1)^3 = -1
			\end{align*}
			\renewcommand{\qedsymbol}{\#}
		\end{proof}
		\renewcommand{\qedsymbol}{$\square$}
	\end{prop}
	\begin{prop}
		Le signe d'un cycle de longueur $l$ est
		\begin{ulist}
			\item $1$ si $l$ est impair
			\item $-1$ si $l$ est pair
		\end{ulist}

		Plus g\'en\'eralement, le signe d'une permutation $\sigma$ est $(-1)^\gamma$, o\`u $\gamma$ est le nombre de transpositions dans \emph{une} d\'ecomposition de $\sigma$ en transpositions.
	\end{prop}
	\begin{rema}
		Une autre fa\c con de calculer le signe d'une permutation.
		\[
		\sigma = \begin{pmatrix}
			1&2&3&4&5&6&7&8\\
			1&3&5&4&6&8&7&2
		\end{pmatrix}
		\]
		\begin{center}
			\begin{tikzpicture}[inner sep=2pt,yscale=2]
				\node (1) at (0,1) {1};
				\node (2) at (1,1) {2};
				\node (3) at (2,1) {3};
				\node (4) at (3,1) {4};
				\node (5) at (4,1) {5};
				\node (6) at (5,1) {6};
				\node (7) at (6,1) {7};
				\node (8) at (7,1) {8};

				\node (1') at (0,0) {1};
				\node (2') at (1,0) {2};
				\node (3') at (2,0) {3};
				\node (4') at (3,0) {4};
				\node (5') at (4,0) {5};
				\node (6') at (5,0) {6};
				\node (7') at (6,0) {7};
				\node (8') at (7,0) {8};

				\path[-]	(1) edge (1')
							(2) edge (3')
							(3) edge (5')
							(4) edge (4')
							(5) edge (6')
							(6) edge (8')
							(7) edge (7')
							(8) edge (2');

				\fill (1.86,.14) ellipse (1.4pt and .7pt);
				\fill (3,.5) ellipse (1.4pt and .7pt);
				\fill (3.25,.38) ellipse (1.4pt and .7pt);
				\fill (3,.33) ellipse (1.4pt and .7pt);
				\fill (4.43,.57) ellipse (1.4pt and .7pt);
				\fill (6,.5) ellipse (1.4pt and .7pt);
				\fill (5.5,.75) ellipse (1.4pt and .7pt);
				\fill (6,.83) ellipse (1.4pt and .7pt);
			\end{tikzpicture}
		\end{center}

		Connecter chaque \'el\'ement, compter les intersections des segments, $\signe(\sigma) = (-1)^\gamma$, o\`u $\gamma$ est le nombre d'intersections. Ici, $\signe(\sigma) = (-1)^8 = 1$.
	\end{rema}

	\stepcounter{chapter}
	\chapter{Homomorphismes}
	\cours
	\begin{defin}
		Soient $G,H$ deux groupes et $f:G \to H$.

		On dit que $f$ est un \emph{homomorphisme} (ou morphisme de groupes) si $f(ab) = f(a)f(b)$, $\forall a,b \in G$.
	\end{defin}
	\begin{exem}~

		\begin{nlist}
			\item Tous les isomorphismes sont des morphismes.
			\item $f:G \to H$, $a \mapsto e_H$ est toujours un homomorphisme.

			En effet, $f(ab) = e_H = e_H e_H = f(a) f(b)$.

			$f$ n'est pas un isomorphisme, sauf si $G=H=\{e\}$.
			\item $\signe:S_n \to \{-1,1\} = C_2$.

			$\signe(\alpha \circ \beta) = \signe(\alpha) \cdot \signe(\beta)$, donc $\signe$ est un homomorphisme.
			\item $\det:GL(n, \reels) \to \reels_*$ est un homomorphisme, car $\det(AB) = \det(A) \cdot \det(B)$.
			\item Pour un $m \in \entiers$ fix\'e, $f:\entiers \to \entiers$, $f(n) = m \cdot n$ est un homomorphisme.

			En effet, $f(n_1+n_2) = m(n_1+n_2) = mn_1 + mn_2 = f(n_1) + f(n_2)$.
			\item Si $H \lte G$, $i:H \to G$, $i(x)=x$ est un homomorphisme.

			Ce n'est pas $\identite$, car $H \neq G$ en g\'en\'eral.
		\end{nlist}
	\end{exem}
	\begin{prop}~

		Si $f:G \to H$ est un homomorphisme, alors
		\begin{nlist}
			\item $f(e_G) = e_H$
			\item $f(a^{-1}) = f(a)^{-1}$
		\end{nlist}
		\begin{proof}~

			\begin{nlist}
				\item
				\begin{align*}
					f(e_G)&= f(e_G e_G)\\
					&= f(e_G) f(e_G)\\
					f(e_G)^{-1}f(e_G)&= f(e_G)^{-1} f(e_G) f(e_G)\\
					e_H&= e_H f(e_G)\\
					&= f(e_G)
				\end{align*}
				\item $f(a^{-1}) f(a) = f(a^{-1}a) = f(e_G) = e_H$, donc $f(a^{-1}) = f(a)^{-1}$.
			\end{nlist}
		\end{proof}
	\end{prop}
	\begin{prop}~

		La composition de deux homomorphismes est un homomorphisme.
		\begin{proof}~

			Soient $f:G \to H$ et $g:H \to K$ deux homomorphismes, et $a,b \in G$.
			\begin{align*}
				(g \circ f)(ab)&= g(f(ab))\\
				&= g(f(a)f(b))\\
				&= g(f(a))g(f(b))\\
				&= (g \circ f)(a) (g \circ f)(b)
			\end{align*}
		\end{proof}
	\end{prop}
	\begin{prop}
		Soit $f:G \to H$ un homomorphisme. Alors,
		\begin{nlist}
			\item 	$\image(f) = \{f(a) \mid a \in G\}$ est un sous-groupe de $H$;
			\item le \emph{noyau} $\ker(f) = \{a \in G \mid f(a) = e\}$ est un sous-groupe de $G$.
		\end{nlist}
		\begin{exem}~

			\begin{enumerate}[label=(\alph*)]
				\item $\image(\det) = \reels_*$, $\det\begin{pmatrix}
					x&0&\dotsb&0\\
					0&1&\dotsb&0\\
					\vdots&\vdots&\ddots&\vdots\\
					0&\dotsb&0&1
				\end{pmatrix}=x$

				$\ker(\det) = \{M \in GL(n,\reels) \mid \det(M)=1\} = SL(n,\reels)$.
				\item $f:\entiers \to \entiers$, $f(n)=mn$.

				$\image(f) = m\entiers \lte \entiers$

				$\ker(f) = \left\{ \begin{array}{rl}
					\entiers&\text{si } m=0\\
					\{0\}&\text{sinon}
				\end{array} \right.$
			\end{enumerate}
		\end{exem}
		\begin{proof}~

			\begin{nlist}
				\item $f(e_G) \in \image(f)$, donc $\image(f) \neq \emptyset$.

				Si $a,b \in \image(f)$, alors $a=f(c)$ et $b=f(d)$.

				On a alors
				\begin{align*}
					ab^{-1}&= f(c)f(d)^{-1}\\
					&= f(c)f(d^{-1})\\
					&= f(cd^{-1}) \in \image(f)
				\end{align*}
				\item $f(e_G) = e_H \Rightarrow f(e_G) \in \ker(f) \Rightarrow \ker(f) \neq \emptyset$.

				Soient $a,b \in \ker(f)$. Alors, $f(a)=e_H$ et $f(b)=e_H$.
				\begin{align*}
					f(ab^{-1})&= f(a)f(b^{-1})\\
					&= f(a)f(b)^{-1}\\
					&= e_H e_H^{-1}\\
					&= e_H
				\end{align*}
				donc $ab^{-1} \in \ker(f)$.
			\end{nlist}
		\end{proof}
		\begin{exem}~

			$\image(\signe) = C_2 = \{-1,1\}$, si $n \gte 1$.

			$\ker(\signe) = \{\sigma \in S_n \mid \signe(\sigma)=1\}$ est un sous-groupe de $S_n$.
			\begin{nota}
				On note ce sous-groupe $A_n$ et on l'appelle le \emph{groupe altern\'e}.
			\end{nota}
			\begin{exem}
				$n=3$.

				$A_3 = \{e,(123),(132)\}$.
			\end{exem}
		\end{exem}
	\end{prop}
	\begin{prop}~

		Soit $f:G \to H$ un morphisme.

		$\ker(f) = \{e\}$ si, et seulement si, $f$ est injective.
		\begin{proof}~

			\begin{itemize}
				\item[$(\Leftarrow)$] Supposons que $f$ est injective.

				On sait que $e_G \in \ker(f)$.

				Supposons que $\exists a \in \ker(f)$. On a
				\[
				\begin{array}{rclll}
					f(a)&=&e_H&\quad&\text{\footnotesize par d\'efinition de }\ker\\
					&=&f(e_G)\\
					a&=&e_G&&\text{\footnotesize car $f$ est injective}
				\end{array}
				\]

				Donc, $\ker(f) = \{e_G\}$.
				\item[$(\Rightarrow)$] Supposons que $\ker(f) = \{e_G\}$.

				Soient $a,b \in G$ t.q. $f(a) = f(b)$.
				\begin{align*}
					f(a)&= f(b)\\
					f(a)f(b)^{-1}&= e_H\\
					f(a)f(b^{-1})&= e_H\\
					f(ab^{-1})&= e_H\\
					ab^{-1}&= e_G\\
					a&= b
				\end{align*}
			\end{itemize}
		\end{proof}
		\begin{rema}
			$\image(f)=H$ si, et seulement si, $f$ est surjective.
		\end{rema}
	\end{prop}

	\cours
	\begin{rappel}~

		\begin{ulist}
			\item homomorphisme: $f:G \to H$ t.q. $f(ab)=f(a)f(b)$
			\item Si $f$ est un homomorphisme, alors
			\begin{ulist}
				\item $f(e_G)=e_H$
				\item $f(a^{-1})=f(a)^{-1}$
				\item $f(a^n)=f(a)^n$
			\end{ulist}
			\item $\image(f) = \{f(a) \mid a \in G\} \lte H$
			\item $\ker(f) = \{a \in G \mid f(a)=e_H\} \lte G$
			\item $\ker(f)=\{e_G\} \Leftrightarrow f$ est injective
		\end{ulist}
	\end{rappel}
	\begin{exem}~

		$\begin{array}{rcl}
			f:\entiers&\to&\entiers_m\\
			n&\mapsto&\overline{n}
		\end{array}$ est un homomorphisme.

		En effet,
		\begin{align*}
			f(n_1+n_2)&= \overline{n_1+n_2}\\
			&= \overline{n_1} + \overline{n_2}\\
			&= f(n_1) + f(n_2)
		\end{align*}

		$\image(f) = \entiers_m$.

		$\ker(f) = m\entiers$.
	\end{exem}

	\phantomsection\section*{\'Equivalence modulo $H$ et th\'eor\`eme de Lagrange}
	\addcontentsline{toc}{section}{\'Equivalence modulo $H$ et th\'eor\`eme de Lagrange}
	$G$ un groupe quelconque et $H \lte G$ un sous-groupe.

	On d\'efinit une relation sur $G$ par $a \sim b$ si, et seulement si, $ab^{-1} \in H$.

	Cette relation est appel\'ee \emph{\'equivalence modulo $H$}.
	\begin{prop}~

		$\sim$ est une relation d'\'equivalence.
		\begin{proof}~

			\begin{itemize}
				\item[(refl)] Soit $a \in G$.

				$aa^{-1} = e_G \in H$, puisque $H \lte G$.

				Alors, $a \sim a$.
				\item[(sym)] Soient $a,b \in G$.

				Supposons que $a \sim b$.

				Alors, $ab^{-1} \in H$.

				Comme $H$ est un groupe, $H$ est ferm\'e pour la prise d'inverses, donc $(ab^{-1})^{-1} = ba^{-1} \in H$.

				Ainsi, $b \sim a$.
				\item[(trans)] Soient $a,b,c \in G$.

				Supposons que $a \sim b$ et $b \sim c$.

				Alors, $ab^{-1} \in H$ et $bc^{-1} \in H$.

				Comme $H$ est un groupe, $H$ est ferm\'e pour son op\'eration, donc $\left( ab^{-1} \right) \left( bc^{-1} \right) = a \left( bb^{-1} \right) c^{-1} = ac^{-1} \in H$.

				Ainsi, $a \sim c$.
			\end{itemize}
		\end{proof}
	\end{prop}

	Le groupe $G$ se partitionne en classes d'\'equivalence modulo $H$: $G = \overline{a_1} \cup \overline{a_2} \cup \dots$
	\begin{exem}~

		$G=\entiers_8$, $H = \{\overline0, \overline4\}$.

		La classe de $\overline0 \in G \mod H$ est l'ensemble des $\overline{n} \in \entiers_8$ t.q. $\overline0 - \overline{n} \in H$.

		Alors, $-\overline{n} \in \{\overline0, \overline4\}$, donc $\overline{n} = \overline0$ ou $\overline{n} = \overline4$.

		$C(\overline0) = \{\overline0,\overline4\}$, $C(\overline1) = \{\overline1,\overline5\}$, $C(\overline2) = \{\overline2,\overline6\}$, $C(\overline3) = \{\overline3,\overline7\}$.
	\end{exem}
	\begin{lem}~

		La classe modulo $H$ de  $a \in G$ est $\{ha \mid h \in H\}$.
		\begin{nota}~

			$\{ha \mid h \in H\}$ est not\'e $Ha$.
		\end{nota}
		\begin{proof}~

			Soient $a,b \in G$.
			\begin{itemize}
				\item[$(\subseteq)$] Supposons que $b \sim a \mod H$, c'est-\`a-dire $b \in \overline{a}$.

				Alors, $ba^{-1} \in H$, disons $ba^{-1} = h$ avec $h \in H$, donc $b=ha$.

				$\therefore \overline{a} \subseteq Ha$.
				\item[$(\supseteq)$] Supposons que $b \in Ha$.

				Alors, $b=ha$ avec $h \in H$, donc $ba^{-1} = h \in H$. Ainsi, $b \sim a$, donc $b \in \overline{a}$.

				$\therefore Ha \subseteq \overline{a}$.
			\end{itemize}

			Ainsi, $Ha = \overline{a}$.
		\end{proof}
	\end{lem}
	\begin{coro}~

		Toutes les classes d'\'equivalence modulo $H$ ont le m\^eme nombre d'\'el\'ements.

		Plus pr\'ecis\'ement, $\card{Ha} = \card{H}$.
		\begin{proof}~

			$\begin{array}{rcl}
				f:H&\to&Ha\\
				h&\mapsto&ha
			\end{array}$ est bijective, car elle est inversible d'inverse
			$\begin{array}{rcl}
				f^{-1}:Ha&\to&H\\
				b&\mapsto&ba^{-1}
			\end{array}$.
		\end{proof}
	\end{coro}

	\cours
	\begin{rappel}
		Pour d\'eterminer $\ker(f)$ et $\image(f)$ d'un homomorphisme $f$.
		\begin{exem}
			$\begin{array}{rcl}
				f:\entiers_{12}&\to&\entiers_6\\
				\overline{n}&\mapsto&\overline{n}
			\end{array}$.

			$\ker(f) = \{a \in G \mid f(a)=e\}$.

			Supposons que $\overline{n} \in \ker(f)$. Alors,
			\begin{align*}
				f(\overline{n})&= \overline0\\
				\overline{n}&= \overline0\\
				n&= 6k
			\end{align*}
			avec $k \in \entiers$. Donc, $\ker(f) \subseteq 6\entiers_{12}$.

			Pour avoir $\ker(f) = 6\entiers_{12}$, il faut aussi montrer $6\entiers_{12} \subseteq \ker(f)$.

			Supposons que $n = 6k \Rightarrow \overline{n} = \overline{6k} \Rightarrow f(\overline{n}) = f(\overline{6k}) = \overline{6k} = \overline0$, donc $\overline{n} \in \ker(f)$.
		\end{exem}
	\end{rappel}
	\begin{rappel}~

		$G$ groupe, $H \lte G$ sous-groupe.
		\begin{ulist}
			\item \'Equivalence$\mod H$:

			$a \sim b \Leftrightarrow ab^{-1} \in H$;
			\item Les  classes d'\'equivalence$\mod H$ sont $\overline{a} = Ha = \{ha \mid h \in H\}$;
			\item Toutes les classes ont la m\^eme taille, $\card{Ha} = \card{H}$.
		\end{ulist}
	\end{rappel}
	\begin{exem}~

		$G = S_3 = \{e, \begin{pmatrix}
			1&2
		\end{pmatrix}, \begin{pmatrix}
			1&3
		\end{pmatrix}, \begin{pmatrix}
			2&3
		\end{pmatrix}, \begin{pmatrix}
			1&2&3
		\end{pmatrix}, \begin{pmatrix}
			1&3&2
		\end{pmatrix}\}, H = \{e, \begin{pmatrix}
			1&2
		\end{pmatrix}\}$.

		Calculons les classes$\mod H$.
		\begin{ulist}
			\item $He = \{e \circ e,\begin{pmatrix}
				1&2
			\end{pmatrix} \circ e\} = H$;
			\item $H\begin{pmatrix}
				1&3
			\end{pmatrix} = \{e \circ \begin{pmatrix}
				1&3
			\end{pmatrix}, \begin{pmatrix}
				1&2
			\end{pmatrix} \circ \begin{pmatrix}
				1&3
			\end{pmatrix}\} = \{\begin{pmatrix}
				1&3
			\end{pmatrix}, \begin{pmatrix}
				1&3&2
			\end{pmatrix}\}$;
			\item $H\begin{pmatrix}
				2&3
			\end{pmatrix} = \{e \circ \begin{pmatrix}
				2&3
			\end{pmatrix}, \begin{pmatrix}
				1&2
			\end{pmatrix} \circ \begin{pmatrix}
				2&3
			\end{pmatrix}\} = \{\begin{pmatrix}
				2&3
			\end{pmatrix}, \begin{pmatrix}
				1&2&3
			\end{pmatrix}\}$.
		\end{ulist}
	\end{exem}
	\begin{thm}[Lagrange]~

		Soit $G$ un groupe fini.

		Si $H \lte G$, alors $\card{H} \big| \card{G}$.
		\begin{proof}~

			Les classes modulo $H$ partitionnent $G$ est sous-ensembles (classes) de taille $\card{H}$ chacun, donc
			\begin{align*}
				G&= Ha_1 \cup Ha_2 \cup \dotsb \cup Ha_n\\
				\card{G}&= \card{H} + \card{H} + \dotsb + \card{H}\\
				&= n\card{H}
			\end{align*}
		\end{proof}
	\end{thm}
	\begin{coro}
		$a \in G$, $o(a) \big| \card{G}$.
		\begin{proof}~

			$\langle a \rangle = \{a^n \mid n \in \entiers\}$, le sous-groupe engendr\'e par $a$, est un sous-groupe car
			\begin{ulist}
				\item $a^0 = e \in \langle a \rangle$, donc $\langle a \rangle \neq \emptyset$;
				\item Si $a^n, a^m \in \langle a \rangle$, alors $a^n(a^m)^{-1} = a^{n-m} \in \langle a \rangle$, donc $\langle a \rangle \lte G$.
			\end{ulist}

			Par le th\'eor\`eme de Lagrange, $\card{\langle a \rangle} \big| \card{G}$.

			De plus, $\card{\langle a \rangle} = o(a)$, qu'on d\'emontrera plus tard, donc $o(a) \big| \card{G}$.
		\end{proof}
	\end{coro}
	\begin{coro}
		Si $\card{G} = m$ et $a \in G$, alors $a^m = e$.
		\begin{proof}~

			On sait que $a^{o(a)} = e$.

			Par le corollaire pr\'ec\'edent, $m = ko(a)$, car $o(a) \big| m$, donc $a^m = a^{ko(a)} = \left( a^{o(a)} \right)^k = e^k = e$.
		\end{proof}
	\end{coro}
	\begin{coro}[petit th\'eor\`eme de Fermat]
		Si $a \in \entiers$ et $p$ est un nombre premier t.q. $p \nmid a$, alors $a^{p-1} \equiv 1 \pmod{p}$.
		\begin{lem}
			${\entiers_p}_*$ est un groupe pour la multiplication.
		\end{lem}
		\begin{proof}~

			${\entiers_p}_*$ est un groupe avec $\card{{\entiers_p}_*} = p-1$.

			Par le corollaire pr\'ec\'edent, si $a \in {\entiers_p}_*$, alors $\overline{a}^{p-1}=\overline1$, donc $a^{p-1} \equiv \pmod{p}$.
		\end{proof}
	\end{coro}
	\begin{nota}
		$\card{G}$ s'appelle aussi l'\emph{ordre} de $G$.
	\end{nota}
	\begin{defin}
		$H \lte G$.

		On appelle \emph{indice} de $H$ dans $G$ le nombre de classes modulo $H$ et on le note $[G:H]$.

		Le th\'eor\`eme de Lagrange implique que si $G$ est fini, $[G:H] = \frac{\card{G}}{\card{H}}$.

		L'indice peut \^etre fini m\^eme si $G$ et $H$ sont infinis.
		\begin{exem}
			$[\entiers:m\entiers]=m$.
		\end{exem}
	\end{defin}
	\begin{exem}
		Les seuls sous-groupes de $\entiers_p$ avec $p$ premier sont $\{\overline0\}$ et $\entiers_p$, car si $H \lte \entiers_p$. Par Lagrange, $\card{H} \in \{1,p\}$.
	\end{exem}
	\begin{exem}
		$H \lte \entiers_8$ qui contient au moins 5 \'el\'ements distincts, alors $H = \entiers_8$, car les diviseurs de 8 sont $\{1,2,4,8\}$.
	\end{exem}

	\setcounter{section}{5}
	\section{Groupes quotients}
	On veut d\'efinir une op\'eration sur les classes d'\'equivalence$\mod H$ pour en faire un groupe.

	\subsection*{Tentative}
	On voudrait d\'efinir $Ha*Hb=H(a*b)$. Est-ce d\'efini sans ambigu\"it\'e?

	Supposons que $Ha=Ha'$ et $Hb=Hb'$, c'est-\`a-dire $a(a')^{-1} \in H$ et $b(b')^{-1} \in H$. Alors, on voudrait montrer que $H(ab) = H(a'b') \Leftrightarrow (ab)(a'b') \in H \Leftrightarrow ab(b')^{-1}(a')^{-1} \in H$. Cependant, ce n'est pas vrai en g\'en\'eral.

	\begin{defin}~

		Un sous-groupe $H \lte G$ est \emph{distingu\'e}, ou \emph{normal}, si $\forall a \in G$, $Ha=aH$, c'est-\`a-dire, $\{ha \mid h \in H\} = \{ah \mid h \in H\}$.
		\begin{rema}~

			$Ha=aH$ \emph{ne veut pas dire} $ha=ah$ pour tout $h \in H$.

			Plut\^ot, $Ha=aH$ signifie $ha=ah'$ pour un certain $h' \in H$.
		\end{rema}
	\end{defin}
	\begin{exem}~

		Pour $G = S_3 = \{e, \begin{pmatrix}
			1&2
		\end{pmatrix}, \begin{pmatrix}
			1&3
		\end{pmatrix}, \begin{pmatrix}
			2&3
		\end{pmatrix}, \begin{pmatrix}
			1&2&3
		\end{pmatrix}, \begin{pmatrix}
			1&3&2
		\end{pmatrix}\}$.

		\begin{ulist}
			\item Le sous-groupe $H=\{e, \begin{pmatrix}
				1&2
			\end{pmatrix}\}$ n'est pas distingu\'e. En effet, regardons la classe de $\begin{pmatrix}
				1&3
			\end{pmatrix}$.
			\begin{align*}
				H\begin{pmatrix}
					1&3
				\end{pmatrix}&= \{\begin{pmatrix}
					1&3
				\end{pmatrix}, \begin{pmatrix}
					1&3&2
				\end{pmatrix}\}\\
				\begin{pmatrix}
					1&3
				\end{pmatrix}H&= \{\begin{pmatrix}
					1&3
				\end{pmatrix} \circ e, \begin{pmatrix}
					1&3
				\end{pmatrix} \circ \begin{pmatrix}
					1&2
				\end{pmatrix}\} = \{\begin{pmatrix}
					1&3
				\end{pmatrix}, \begin{pmatrix}
					1&2&3
				\end{pmatrix}\}
			\end{align*}
			\item Le sous-groupe $N=\{e, \begin{pmatrix}
				1&2&3
			\end{pmatrix}, \begin{pmatrix}
				1&3&2
			\end{pmatrix}\}$ est distingu\'e. En effet
			\begin{align*}
				\begin{split}
					Ne&= \{e \circ e, \begin{pmatrix}
						1&2&3
					\end{pmatrix} \circ e, \begin{pmatrix}
						1&3&2
					\end{pmatrix} \circ e\} = \{e, \begin{pmatrix}
						1&2&3
					\end{pmatrix}, \begin{pmatrix}
						1&3&2
					\end{pmatrix}\} = N
				\end{split}
				\\
				\begin{split}
					eN&= \{e \circ e, e \circ \begin{pmatrix}
						1&2&3
					\end{pmatrix}, e \circ \begin{pmatrix}
						1&3&2
					\end{pmatrix}\} = \{e, \begin{pmatrix}
						1&2&3
					\end{pmatrix}, \begin{pmatrix}
						1&3&2
					\end{pmatrix}\} = N
				\end{split}
				\\[1em]
				\begin{split}
					N\begin{pmatrix}
						1&2
					\end{pmatrix}&= \{e \circ \begin{pmatrix}
						1&2
					\end{pmatrix}, \begin{pmatrix}
						1&2&3
					\end{pmatrix} \circ \begin{pmatrix}
						1&2
					\end{pmatrix}, \begin{pmatrix}
						1&3&2
					\end{pmatrix} \circ \begin{pmatrix}
						1&2
					\end{pmatrix}\} = \{\begin{pmatrix}
						1&2
					\end{pmatrix}, \begin{pmatrix}
						1&3
					\end{pmatrix}, \begin{pmatrix}
						2&3
					\end{pmatrix}\}
				\end{split}
				\\
				\begin{split}
					\begin{pmatrix}
						1&2
					\end{pmatrix}N&= \{\begin{pmatrix}
						1&2
					\end{pmatrix} \circ e, \begin{pmatrix}
						1&2
					\end{pmatrix} \circ \begin{pmatrix}
						1&2&3
					\end{pmatrix}, \begin{pmatrix}
						1&2
					\end{pmatrix} \circ \begin{pmatrix}
						1&3&2
					\end{pmatrix}\} = \{\begin{pmatrix}
						1&2
					\end{pmatrix}, \begin{pmatrix}
						2&3
					\end{pmatrix}, \begin{pmatrix}
						1&3
					\end{pmatrix}\}
				\end{split}
			\end{align*}
		\end{ulist}
	\end{exem}
	\begin{prop}
		Si $f:G \to H$ est un homomorphisme, $\ker(f)$ est un sous-groupe distingu\'e de $G$.
		\begin{proof}
			$\ker(f) = \{a \in G \mid f(a)=e\}$.

			Soit $b \in G$. On veut montrer que $b\ker(f)=\ker(f)b$.
			\begin{itemize}
				\item[$(\subseteq)$] Soit $ba \in b\ker(f)$.
				\begin{align*}
					ba&= bab^{-1}b
				\end{align*}
				posons $a'=bab^{-1}$
				\begin{align*}
					f(a')&= f(bab^{-1})\\
					&= f(b)f(a)f(b^{-1})\\
					&= f(b)ef(b^{-1})\\
					&= e
				\end{align*}
				donc $a' \in \ker(f) \Rightarrow ba=a'b \in \ker(f)b \Rightarrow b\ker(f) \subseteq \ker(f)b$.
				\item[$(\supseteq)$] Soit $ab \in \ker(f)b$.
				\begin{align*}
					ab&= bb^{-1}ab
				\end{align*}
				posons $a'=b^{-1}ab$
				\begin{align*}
					f(a')&= f(b^{-1}ab)\\
					&= f(b^{-1})f(a)f(b)\\
					&= f(b^{-1})ef(b)\\
					&= e
				\end{align*}
				donc $a' \in \ker(f) \Rightarrow ab=ba' \in b\ker(f) \Rightarrow \ker(f)b \subseteq b\ker(f)$.
			\end{itemize}
		\end{proof}
	\end{prop}

	\cours
	\begin{rappel}~

		\begin{ulist}
			\item Classes d'\'equivalence \`a droite modulo $H$ de $a \in G$: $Ha$
			\item Th\'eor\`eme de Lagrange: $\card{H} \bigm| \card{G}$
			\item Indice de $H$ dans $G$: $\left[G:H\right]$ repr\'esente le nombre de classes
			\item Corollaires de Lagrange:
			\begin{ulist}
				\item $o(a) \bigm| \card{G}$
				\item $a^{\card{G}}=e$
				\item petit th\'eor\`eme de Fermat: si $p \nmid a$, $a^{p-1} \equiv 1 \pmod{p}$, avec $p$ un nombre premier
			\end{ulist}
			\item Sous-groupe \emph{normal} ou \emph{distingu\'e}: $Ha=aH$
			\item $\ker(f)$ est toujours distingu\'e
		\end{ulist}
	\end{rappel}
	\begin{prop}~

		$H \lte G$ est distingu\'e si, et seulement si, pour tous $h \in H$ et $g \in G$, $ghg^{-1} \in H$.
		\begin{proof}~

			\begin{itemize}
				\item[$(\Rightarrow)$] Supposons que $H$ est distingu\'e.

				Soient $h \in H$ et $g \in G$. On a $gh \in gH$. Mais alors, $gh \in Hg$, donc $gh = h'g$ avec $h' \in H$.
				\begin{align*}
					ghg^{-1}&= h'gg^{-1}\\
					&= h' \in H
				\end{align*}
				\item[$(\Leftarrow)$] Supposons que $ghg^{-1} \in H$, $\forall g \in G$, $\forall h \in H$.

				On veut montrer que $gH=Hg$, $\forall g \in G$.
				\begin{itemize}
					\item[$(\subseteq)$] Soit $gh \in gH$. On a
					\begin{align*}
						gh&= ghg^{-1}g\\
						&= h'g \in Hg
					\end{align*}
					\item[$(\supseteq)$] Soit $hg \in Hg$. Alors
					\begin{align*}
						hg&= gg^{-1}hg\\
						&= gh' \in gH
					\end{align*}
				\end{itemize}
			\end{itemize}
		\end{proof}
	\end{prop}
	\subsection*{Nouvelle tentative}
	Soit $G$ un groupe. Soit $N$ un sous-groupe normal de $G$.

	On a $\sfrac{G}{N}$ l'ensemble des classes$\mod N$:
	\begin{equation*}
		\sfrac{G}{N} = \{Na \mid a \in G\}
	\end{equation*}
	\begin{prop}~

		$\sfrac{G}{N}$ est un groupe par l'op\'eration $(Na)(Nb) = N(ab)$.
		\begin{proof}~

			Montrons que l'op\'eration est d\'efinie sans ambigu\"it\'e.

			Supposons que $Na=Na'$ et $Nb=Nb'$. Alors, $a(a')^{-1} \in N$ et $b(b')^{-1} \in N$.
			\begin{align*}
				ab(a'b')^{-1}&= ab(b')^{-1}(a')^{-1}\\
				&= ab(b')^{-1}(a')^{-1}aa^{-1}
			\end{align*}
			On sait que $b(b')^{-1} \in N$.

			De plus, $(a')^{-1}a \in N$, car $a(a')^{-1} \in N$ et $N$ est normal, donc $a^{-1}\left(a(a')^{-1}\right)a \in N$.

			Comme $N$ est ferm\'e pour la multiplication, $b(b')^{-1}(a')^{-1}a \in N$ que nous noterons $n$.

			On a donc
			\begin{equation*}
				ab(a'b')^{-1} = ana^{-1} \in N
			\end{equation*}

			Ainsi, $N(ab) = N(a'b')$, donc l'op\'eration est d\'efinie sans ambigu\"it\'e.

			Maintenant, montrons que $\sfrac{G}{N}$ est un groupe.
			\begin{itemize}
				\item[(A)] Soient $Na, Nb, Nc \in \sfrac{G}{N}$.
				\begin{align*}
					(NaNb)Nc&= N(ab)Nc\\
					&= N((ab)c)\\
					&= N(a(bc))\\
					&= NaN(bc)\\
					&= Na(NbNc)
				\end{align*}
				\item[(N)] On v\'erifie que $Ne=N$ est neutre.
				\begin{align*}
					\begin{split}
						NeNa&= N(ea)\\
						&= Na
					\end{split}
					&
					\begin{split}
						NaNe&= N(ae)\\
						&= Na
					\end{split}
				\end{align*}
				\item[(I)] L'inverse de $Na$ est $Na^{-1}$
				\begin{align*}
					\begin{split}
						NaNa^{-1}&= N(aa^{-1})\\
						&= N
					\end{split}
					&
					\begin{split}
						Na^{-1}Na&= N(a^{-1}a)\\
						&= N
					\end{split}
				\end{align*}
			\end{itemize}
		\end{proof}
	\end{prop}
	\begin{nota}
		$N \trianglelefteq G$ repr\'esente $N$ est un sous-groupe normal de $G$.
	\end{nota}
	\begin{prop}~

		$\begin{array}{rcl}
			p:G&\to&\sfrac{G}{N}\\
			a&\mapsto&Na
		\end{array}$ est un homomorphisme.
		\begin{proof}~

			\begin{align*}
				p(ab)&= N(ab)\\
				&= NaNb\\
				&= p(a)p(b)
			\end{align*}
		\end{proof}
	\end{prop}

	\cours
	\begin{rappel}~

		\begin{ulist}
			\item $N \trianglelefteq G$ sous-groupe normal (ou distingu\'e)
			\begin{ulist}
				\item $\forall g \in G$, $gN=Ng$
				\item $\forall g \in G, \forall n \in N$, $gng^{-1} \in N$
			\end{ulist}
			\item $f:G \to H$ homomorphisme $\Rightarrow \ker(f) \trianglelefteq G$
			\item Groupe quotient: $\sfrac{G}{N} = \{Ng \mid g \in G\}$
			\begin{ulist}
				\item op\'eration: $NaNb=Nab$
				\item neutre: $N$
				\item inverse: $(Na)^{-1} = Na^{-1}$
			\end{ulist}
			\item $p:G \to \sfrac{G}{N}$ est un homomorphisme avec $p(a)=Na$.
		\end{ulist}
	\end{rappel}
	\begin{nota}
		Effectuer $ghg^{-1}$ s'appelle la \emph{conjugaison} de $h$ par $g$.
	\end{nota}
	\begin{rema}
		$H \lte G$ et $g \in G$, alors $gHg^{-1} \lte G$.

		Dans le cas d'un sous-groupe normal, le sous-groupe obtenu par la conjugaison reste le m\^eme sous-groupe.
	\end{rema}
	\begin{exem}~

		\begin{ulist}
			\item Dans $\mathbb{D}_3$\footnote{Pour la d\'efinition de $\mathbb{D}_3$, voir p.\pageref{D3}}
			\begin{ulist}
				\item $\{\varepsilon, \alpha\}$
				\item $\{\varepsilon, \beta\}$
				\item $\{\varepsilon, \gamma\}$
			\end{ulist}
			ne sont pas normaux.
			\begin{ulist}
				\item $\{\varepsilon, \rho, \sigma\}$
			\end{ulist}
			est normal.
			\item Soient $G = \mathbb{D}_3$ et $N=\{\varepsilon, \rho, \sigma\}$.

			Les classes \`a droites sont
			\begin{ulist}
				\item $N = N\varepsilon = \{\varepsilon, \rho, \sigma\}$
				\item $N\alpha = N\beta = N\gamma = \{\alpha, \beta, \gamma\}$
			\end{ulist}

			Alors, $\sfrac{G}{N} = \{N, N\alpha\}$.
			\renewcommand{\arraystretch}{1.2}
			\[
			\begin{array}{c||c|c}
				\circ&N&N\alpha\\
				\hline\hline
				N&N&N\alpha\\
				\hline
				N\alpha&N\alpha&N
			\end{array}
			\]
			\renewcommand{\arraystretch}{1}
			\item Soient $G=\entiers_{10}$ et $N=\{\overline0,\overline5\}$

			Les classes d'\'euivalence sont
			\begin{ulist}
				\item $N = N+\overline0 = \{\overline0,\overline5\}$
				\item $N+\overline1 = \{\overline1,\overline6\}$
				\item $N+\overline2 = \{\overline2,\overline7\}$
				\item $N+\overline3 = \{\overline3,\overline8\}$
				\item $N+\overline4 = \{\overline4,\overline9\}$
			\end{ulist}
			$\sfrac{G}{N} = \{N, N+\overline1, N+\overline2, N+\overline3, N+\overline4\}$.
			\begin{center}
				\begin{tikzpicture}[every node/.style={shape=circle, very thick, inner sep=2pt}]
					\node () at (0,0) {$\entiers_{10}:$};
					\begin{scope}[xshift=22mm]
						\node[draw=col1,fill=col1!30] (0) at (90:12mm) {$\overline0$};
						\node[draw=col2,fill=col2!30] (1) at (90-36:12mm) {$\overline1$};
						\node[draw=col3,fill=col3!30] (2) at (90-2*36:12mm) {$\overline2$};
						\node[draw=col4,fill=col4!30] (3) at (90-3*36:12mm) {$\overline3$};
						\node[draw=col5,fill=col5!30] (4) at (90-4*36:12mm) {$\overline4$};
						\node[draw=col1,fill=col1!30] (5) at (90-5*36:12mm) {$\overline5$};
						\node[draw=col2,fill=col2!30] (6) at (90-6*36:12mm) {$\overline6$};
						\node[draw=col3,fill=col3!30] (7) at (90-7*36:12mm) {$\overline7$};
						\node[draw=col4,fill=col4!30] (8) at (90-8*36:12mm) {$\overline8$};
						\node[draw=col5,fill=col5!30] (9) at (90-9*36:12mm) {$\overline9$};
					\end{scope}
					\draw (0)--(5) (1)--(6) (2)--(7) (3)--(8) (4)--(9);
				\end{tikzpicture}
			\end{center}
			$\sfrac{\entiers_{10}}{5\entiers_{10}} =$ \{\colorbox{col1}{\textcolor{col1}{\textperiodcentered}}, \colorbox{col2}{\textcolor{col2}{\textperiodcentered}}, \colorbox{col3}{\textcolor{col3}{\textperiodcentered}}, \colorbox{col4}{\textcolor{col4}{\textperiodcentered}}, \colorbox{col5}{\textcolor{col5}{\textperiodcentered}}\}.
			\item Soient $G = \reels^2 = \{(x,y) \mid x,y \in \reels\}$ et $N = \{(a,a) \in \reels^2 \mid a \in \reels\}$.

			$N \trianglelefteq G$.

			$\sfrac{G}{N}$ est infini et l'op\'eration sur $\sfrac{G}{N}$ est: $(N+(x,y))+(N+(x',y')) = N+(x+x',y+y')$.
		\end{ulist}
	\end{exem}

	\subsection*{Motivation}
	\begin{center}
		\begin{tikzpicture}
			\draw (0,0) ellipse (1cm and 2cm);
			\node (A) at (1,1.75) {$A$};
			\node (1) at (0,1.5) {$1$};
			\node (2) at (0,.9) {$2$};
			\node (3) at (0,.3) {$3$};
			\node (4) at (0,-.3) {$4$};
			\node (5) at (0,-.9) {$5$};
			\node (6) at (0,-1.5) {$6$};

			\node[shape=coordinate] (12) at (2,1.2) {};
			\node[shape=coordinate] (345) at (2,-.3) {};

			\begin{scope}[xshift=5cm]
				\draw (0,0) ellipse (1cm and 2cm);
				\node (B) at (-1,1.75) {$B$};
				\node (e) at (0,1.5) {$*$};
				\node (t) at (0,.9) {$\triangle$};
				\node (r) at (0,.3) {$\circ$};
				\node (c) at (0,-.3) {$\square$};
				\node (h) at (0,-.9) {$\#$};
				\node (x) at (0,-1.5) {$\times$};
			\end{scope}

			\path[arrows=-{Stealth[length=3mm]}] (A) edge[bend left] node[above] {$f$} (B);

			\draw (1) -- (12) (2) -- (12);
			\draw (3) -- (345) (4) -- (345) (5) -- (345);

			\path[arrows=-{Stealth[length=3mm]}] (12) edge (t) (345) edge (c) (6) edge (x);
		\end{tikzpicture}
	\end{center}
	$f$ n'est ni injective ni surjective.
	\begin{center}
		\begin{tikzpicture}
			\draw (0,0) ellipse (1cm and 2cm);
			\node (A) at (1,1.75) {$A$};
			\node (1) at (0,1.5) {$1$};
			\node (2) at (0,.9) {$2$};
			\node (3) at (0,.3) {$3$};
			\node (4) at (0,-.3) {$4$};
			\node (5) at (0,-.9) {$5$};
			\node (6) at (0,-1.5) {$6$};

			\node[shape=coordinate] (12) at (2,1.2) {};
			\node[shape=coordinate] (345) at (2,-.3) {};

			\begin{scope}[xshift=5cm]
				\draw (0,0) ellipse (1cm and 2cm);
				\node (B) at (-1,1.75) {$B'$};
				\node (t) at (0,.9) {$\triangle$};
				\node (c) at (0,0) {$\square$};
				\node (x) at (0,-.9) {$\times$};
			\end{scope}

			\path[arrows=-{Stealth[length=3mm]}] (A) edge[bend left] node[above] {$f'$} (B);

			\draw (1) -- (12) (2) -- (12);
			\draw (3) -- (345) (4) -- (345) (5) -- (345);

			\path[arrows=-{Stealth[length=3mm]}] (12) edge (t) (345) edge (c) (6) edge (x);
		\end{tikzpicture}
	\end{center}
	$f'$ est surjective, mais elle n'est toujours pas injective.
	\begin{center}
		\begin{tikzpicture}
			\draw (0,0) ellipse (1cm and 2cm);
			\node (A) at (1,1.75) {$A'$};
			\node[draw,shape=ellipse] (12) at (0,.9) {$1~2$};
			\node[draw,shape=ellipse] (345) at (0,0) {$3~4~5$};
			\node[draw,shape=ellipse] (6) at (0,-.9) {$6$};

			\begin{scope}[xshift=5cm]
				\draw (0,0) ellipse (1cm and 2cm);
				\node (B) at (-1,1.75) {$B'$};
				\node (t) at (0,.9) {$\triangle$};
				\node (c) at (0,0) {$\square$};
				\node (x) at (0,-.9) {$\times$};
			\end{scope}

			\path[arrows=-{Stealth[length=3mm]}] (A) edge[bend left] node[above] {$f''$} (B);

			\path[arrows=-{Stealth[length=3mm]}] (12) edge (t) (345) edge (c) (6) edge (x);
		\end{tikzpicture}
	\end{center}
	$f''$ est injective et surjective, donc bijective.
	\begin{center}
		\begin{tikzpicture}
			\node (A) at (-1,1) {$A$};
			\node (A') at (-1,-1) {$A'$};
			\node (B') at (1,-1) {$B'$};
			\node (B) at (1,1) {$B$};
			\path[arrows=-{Stealth[length=2mm]}] (A) edge node[above] {$f$} (B) edge node[left] {$p$} (A') (A') edge node[below] {$f''$} (B') (B') edge node[right] {$i$} (B);
		\end{tikzpicture}
	\end{center}
	o\`u $p$ repr\'esente une projection et $i$ une inclusion. Donc, $f=i \circ f'' \circ p$, o\`u $i$ est injective, $p$ est surjective et $f''$ est bijective.
	\begin{thm}[d'isomorphisme de Jordan]~

		Soit $f:G \to H$ un homomorphisme de groupes. Posons $N = \ker(f) \trianglelefteq G$.
		\begin{center}
			\begin{tikzpicture}
				\node (G) at (-1,1) {$G$};
				\node (H) at (1,1) {$H$};
				\node (GN) at (-1,-1) {$\sfrac{G}{N}$};
				\node (Im) at (1,-1) {$\image(f)$};
				\path[arrows=-{Stealth[length=2mm]}] (G) edge node[above] {$f$} (H) edge node[left] {$p$} (GN) (GN) edge[dashed] node[below] {$\bar{f}$} (Im) (Im) edge node[right] {$i$} (H);
			\end{tikzpicture}
		\end{center}

		$\left(\begin{array}{rcl}
			P:G&\to&\sfrac{g}{N}\\
			a&\mapsto&Na
		\end{array} ~\middle|~ \begin{array}{rcl}
			i:\image(f)&\to&H\\
			h&\mapsto&h
		\end{array}\right)$.

		Alors, il existe un unique homomorphisme $\bar{f}:\sfrac{G}{N} \to \image(f)$ t.q. $f=i \circ \bar{f} \circ p$. De plus, $\bar{f}$ est un isomorphisme.
		\begin{exem}
			$\det:GL(n,\reels) \to \reels_*$.

			$N = \ker(\det) = SL(n,\reels)$ les matrices dont le d\'etrerminant vaut $1$.

			$\image(\det) = \reels_*$.

			Le th\'eor\`eme implique que $\sfrac{GL(n,\reels)}{SL(n,\reels)} \cong \reels_*$.
		\end{exem}
		\begin{proof}~

			\begin{subproof}{Unicit\'e de $\bar{f}$}~

				Supposons que $\bar{f}$ existe.

				Soit $\tilde{f}$ un autre homomorphisme t.q. $f=i \circ \tilde{f} \circ p$.

				Alors, soit $Na \in \sfrac{G}{N}$. On a
				\begin{align*}
					f(a)&= \left(i \circ \tilde{f} \circ p\right)(a)\\
					&= i(\tilde{f}(p(a)))\\
					&= \tilde{f}(Na)
				\end{align*}

				De m\^eme, $f(a)=\bar{f}(a)$.

				Donc, $\bar{f}(Na) = \tilde{f}(Na) = f(a)$, alors $\bar{f} = \tilde{f}$.
			\end{subproof}

			\begin{subproof}{Existence de $\bar{f}$}~

				On veut d\'efinir $\bar{f}(Na)=f(a)$. V\'erifions que cette d\'efinition est sans abigu\"it\'e.

				Supposons $Na=Nb$, c'est-\`a-dire, $ab^{-1} \in N = \ker(f)$.

				Alors, $f(ab^{-1})=e$, donc $f(a)f(b)^{-1}=e$, ainsi $f(a)=f(b)$, qui est \'equivalent \`a $\bar{f}(Na) = \bar{f}(Nb)$.
			\end{subproof}

			\begin{subproof}{$\bar{f}$ est un homomorphisme}~

				$\bar{f}(NaNb) = \bar{f}(Nab) = f(ab) = f(a)f(b) = \bar{f}(Na)\bar{f}(Nb)$.
			\end{subproof}

			\begin{subproof}{$\bar{f}$ est un isomorphisme}~

				\begin{subsubproof}{surjectivit\'e}~

					Soit $b \in \image(f)$. Alors, $b=f(a)$ pour un certain $a \in G$. Donc, $b=\bar{f}(Na)$.
				\end{subsubproof}

				\begin{subsubproof}{injectivit\'e}~

					Supposons $Na \in \ker(\bar{f})$. Alors, $\bar{f}(Na)=e$, donc $f(a)=e$, ainsi $a \in \ker(f) = N$.

					Comme $a \in N$, $Na=N$. Donc, $\ker(\bar{f}) \subseteq \{N\}$, alors $\ker(\bar{f})=N$.
				\end{subsubproof}
			\end{subproof}
		\end{proof}
	\end{thm}
	\begin{exem}~

		\begin{ulist}
			\item $\signe:S_n \to C_2$.

			$\ker(\signe) = A_n$.

			$\image(\signe) = C_2$.

			Alors, $\sfrac{S_n}{A_n} \cong C_2 \cong \entiers_2$.
			\item $\begin{array}{rcl}
				f:\entiers_{10}&\to&\entiers_5\\
				\overline{n}&\mapsto&\overline{n}
			\end{array}$. $f$ est un homomorphisme.

			$\ker(f) = \{\overline0, \overline5\}$.

			$\image(f) = \entiers_5$.

			Alors, $\sfrac{\entiers_{10}}{\{\overline0,\overline5\}} \cong \entiers_5$.
			\item $\begin{array}{rcl}
				\reels^2&\to&\reels\\
				(x,y)&\mapsto&x-y
			\end{array}$. $f$ est un homomorphisme.

			$\ker(f) = \{(a,a) \mid a \in \reels\} \eqcolon N$.

			$\image(f) = \reels$.

			Alors, $\sfrac{\reels^2}{N} \cong \reels$.
		\end{ulist}
	\end{exem}

	\cours
	\begin{rappel}~

		\begin{ulist}
			\item Th\'eor\`eme d'isomorphisme

			\begin{center}
				\begin{tikzcd}[row sep=large]
					G \arrow[r, "f"'] \arrow[d, twoheadrightarrow, "p"']&H\\
					\sfrac{G}{\ker(f)} \arrow[r, dashed, "\bar{f}"]&\image(f) \arrow[u, hookrightarrow, "i"']
				\end{tikzcd}
			\end{center}

			$\exists!$ isomorphisme $\bar{f}: \sfrac{G}{\ker(f)} \to \image(f)$ t.q. $f=i \circ \bar{f} \circ p$.
		\end{ulist}
		\begin{nota}
			$\hookrightarrow$: fonction injective, $\twoheadrightarrow$: fonction surjective.
		\end{nota}

		2 cas importants:
		\begin{nlist}
			\item Si $f$ est surjectif

			$\image(f)=H$.

			Le th\'eor\`eme dit que $\begin{array}{rcl}
				\bar{f}:\sfrac{G}{\ker(f)}&\to&H\\
				Na&\mapsto&f(a)
			\end{array}$ est un isomorphisme.
			\item Si $f$ est injectif

			$\ker(f)=\{e\}$.

			$\sfrac{G}{\ker(f)} = \sfrac{G}{\{e\}}$, les classes sont de la forme $\{e\}a=\{a\}$, donc chaque \'el\'ement de $G$ est seul dans sa classe d'\'equivalence.

			$\begin{array}{rcl}
				G&\xrightarrow{\cong}&\sfrac{G}{\{e\}}\\
				a&\mapsto&\{a\}
			\end{array}$ est un isomorphisme.

			Dans ce cas, $G \cong \sfrac{G}{\{e\}} \cong \image(f)$. On a que $G$ est isomorphe au sous-groupe $\image(f)$ de $H$.
		\end{nlist}
	\end{rappel}
	\begin{coro}~

		$f:G \to H$ un homomorphisme entre groupes finis. Alors, $\card{G} = \card{\ker(f)} \card{\image(f)}$.
		\begin{proof}~

			$\bar{f}$ est une bijection entre $\sfrac{G}{\ker(f)}$ et $\image(f)$. Alors, $\card{\sfrac{G}{\ker(f)}} = \card{\image(f)}$. Donc, $\dfrac{\card{G}}{\card{\ker(f)}} = \card{\image(f)}$.
		\end{proof}
	\end{coro}

	\phantomsection\section*{Groupes monog\`enes}\addcontentsline{toc}{section}{Groupes monog\`enes}

	$a \in G$, $\langle a \rangle = \{a^n \mid n \in \entiers\}$ le groupe \emph{engendr\'e par $a$}.
	\begin{thm}[Classification des groupes monog\`enes]~

		Soit $G = \langle a \rangle$ un groupe monog\`ene. Alors
		\begin{nlist}
			\item Si $o(a)=\infty$, alors $G \cong \entiers$.
			\item Si $o(a)=m$, alors $G \cong \entiers_m$.
		\end{nlist}
		\begin{proof}~

			La fonction $\begin{array}{rcl}
				f:\entiers&\to&G\\
				n&\mapsto&a^n
			\end{array}$ est un homomorphisme. En effet
			\begin{align*}
				f(n_1+n_2)&= a^{n_1+n_2}\\
				&= a^{n_1}a^{n_2}\\
				&= f(n_1)f(n_2)
			\end{align*}

			$f$ est surjectif, car tout \'el\'ement de $G$ s'\'ecrit $a^n$ par d\'efinition.

			Le th\'eor\`eme d'isomorphisme nous dit que $\sfrac{\entiers}{\ker(f)} \cong G$.
			\begin{nlist}
				\item Le seul $n$ t.q. $a^n=e$ est $n=0$, donc $\ker(f)=\{0\}$, alors $\entiers \cong \sfrac{\entiers}{\{0\}} \cong G$.
				\item On veut montrer que $\ker(f)=m\entiers$.
				\begin{subproof}{$\subseteq$:}~

					Supposons $n \in \ker(f)$, alors $f(n) = a^n = e$, donc $o(a) \bigm| n$, c'est-\`a-dire, $n = ko(a) = km \in m\entiers$.

					Ainsi, $\ker(f) \subseteq m\entiers$.
				\end{subproof}
				\begin{subproof}{$\supseteq$:}~

					Supposons $n \in m\entiers$, alors $n=km$, donc $f(n) = a^n = a^{mk} = (a^m)^k = e$.

					Ainsi, $m\entiers \subseteq \ker(f)$.
				\end{subproof}

				Ainsi, $\sfrac{\entiers}{m\entiers} = m\entiers \cong G$.
			\end{nlist}
		\end{proof}
	\end{thm}
	\begin{exem}~

		$\sigma = \begin{pmatrix}
			1&2&3&4&5&6\\
			3&6&1&2&4&5
		\end{pmatrix} \in S_6 = \begin{pmatrix}
			1&3
		\end{pmatrix}\begin{pmatrix}
			2&6&4&5
		\end{pmatrix}$. $o(\sigma) = \verb|ppcm|(2,4) = 4$.

		Par le th\'eor\`eme de classification, $\langle \sigma \rangle \cong \entiers_4$.

		Autrement dit, $\langle \sigma \rangle = \{e, \sigma, \sigma^2, \sigma^3\} = \{\sigma^0, \sigma^1, \sigma^2, \sigma^3\}$.
	\end{exem}

	\setcounter{section}{2}
	\section{Th\'eor\`eme de Cayley}
	$G$ un groupe quelconque, $a \in G$.

	On d\'efinit une fonction
	\begin{equation*}
		\begin{array}{rcl}
			\sigma_a:G&\to&G\\
			b&\mapsto&ab
		\end{array}
	\end{equation*}

	\begin{lem}
		$\sigma_a$ est bijective.
		\begin{proof}~

			$\sigma_{a^{-1}}$ est un inverse de $\sigma_a$. En effet
			\begin{align*}
				\begin{split}
					\sigma_{a^{-1}}(\sigma_a(b))&= \sigma_{a^{-1}}(ab)\\
					&= a^{-1}ab\\
					&= b
				\end{split}
				&
				\begin{split}
					\sigma_a(\sigma_{a^{-1}}(b))&= \sigma_a(a^{-1}b)\\
					&= aa^{-1}b\\
					&= b
				\end{split}
			\end{align*}
		\end{proof}
	\end{lem}
	\begin{rema}
		Ce lemme implique que $\sigma_a \in S(G) = \{f:G \to G \mid f \text{ est bijective}\}$.
	\end{rema}
	\begin{defin}
		On d\'efinit $\begin{array}{rcl}
			\Psi:G&\to&S(G)\\
			a&\mapsto&\sigma_a
		\end{array}$.
	\end{defin}
	\begin{prop}
		$\Psi$ est un homomorphisme de groupes.
		\begin{proof}~

			Soient $a,b,c \in G$.
			\begin{align*}
				\Psi(ab)(c)&= \sigma_{ab}(c)\\
				&= (ab)c\\
				&= a(bc)\\
				&= a\sigma_b(c)\\
				&= \sigma_a(\sigma_b(c))\\
				&= (\sigma_a \circ \sigma_b)(c)
			\end{align*}

			Ainsi, $\sigma_{ab} = \sigma_a \circ \sigma_b$, c'est-\`a-dire, $\Psi(ab) = \Psi(a) \circ \Psi(b)$.
		\end{proof}
	\end{prop}

	\cours
	\begin{rappel}~

		\begin{ulist}
			\item \emph{Th\'eor\`eme d'isomorphismes}: $f:G \to H$, $\sfrac{G}{\ker(f)} \cong \image(f)$.
			\begin{ulist}
				\item Si $f$ \emph{surjective}, $\sfrac{G}{\ker(f) \cong H}$;
				\item Si $f$ \emph{injective}, $G \cong \image(f)$.
			\end{ulist}
			\item $\card{G} = \card{\ker(f)} \cdot \card{\image(f)}$;
			\item Groupe \emph{monog\`ene} (ou cyclique) $G = \langle a \rangle = \{a^n \mid n \in \entiers\}$;
			\item Classification
			\begin{ulist}
				\item $\langle a \rangle \cong \entiers$ si $o(a)=\infty$;
				\item $\langle a \rangle \cong \entiers_m$ si $o(a)=m$.
			\end{ulist}
			\item $a \in G$
			\begin{ulist}
				\item $\begin{array}{rcl}
					\sigma_a:G&\to&G\\
					b&\mapsto&ab
				\end{array}$ est une bijection;
				\item $\begin{array}{rcl}
					\Psi:G&\to&S(G)\\
					a&\mapsto&\sigma_a
				\end{array}$, o\`u $S(G)$ est le groupe des bijections de $G$.
			\end{ulist}
		\end{ulist}
	\end{rappel}
	\begin{prop}
		$\Psi$ est injectif.
		\begin{proof}
			Montrons que $\ker(\Psi)=\{e\}$.

			Supposons que $(a \in G) \in \ker(\Psi)$, alors $\Psi(a)=e$.
			\begin{align*}
				\Psi(a)&= \identite_G\\
				\sigma_a&= \identite_G\\
				\sigma_a(e)&= \identite_G(e)\\
				ae&= e\\
				a&= e
			\end{align*}

			Alors, $\ker(\Psi) \subseteq \{e\}$.

			De plus, $\{e\} \subseteq \ker(\Psi)$ trivialement.

			Ainsi, $\ker(\Psi) = \{e\}$, donc $\Psi$ est injectif.
		\end{proof}
	\end{prop}
	\begin{thm}[Cayley]
		Tout groupe $G$ est isomorphe \`a un sous-groupe d'un groupe sym\'etrique $S(E)$.
		\begin{proof}~

			On prend $E=G$. $\Psi:G \to S(G)$ est un homomorphisme injectif.

			Alors, $G \cong \image(\Psi) \lte S(G)$.
		\end{proof}
	\end{thm}
	\begin{exem}~

		$\mathbb{D}_3 = \{\varepsilon, \alpha, \beta, \gamma, \rho, \sigma\}$.

		Calculons $\Psi$.
		\begin{align*}
			\begin{split}
				\Psi(\varepsilon)&= \begin{pmatrix}
					\varepsilon&\alpha&\beta&\gamma&\rho&\sigma\\
					\varepsilon\varepsilon&\varepsilon\alpha&\varepsilon\beta&\varepsilon\gamma&\varepsilon\rho&\varepsilon\sigma
				\end{pmatrix}\\
				&= \begin{pmatrix}
					\varepsilon&\alpha&\beta&\gamma&\rho&\sigma\\
					\varepsilon&\alpha&\beta&\gamma&\rho&\sigma
				\end{pmatrix}
			\end{split}
			&
			\begin{split}
				\Psi(\alpha)&= \begin{pmatrix}
					\varepsilon&\alpha&\beta&\gamma&\rho&\sigma\\
					\alpha\varepsilon&\alpha\alpha&\alpha\beta&\alpha\gamma&\alpha\rho&\alpha\sigma
				\end{pmatrix}\\
				&= \begin{pmatrix}
					\varepsilon&\alpha&\beta&\gamma&\rho&\sigma\\
					\alpha&\varepsilon&\rho&\sigma&\beta&\gamma
				\end{pmatrix}\\
				&= \begin{pmatrix}
					\varepsilon&\alpha
				\end{pmatrix} \begin{pmatrix}
					\beta&\rho
				\end{pmatrix} \begin{pmatrix}
					\gamma&\sigma
				\end{pmatrix}
			\end{split}
			\\[1em]
			\begin{split}
				\Psi(\beta)&= \begin{pmatrix}
					\varepsilon&\alpha&\beta&\gamma&\rho&\sigma\\
					\beta\varepsilon&\beta\alpha&\beta\beta&\beta\gamma&\beta\rho&\beta\sigma
				\end{pmatrix}\\
				&= \begin{pmatrix}
					\varepsilon&\alpha&\beta&\gamma&\rho&\sigma\\
					\beta&\sigma&\varepsilon&\rho&\gamma&\alpha
				\end{pmatrix}\\
				&= \begin{pmatrix}
					\varepsilon&\beta
				\end{pmatrix} \begin{pmatrix}
					\alpha&\sigma
				\end{pmatrix} \begin{pmatrix}
					\gamma&\rho
				\end{pmatrix}
			\end{split}
			&
			\begin{split}
				\Psi(\gamma)&= \begin{pmatrix}
					\varepsilon&\alpha&\beta&\gamma&\rho&\sigma\\
					\gamma\varepsilon&\gamma\alpha&\gamma\beta&\gamma\gamma&\gamma\rho&\gamma\sigma
				\end{pmatrix}\\
				&= \begin{pmatrix}
					\varepsilon&\alpha&\beta&\gamma&\rho&\sigma\\
					\gamma&\rho&\sigma&\varepsilon&\alpha&\beta
				\end{pmatrix}\\
				&= \begin{pmatrix}
					\varepsilon&\gamma
				\end{pmatrix} \begin{pmatrix}
					\alpha&\rho
				\end{pmatrix} \begin{pmatrix}
					\beta&\sigma
				\end{pmatrix}
			\end{split}
			\\[1em]
			\begin{split}
				\Psi(\rho)&= \begin{pmatrix}
					\varepsilon&\alpha&\beta&\gamma&\rho&\sigma\\
					\rho\varepsilon&\rho\alpha&\rho\beta&\rho\gamma&\rho\rho&\rho\sigma
				\end{pmatrix}\\
				&= \begin{pmatrix}
					\varepsilon&\alpha&\beta&\gamma&\rho&\sigma\\
					\rho&\gamma&\alpha&\beta&\sigma&\varepsilon
				\end{pmatrix}\\
				&= \begin{pmatrix}
					\varepsilon&\rho&\sigma
				\end{pmatrix} \begin{pmatrix}
					\alpha&\gamma&\beta
				\end{pmatrix}
			\end{split}
			&
			\begin{split}
				\Psi(\sigma)&= \begin{pmatrix}
					\varepsilon&\alpha&\beta&\gamma&\rho&\sigma\\
					\sigma\varepsilon&\sigma\alpha&\sigma\beta&\sigma\gamma&\sigma\rho&\sigma\sigma
				\end{pmatrix}\\
				&= \begin{pmatrix}
					\varepsilon&\alpha&\beta&\gamma&\rho&\sigma\\
					\sigma&\beta&\gamma&\alpha&\varepsilon&\rho
				\end{pmatrix}\\
				&= \begin{pmatrix}
					\varepsilon&\sigma&\rho
				\end{pmatrix} \begin{pmatrix}
					\alpha&\beta&\gamma
				\end{pmatrix}
			\end{split}
		\end{align*}
	\end{exem}
	\begin{rema}
		Le groupe $S(G)$ est toujours plus grand que $G$.
		\begin{exem}
			Si on prend $G=S_n$, $\Psi:S_n \to S(S_n)$, o\`u $\card{S_n} = n!$ et $\card{S(S_n)} = (n!)!$.
		\end{exem}
	\end{rema}
	\begin{exem}
		$G = \entiers_2 \times \entiers_2 = \{(\overline0,\overline0), (\overline0,\overline1), (\overline1,\overline0), (\overline1,\overline1)\}$.
		\begin{equation*}
			\Psi:G \to S(G) \cong S_4
		\end{equation*}

		Num\'erotons les \'el\'ements de $G$ comme $\{1,2,3,4\}$.
		\begin{align*}
			\begin{split}
				\Psi(\overline0,\overline0)&= \identite
			\end{split}
			&
			\begin{split}
				\Psi(\overline0,\overline1)&= \begin{pmatrix}
					1&2&3&4\\
					2&1&4&3
				\end{pmatrix}\\
				&= \begin{pmatrix}
					1&2
				\end{pmatrix} \begin{pmatrix}
					3&4
				\end{pmatrix}
			\end{split}
			&
			\begin{split}
				\Psi(\overline1,\overline0)&= \begin{pmatrix}
					1&2&3&4\\
					3&4&1&2
				\end{pmatrix}\\
				&= \begin{pmatrix}
					1&3
				\end{pmatrix} \begin{pmatrix}
					2&4
				\end{pmatrix}
			\end{split}
			&
			\begin{split}
				\Psi(\overline1,\overline1)&= \begin{pmatrix}
					1&2&3&4\\
					4&3&2&1
				\end{pmatrix}\\
				&= \begin{pmatrix}
					1&4
				\end{pmatrix} \begin{pmatrix}
					2&3
				\end{pmatrix}
			\end{split}
		\end{align*}

		Cayley nous dit que $\entiers_2 \times \entiers_2 \cong \{\identite, \begin{pmatrix}
			1&2
		\end{pmatrix} \begin{pmatrix}
			3&4
		\end{pmatrix}, \begin{pmatrix}
			1&3
		\end{pmatrix} \begin{pmatrix}
			2&4
		\end{pmatrix}, \begin{pmatrix}
			1&4
		\end{pmatrix} \begin{pmatrix}
			2&3
		\end{pmatrix}\}$.
	\end{exem}

	\phantomsection\chapter*{Actions de groupes}\addcontentsline{toc}{chapter}{Actions de groupes}
	\renewcommand{\leftmark}{ACTIONS DE GROUPES}
	Soient $G$ un groupe et $E$ un ensemble.
	\begin{defin}
		Une \emph{action} de $G$ sur $E$ est une fonction $\begin{array}{rcl}
			\bullet:G \times E&\to&E\\
			(a,x)&\mapsto&a \bullet x
		\end{array}$, satisfaisant \`a
		\begin{nlist}
			\item $\forall x \in E$, $e \bullet x = x$;
			\item $\forall a,b \in G$, $\forall x \in e$, $a \bullet (b \bullet x) = (ab) \bullet x$.
		\end{nlist}
	\end{defin}
	\begin{exem}~

		\begin{nlist}
			\item $G = GL(n, \reels)$, $E=\reels^n$.
			\[
			\begin{array}{rcl}
				\bullet:GL(n, \reels) \times \reels^n&\to&\reels^n\\
				(M, \vec{v})&\mapsto&M\vec{v}
			\end{array}
			\]
			est une action.
			\item $G = S_n$, $E = \{1,2,\dots,n\}$.
			\[
			\begin{array}{rcl}
				\bullet:S_n \times \{1,2,\dots,n\}&\to&\{s,2,\dots,n\}\\
				(\sigma,i)&\mapsto&\sigma(i)
			\end{array}
			\]
			est une action.
			\item $G = \entiers_2$, $E =$ {\LARGE\faChild} $\subseteq \reels^2$.

			$\entiers_2$ agit sur $E$ par
			\begin{align*}
				\overline0 \bullet (x,y)&= (x,y)\\
				\overline1 \bullet (x,y)&= (-x,y)
			\end{align*}
			\item $G = \entiers_3$, $E =$ {\LARGE\faRecycle} $\subseteq \reels^2$.

			$G$ agit sur $E$ o\`u $\overline1$ est une rotation de $120$\textdegree et $\overline2$ est une rotation de $240$\textdegree.
			\item $\mathbb{D}_3$ agit sur un triangle \'equilat\'eral.
			\item $G$ agit sur lui-m\^eme par
			\[
			\begin{array}{rcl}
				\bullet:G \times G&\to&G\\
				(a,b)&\mapsto&ab
			\end{array}
			\]
			\item $G$ agit sur lui-m\^eme aussi par
			\[
			\begin{array}{rcl}
				\bullet:G \times G&\to&G\\
				(a,b)&\mapsto&aba^{-1}
			\end{array}
			\]
			\begin{subproof}{En effet}~

				\begin{nlist}
					\item Soit $b \in G$, alors
					\begin{align*}
						e \bullet b&= ebe^{-1}\\
						&= b
					\end{align*}
					\item Soient $a,b,c \in G$
					\begin{align*}
						(ab) \bullet c&= (ab)c(ab)^{-1}\\
						&= abcb^{-1}a^{-1}\\
						&= a(b \bullet c)a^{-1}\\
						&= a \bullet (b \bullet c)
					\end{align*}
				\end{nlist}
			\end{subproof}
		\end{nlist}
	\end{exem}
	\begin{defin}
		Une \emph{action} de $G$ sur $E$ est un homomorphisme $\Psi:G \to E$.
	\end{defin}

	Pour passer de la premi\`ere d\'efinition \`a la deuxi\`eme, on pose $\Psi(a) = \sigma_a$, o\`u $\sigma_a(x) = a \bullet x$.

	Pour passer de la deuxi\`eme d\'efinition \`a la premi\`ere, on pose $a \bullet x = (\Psi(a))(x)$.
	\begin{defin}
		$G$ un groupe qui agit sur un ensemble $E$.

		Soit $x \in E$.

		L'\emph{orbite} de $x$ est l'ensemble $G \bullet x = \{a \bullet x \mid a \in G\} \subseteq E$.
	\end{defin}
	\begin{defin}
		Soit $x \in E$.

		Le \emph{stabilisateur} de $x$ est $\stab(x) = \{a \in G \mid a \bullet x = x\} \lte G$.
	\end{defin}
	\begin{exem}
		avec $\mathbb{D}_3\footnote{voir p.\pageref{D3}}$

		\begin{figure}[h]
			\centering
			\begin{tikzpicture}[scale=3]
				\draw (0:5mm) node (a) {} -- ++(120:1cm) node (b) {} -- ++(240:1cm) node (c) {} -- cycle;
				\fill[Col1] (a) circle (1pt) (b) circle (1pt) (c) circle (1pt);
				\fill[Col4] ($(a)!.2!(b)$) circle (1pt) ($(b)!.2!(c)$) circle (1pt) ($(c)!.2!(a)$) circle (1pt) ($(b)!.2!(a)$) circle (1pt) ($(c)!.2!(b)$) circle (1pt) ($(a)!.2!(c)$) circle (1pt);
				\fill[Col6] ($(a)!.5!(b)$) circle (1pt) ($(b)!.5!(c)$) circle (1pt) ($(c)!.5!(a)$) circle (1pt);
				\node[anchor=south] () at (b.north) {$x$};
				\begin{scope}[xshift=2cm, yshift=5mm, scale=1]
					\node () at (0,0) {\textcolor{Col1}{\CIRCLE}, \textcolor{Col4}{\CIRCLE} et \textcolor{Col6}{\CIRCLE} sont trois orbites.};
					\node () at (0,-.2) {$\stab(x) = \{\varepsilon, \alpha\}$};
				\end{scope}
			\end{tikzpicture}
		\end{figure}
	\end{exem}

	\cours
	\begin{rappel}~

		\begin{ulist}
			\item Th\'eor\`eme de Cayley
			\begin{ulist}
				\item $\begin{array}{rcl}
					\Psi:G&\to&S(G)\\
					a&\mapsto&\sigma_a
				\end{array}$, o\`u $\sigma_a(b)=ab$ est un homomorphisme injectif;
				\item $G \cong \image(\Psi) \lte S(G)$.
			\end{ulist}
			\item Action de $G$ sur $E$
			\begin{ulist}
				\item $a \bullet x \in E$
				\begin{nlist}
					\item $e \bullet x = x$
					\item $(ab) \bullet x = a \bullet (b \bullet x)$
				\end{nlist}
				\item homomorphisme $\begin{array}{rcl}
					\Psi:G&\to&S(E)\\
					a&\mapsto&(x \mapsto a \bullet x)
				\end{array}$
			\end{ulist}
			\item Orbite de $x \in E$: $G \bullet x = \{a \bullet x \in e \mid a \in G\} \subseteq E$
			\item Stabilisateur de $x \in E$: $\stab(x) = \{a \in G \mid a \bullet x = x\} \lte G$
		\end{ulist}
	\end{rappel}
	\begin{rema}
		$x \bullet a$ n'est pas d\'efini.
	\end{rema}
	\begin{prop}
		$\stab(x) \lte G$.
		\begin{proof}~

			\begin{align*}
				e \bullet x&= x\\
				\Rightarrow e&\in \stab(x)\\
				\Rightarrow \stab(x)&\neq \emptyset
			\end{align*}

			Soient $a,b \in \stab(x)$. On veut v\'erifier que $ab^{-1} \in \stab(x)$, c'est-\`a-dire $(ab^{-1}) \bullet x = x$.

			Remarquons que $b \bullet x = x$, alors $b^{-1} \bullet (b \bullet x) = b^{-1} \bullet x$, donc $(b^{-1}b) \bullet x = b^{-1} \bullet x$, donc $e \bullet x = b^{-1} \bullet x$, ainsi $x = b^{-1} \bullet x$.

			On a donc
			\begin{align*}
				(ab^{-1}) \bullet x&= a \bullet (b^{-1} \bullet x)\\
				&= a \bullet x\\
				\text{\footnotesize car $a \in \stab(x)$}&= x
			\end{align*}

			Ainsi, $ab^{-1} \in \stab(x)$.
		\end{proof}
	\end{prop}
	\begin{prop}
		La relation $x \sim y$ si, et seulement si, $\exists a \in G$ t.q. $y=a \bullet x$ est une \'equivalence donc les classes d'\'equivalence sont les orbites, c'est-\`a-dire, $\overline{x} = G \bullet x$.

		En particulier, les orbites forment une \emph{partition} de $E$.
		\begin{proof}~

			\begin{subproof}{\'Equivalence}~

				\begin{itemize}
					\item[(Refl)] Soit $x \in E$.

					Or, $x = e \bullet x$, donc $x \sim x$.
					\item[(Sym)] Soient $x,y \in E$ t.q. $x \sim y$.

					Ainsi, $\exists a \in G$ t.q. $y = a \bullet x$.

					Or, $a^{-1} \in G$, donc $a^{-1} \bullet y = a^{-1} \bullet (a \bullet x) = (a^{-1}a) \bullet x = x$.

					Ainsi, $y \sim x$.
					\item[(Trans)] Soient $x,y,z \in E$ t.q. $x \sim y$ et $y \sim z$. Alors, $\exists a,b \in G$ t.q. $y = a \bullet x$ et $z = b \bullet y$.

					Or, $z = b \bullet y = b \bullet (a \bullet x) = (ba) \bullet x$, o\`u $ba \in G$.

					Ainsi, $x \sim z$.
				\end{itemize}
			\end{subproof}
			\begin{subproof}{Les classes d'\'equivalence sont les orbites}~

				La classe d'\'equivalence de $x \in E$, par d\'efinition, est
				\begin{align*}
					\overline{x} &= \{y \in E \mid x \sim y\}\\
					&= \{y \in E \mid \exists a \in G, y = a \bullet x\}\\
					&= \{a \bullet x \mid a \in G\}\\
					&= G \bullet x
				\end{align*}
			\end{subproof}
		\end{proof}
	\end{prop}
	\begin{exem}~

		\begin{nlist}
			\item $G = \left\{\begin{psmallmatrix}
				1&a\\0&1
			\end{psmallmatrix} ~\middle|~ a \in \reels\right\}$, $E = \reels^2$ avec l'action $M \bullet \vec{v} = M\vec{v}$ la multiplication matricielle usuelle.

			Calculons l'orbite d'un vecteur $\vec{v} = \begin{psmallmatrix}
				x\\y
			\end{psmallmatrix} \in \reels^2$.
			\begin{align*}
				G \bullet \vec{v}&= \left\{\begin{pmatrix}
					1&a\\0&1
				\end{pmatrix} \begin{pmatrix}
					x\\y
				\end{pmatrix} ~\middle|~ a \in \reels\right\}\\
				&= \left\{\begin{pmatrix}
					x+ay\\y
				\end{pmatrix} ~\middle|~ a \in \reels\right\}
			\end{align*}

			Si $y=0$, $G \bullet \begin{psmallmatrix}
				x\\0
			\end{psmallmatrix} = \left\{\begin{psmallmatrix}
				x\\0
			\end{psmallmatrix}\right\}$.

			Si $y \neq 0$, $\{x+ay, a \in \reels\} = \reels$, donc $G \bullet \begin{psmallmatrix}
				x\\y
			\end{psmallmatrix} = \left\{\begin{psmallmatrix}
				a\\y
			\end{psmallmatrix} ~\middle|~ a \in \reels\right\}$.
			\item $G = S_3$, $E = \mathcal{P}(\{1,2,3\}) = \{\emptyset, \{1\}, \{2\}, \{3\}, \{1,2\}, \{1,3\}, \{2,3\}, \{1,2,3\}\}$.

			$S_3$ agit sur un sous-ensemble en agissant sur chaque \'el\'ement du sous-ensemble, c'est-\`a-dire, par exemple, $\sigma \bullet \{i,j\} = \{\sigma(i), \sigma(j)\}$.

			Les orbites de l'action sont
			\begin{equation*}
					\{\emptyset\}; \{\{1\}, \{2\}, \{3\}\}; \{\{1,2\}, \{1,3\}, \{2,3\}\}; \{\{1,2,3\}\}
			\end{equation*}
		\end{nlist}
	\end{exem}

	\cours
	\begin{rappel}~

		\begin{ulist}
			\item Orbite d'un \'el\'ement $x \in E$
			\begin{ulist}
				\item $G \bullet x = \{a \bullet x \mid a \in G\}$
			\end{ulist}
			\item Stabilisateur de $x \in E$
			\begin{ulist}
				\item $\stab(x) = \{a \in G \mid a \bullet x = x\} \lte G$
			\end{ulist}
			\item Les orbites forment une \emph{partition} de $E$
			\begin{ulist}
				\item la m\^eme que la relation $x \sim y \Leftrightarrow y = a \bullet x$
			\end{ulist}
		\end{ulist}
	\end{rappel}
	\begin{lem}~

		Si $y = a \bullet x$, alors $a^{-1} \bullet y = x$.
		\begin{proof}~

			\begin{align*}
				y&= a \bullet x\\
				a^{-1} \bullet y&= (a^{-1}a) \bullet x\\
				&= x
			\end{align*}
		\end{proof}
	\end{lem}
	\begin{prop}~

		Si $y=a \bullet x$, alors $\stab(y) = a \stab(x) a^{-1} = \{aha^{-1} \mid h \in \stab(x)\}$.

		En particulier, si $x$ et $y$ sont dans la m\^eme orbite, alors $\stab(x) \cong \stab(y)$.
		\begin{proof}~

			\begin{subproof}{$\subseteq$:}~

				Supposons $b \in \stab(y)$. Alors, $b \bullet y = y$.

				\'Ecrivons $b = a(a^{-1}ba)b^{-1}$.

				De plus
				\begin{align*}
					(a^{-1}ba) \bullet x&= a^{-1} \bullet (b \bullet (a \bullet x))\\
					&= a^{-1} \bullet (b \bullet y)\\
					&= a^{-1} \bullet y\\
					&= x
				\end{align*}

				Donc, $a^{-1}ba \in \stab(x)$, alors $b \in a \stab(y) a^{-1}$.
			\end{subproof}
			\begin{subproof}{$\supseteq$:}~

				Supposons $b \in \stab(x)$, alors $aba^{-1} \in a \stab(x) a^{-1}$.
				\begin{align*}
					(aba^{-1}) \bullet y&= a \bullet(b \bullet (a^{-1} \bullet y))\\
					&= a \bullet (b \bullet x)\\
					&= a \bullet x\\
					&= y
				\end{align*}

				Alors, $aba^{-1} \in \stab(y)$.
			\end{subproof}

			Ainsi, $\stab(y) = a \stab(x) a^{-1}$.
		\end{proof}
		\begin{subproof}{De plus,}~

			On d\'efinit $\begin{array}{rcl}
				f:\stab(x)&\to&\stab(y)\\
				h&\mapsto&aha^{-1}
			\end{array}$.

			$f$ est bijective, car son inverse est $\begin{array}{rcl}
				g:\stab(y)&\to&\stab(x)\\
				b&\mapsto&a^{-1}ba
			\end{array}$.%TODO prouver

			En effet,
			\begin{align*}
				\begin{split}
					f(g(i))&= f(a^{-1}ia)\\
					&= aa^{-1}iaa^{-1}\\
					&= i
				\end{split}
				&
				\begin{split}
					g(f(i))&= g(aia^{-1})\\
					&= a^{-1}aia^{-1}a\\
					&= i
				\end{split}
			\end{align*}

			$f$ est un homomorphisme, car
			\begin{align*}
				f(h_1h_2)&= a(h_1h_2)a^{-1}\\
				&= ah_1a^{-1}ah_2a^{-1}\\
				&= f(h_1)f(h_2)
			\end{align*}

			Ainsi, $f$ est un isomorphisme entre $\stab(x)$ et $\stab(y)$.
		\end{subproof}
	\end{prop}
	\begin{exem}~

		\begin{ulist}
			\item $G = \mathbb{D}_6$, les isom\'etries d'un hexagone r\'egulier, et $E = \{\text{diagonales de l'hexagone}\}$.
			\begin{figure}[h]
				\centering
				\begin{tikzpicture}[scale=.75]
					\begin{scope}[rotate=90]
						\coordinate (1) at (0:1);
						\coordinate (2) at (-60:1);
						\coordinate (3) at (-120:1);
						\coordinate (4) at (-180:1);
						\coordinate (5) at (-240:1);
						\coordinate (6) at (-300:1);
						\draw (1) -- (2) -- (3) -- (4) -- (5) -- (6) -- cycle;
						\draw (1) -- (3);
					\end{scope}
					\begin{scope}[rotate=90, yshift=-25mm]
						\coordinate (1) at (0:1);
						\coordinate (2) at (-60:1);
						\coordinate (3) at (-120:1);
						\coordinate (4) at (-180:1);
						\coordinate (5) at (-240:1);
						\coordinate (6) at (-300:1);
						\draw (1) -- (2) -- (3) -- (4) -- (5) -- (6) -- cycle;
						\draw (1) -- (4);
					\end{scope}
					\begin{scope}[rotate=90, yshift=-5cm]
						\coordinate (1) at (0:1);
						\coordinate (2) at (-60:1);
						\coordinate (3) at (-120:1);
						\coordinate (4) at (-180:1);
						\coordinate (5) at (-240:1);
						\coordinate (6) at (-300:1);
						\draw (1) -- (2) -- (3) -- (4) -- (5) -- (6) -- cycle;
						\draw (1) -- (5);
					\end{scope}
					\begin{scope}[rotate=90, yshift=-75mm]
						\coordinate (1) at (0:1);
						\coordinate (2) at (-60:1);
						\coordinate (3) at (-120:1);
						\coordinate (4) at (-180:1);
						\coordinate (5) at (-240:1);
						\coordinate (6) at (-300:1);
						\draw (1) -- (2) -- (3) -- (4) -- (5) -- (6) -- cycle;
						\draw (2) -- (4);
					\end{scope}
					\begin{scope}[rotate=90, yshift=-10cm]
						\coordinate (1) at (0:1);
						\coordinate (2) at (-60:1);
						\coordinate (3) at (-120:1);
						\coordinate (4) at (-180:1);
						\coordinate (5) at (-240:1);
						\coordinate (6) at (-300:1);
						\draw (1) -- (2) -- (3) -- (4) -- (5) -- (6) -- cycle;
						\draw (2) -- (5);
					\end{scope}
					\begin{scope}[rotate=90, yshift=-125mm]
						\coordinate (1) at (0:1);
						\coordinate (2) at (-60:1);
						\coordinate (3) at (-120:1);
						\coordinate (4) at (-180:1);
						\coordinate (5) at (-240:1);
						\coordinate (6) at (-300:1);
						\draw (1) -- (2) -- (3) -- (4) -- (5) -- (6) -- cycle;
						\draw (2) -- (6);
					\end{scope}
					\begin{scope}[rotate=90, yshift=-15cm]
						\coordinate (1) at (0:1);
						\coordinate (2) at (-60:1);
						\coordinate (3) at (-120:1);
						\coordinate (4) at (-180:1);
						\coordinate (5) at (-240:1);
						\coordinate (6) at (-300:1);
						\draw (1) -- (2) -- (3) -- (4) -- (5) -- (6) -- cycle;
						\draw (3) -- (5);
					\end{scope}
					\begin{scope}[rotate=90, yshift=-175mm]
						\coordinate (1) at (0:1);
						\coordinate (2) at (-60:1);
						\coordinate (3) at (-120:1);
						\coordinate (4) at (-180:1);
						\coordinate (5) at (-240:1);
						\coordinate (6) at (-300:1);
						\draw (1) -- (2) -- (3) -- (4) -- (5) -- (6) -- cycle;
						\draw (3) -- (6);
					\end{scope}
					\begin{scope}[rotate=90, yshift=-20cm]
						\coordinate (1) at (0:1);
						\coordinate (2) at (-60:1);
						\coordinate (3) at (-120:1);
						\coordinate (4) at (-180:1);
						\coordinate (5) at (-240:1);
						\coordinate (6) at (-300:1);
						\draw (1) -- (2) -- (3) -- (4) -- (5) -- (6) -- cycle;
						\draw (4) -- (6);
					\end{scope}
					\node () at (0,-1.5) {$a$};
					\node () at (2.5,-1.5) {$b$};
					\node () at (5,-1.5) {$c$};
					\node () at (7.5,-1.5) {$d$};
					\node () at (10,-1.5) {$e$};
					\node () at (12.5,-1.5) {$f$};
					\node () at (15,-1.5) {$g$};
					\node () at (17.5,-1.5) {$h$};
					\node () at (20,-1.5) {$i$};
				\end{tikzpicture}
				\caption{Repr\'esentation des diagonales d'un hexagone r\'egulier}
			\end{figure}

			Calculons les orbites:
			\begin{align*}
				o_1&= \{a,c,d,f,g,i\}\\
				o_2&= \{b,e,h\}
			\end{align*}

			Calculons les stabilisateurs:
			\begin{align*}
				\stab(a)&= \{\varepsilon, \mu\}\\
				\stab(b)&= \{\varepsilon, \phi, \eta, \lambda\}
			\end{align*}
			\begin{figure}[h]
				\centering
				\begin{tikzpicture}
					\coordinate (1) at (90:1);
					\coordinate (2) at (30:1);
					\coordinate (3) at (-30:1);
					\coordinate (4) at (-90:1);
					\coordinate (5) at (-150:1);
					\coordinate (6) at (150:1);
					\draw (1) -- (2) -- (3) -- (4) -- (5) -- (6) -- cycle;
					\draw (30:-1.25) -- (30:1.5);
					\draw[{To[length=3pt]}-{To[length=3pt]}] (17:1.25) -- (43:1.25);
					\node () at (0,-2) {$\mu$};
					\begin{scope}[xshift=4cm]
						\coordinate (1) at (90:1);
						\coordinate (2) at (30:1);
						\coordinate (3) at (-30:1);
						\coordinate (4) at (-90:1);
						\coordinate (5) at (-150:1);
						\coordinate (6) at (150:1);
						\draw (1) -- (2) -- (3) -- (4) -- (5) -- (6) -- cycle;
						\draw[-{To[length=5pt]}] (30:1.5) arc(30:210:1.5);
						\node () at (0,-2) {$\phi$};
					\end{scope}
					\begin{scope}[xshift=8cm]
						\coordinate (1) at (90:1);
						\coordinate (2) at (30:1);
						\coordinate (3) at (-30:1);
						\coordinate (4) at (-90:1);
						\coordinate (5) at (-150:1);
						\coordinate (6) at (150:1);
						\draw (1) -- (2) -- (3) -- (4) -- (5) -- (6) -- cycle;
						\draw (90:-1.25) -- (90:1.5);
						\draw[{To[length=3pt]}-{To[length=3pt]}] (103:1.25) -- (77:1.25);
						\node () at (0,-2) {$\eta$};
					\end{scope}
					\begin{scope}[xshift=12cm]
						\coordinate (1) at (90:1);
						\coordinate (2) at (30:1);
						\coordinate (3) at (-30:1);
						\coordinate (4) at (-90:1);
						\coordinate (5) at (-150:1);
						\coordinate (6) at (150:1);
						\draw (1) -- (2) -- (3) -- (4) -- (5) -- (6) -- cycle;
						\draw (0:-1.25) -- (0:1.5);
						\draw[{To[length=3pt]}-{To[length=3pt]}] (-13:1.25) -- (13:1.25);
						\node () at (0,-2) {$\lambda$};
					\end{scope}
				\end{tikzpicture}
				\caption{Repr\'esentation des isom\'etries pr\'esentes dans les stabilisateurs}
			\end{figure}
			\item Combien un cube a-t-il de rotations?
			\begin{figure}[h]
				\centering
				\begin{tikzpicture}[scale=2]
					\coordinate (1) at (0,0,0);
					\coordinate (2) at (0,0,1);
					\coordinate (3) at (0,1,0);
					\coordinate (4) at (0,1,1);
					\coordinate (5) at (1,0,0);
					\coordinate (6) at (1,0,1);
					\coordinate (7) at (1,1,0);
					\coordinate (8) at (1,1,1);
					\draw (2) -- (4) -- (8) -- (6) -- (2) (6) -- (5) -- (7) -- (8) (4) -- (3) -- (7);
					\draw[dashed] (3) -- (1) -- (2) (1) -- (5);
					\fill[fill opacity=.6, fill=col1] (3) -- (4) -- (8) -- (7) -- cycle;
					\draw (.5,1.5,.5) edge (.5,1,.5) (.5,1,.5) edge[dashed] (.5,-.2,.5) (.5,-.2,.5) edge (.5,-.5,.5);
					\begin{scope}[canvas is xz plane at y=1]
						\fill (.5,.5) circle (1.5pt);
					\end{scope}
					\begin{scope}[canvas is xz plane at y=0]
						\fill (.5,.5) circle (1.5pt);
					\end{scope}
					\begin{scope}[canvas is xz plane at y=1.35]
						\draw[->] (.5,.6) arc(180:-90:.1);
					\end{scope}
				\end{tikzpicture}
			\end{figure}
			\begin{ulist}
				\item Combien de rotations pr\'eservent la face du haut?

				$4$, en incluant l'identit\'e.
				\item Il y a $6$ faces et $4$ fa\c cons d'envoyer la face du haut \`a une face quelconque, donc il y a $4 \times 6 = 24$ rotations au total.
			\end{ulist}
		\end{ulist}
	\end{exem}
	\begin{thm}[orbite-stabilisateur]~

		Soit $G$ un groupe fini qui agit sur $E$ un ensemble fini. Soit $x \in E$.

		Alors, $\card{G} = \card{G \bullet x} \cdot \card{\stab(x)}$.
		\begin{proof}~

			Posons $H \coloneq \stab(x)$. Notons $\sfrac{G}{H} \coloneq \{aH \mid a \in G\}$ l'ensemble des classes d'\'equivalence \`a gauche modulo $H$.

			On d\'efinit $\begin{array}{rcl}
				f:\sfrac{G}{H}&\to&G \bullet x\\
				aH&\mapsto&a \bullet x
			\end{array}$.

			On veut montrer que $f$ est bijective.
			\begin{subproof}{$f$ est d\'efinie sans ambigu\"it\'e}~

				Supposons que $aH=bH$, avec $a,b \in G$. Alors, $a^{-1}b \in H$. Donc,
				\begin{align*}
					(a^{-1}b) \bullet x&= x\\
					a^{-1} \bullet (b \bullet x)&= x\\
					b \bullet x&= a \bullet x\\
					f(bH)&= f(aH)
				\end{align*}
			\end{subproof}
			\begin{subproof}{$f$ est bijective}~

				\begin{subsubproof}{$f$ est injective}~

					Soient $aH, bH \in \sfrac{G}{H}$ t.q. $f(aH) = f(bH)$. Alors
					\begin{align*}
						a \bullet x&= b \bullet x\\
						x&= a^{-1} \bullet (b \bullet x)\\
						&= (a^{-1}b) \bullet x\\
						&\Rightarrow a^{-1}b \in \stab(x) = H\\
						&\Rightarrow aH=bH
					\end{align*}

					Ainsi, $f$ est injective.
				\end{subsubproof}
				\begin{subsubproof}{$f$ est surjective}~

					Soit $y \in G \bullet x$, alors, $\exists a \in G$ t.q. $a \bullet x = y$, donc $f(aH)=y$.

					Ainsi, $f$ est surjective.
				\end{subsubproof}

				Comme $f$ est injective et surjective, $f$ est bijective.
			\end{subproof}

			Ainsi, $\card{\sfrac{G}{H}} = \card{G \bullet x}$, c'est-\`a-dire $\dfrac{\card{G}}{\card{H}} = \card{G \bullet x}$, donc $\card{G} = \card{G \bullet x} \cdot \card{H}$.
		\end{proof}
	\end{thm}
	\begin{exem}[revenons au cube]~

		$E_1 = \{\text{faces du cube}\}$. $E_2 = \{\text{sommets du cube}\}$.

		$G$ agit sur $E_2$. Il n'a qu'une seule orbite et $\card{E_2} = 8$.

		$\stab(a) = \{\varepsilon, \theta, \phi\}$.

		$24 = \card{G} = \card{E_2} \cdot \card{\stab(a)} = 8 \times 3$.
		\begin{figure}[h]
			\centering
			\begin{tikzpicture}[scale=2]
				%cube
				\coordinate (1) at (0,0,0);
				\coordinate (2) at (0,0,1);
				\coordinate (3) at (0,1,0);
				\coordinate (4) at (0,1,1);
				\coordinate (5) at (1,0,0);
				\coordinate (6) at (1,0,1);
				\coordinate (7) at (1,1,0);
				\coordinate (8) at (1,1,1);
				\draw (2) -- (4) -- (8) -- (6) -- (2) (6) -- (5) -- (7) -- (8) (4) -- (3) -- (7);
				\draw[dashed] (3) -- (1) -- (2) (1) -- (5);

				\fill (4) circle (1.2pt);
				\node[font=\Large] () at (-.2,1.1,1.1) {$a$};
			\end{tikzpicture}
			\hspace{1.5cm}
			\begin{tikzpicture}[scale=2]
				%cube
				\coordinate (1) at (0,0,0);
				\coordinate (2) at (0,0,1);
				\coordinate (3) at (0,1,0);
				\coordinate (4) at (0,1,1);
				\coordinate (5) at (1,0,0);
				\coordinate (6) at (1,0,1);
				\coordinate (7) at (1,1,0);
				\coordinate (8) at (1,1,1);
				\draw (2) -- (4) -- (8) -- (6) -- (2) (6) -- (5) -- (7) -- (8) (4) -- (3) -- (7);
				\draw[dashed] (3) -- (1) -- (2) (1) -- (5);

				\draw (-.25,1.25,1.25) edge (0,1,1) (0,1,1) edge[dashed] (1,0,0) (1,0,0) edge (1.25,-.25,-.25);
			\end{tikzpicture}
			\hspace{1.5cm}
			\begin{tikzpicture}[scale=1.2]
				\coordinate (1) at (90:1);
				\coordinate (2) at (30:1);
				\coordinate (3) at (-30:1);
				\coordinate (4) at (-90:1);
				\coordinate (5) at (-150:1);
				\coordinate (6) at (150:1);
				\draw (1) -- (2) -- (3) -- (4) -- (5) -- (6) -- cycle;

				\draw (0,0) edge (2) edge (4) edge (6);
				\draw[dashed] (0,0) edge (1) edge (3) edge (5);

				\fill (0,0) circle (2pt);

				\draw[-{To[length=5pt]}] (85:1.5) arc(85:-25:1.5);
				\draw[-{To[length=5pt]}] (95:1.5) arc(95:205:1.5);
				\node[font=\Large] () at (30:1.7) {$\theta$};
				\node[font=\Large] () at (150:1.7) {$\phi$};
			\end{tikzpicture}
			\caption{Repr\'esentation du sommet $a$ et des rotations $\theta$ et $\phi$}
		\end{figure}

		$G$ agit aussi sur $A = \{\text{ar\^etes du cube}\}$.

		Il y a encore une seule orbite, $A$ et $\card{A} = 12$.

		$\stab\left( \overline{\alpha\beta} \right) = \{\varepsilon, \rho\}$.

		$24 = \card{G} = \card{A} \cdot \card{\stab\left( \overline{\alpha\beta} \right)} = 12 \times 2$.
		\begin{figure}[h]
			\centering
			\begin{tikzpicture}[scale=2]
				%cube
				\coordinate (1) at (0,0,0);
				\coordinate (2) at (0,0,1);
				\coordinate (3) at (0,1,0);
				\coordinate (4) at (0,1,1);
				\coordinate (5) at (1,0,0);
				\coordinate (6) at (1,0,1);
				\coordinate (7) at (1,1,0);
				\coordinate (8) at (1,1,1);
				\draw (2) -- (4) -- (8) -- (6) -- (2) (6) -- (5) -- (7) -- (8) (4) -- (3) -- (7);
				\draw[dashed] (3) -- (1) -- (2) (1) -- (5);

				\draw[ultra thick] (4) -- (3);

				\node[font=\Large] () at (-.1,1.1,1.1) {$\alpha$};
				\node[font=\Large] () at (-.1,1.1,-.1) {$\beta$};

				\draw (-.2,1.2,.5) edge (0,1,.5) (0,1,.5) edge[dashed] (1,0,.5) (1,0,.5) edge (1.2,-.2,.5);
			\end{tikzpicture}
			\caption{Repr\'esentation de l'ar\^ete $\overline{\alpha\beta}$ et de l'axe de la rotation $\rho$}
		\end{figure}
	\end{exem}
	\begin{nota}
		Nous noterons le groupe des rotation d'un cube comme $\overset{+}{\operatorname{Isom}}(\text{cube})$.
	\end{nota}
	\begin{prop}~

		$\overset{+}{\operatorname{Isom}}(\mathrm{cube})$ est isomorphe \`a $S_4$.
		\begin{subsubproof}{Id\'ee de la d\'emonstration}~

			On regarde l'action de $\overset{+}{\operatorname{Isom}}(\mathrm{cube})$ sur $E = \{\text{paires de sommets oppos\'es}\}$.

			Cette action d\'efinit un homomorphisme $\begin{array}{rcl}
				\overset{+}{\operatorname{Isom}}(\mathrm{cube})&\to&S_4
			\end{array}$.
			\begin{exem}~

				\begin{ulist}
					\item $\begin{pmatrix}
						1&2&3&4\\
						4&1&2&3
					\end{pmatrix} = \begin{pmatrix}
						1&4&3&2
					\end{pmatrix}$
					\item $\begin{pmatrix}
						1&2&3&4\\
						1&3&4&2
					\end{pmatrix} = \begin{pmatrix}
						2&3&4
					\end{pmatrix}$
					\item $\begin{pmatrix}
						1&2&3&4\\
						4&2&3&1
					\end{pmatrix} = \begin{pmatrix}
						1&4
					\end{pmatrix}$
				\end{ulist}
				\begin{figure}[h]
					\centering
					\begin{tikzpicture}[scale=1.5]
						%cube
						\coordinate (1) at (0,0,0);
						\coordinate (2) at (0,0,1);
						\coordinate (3) at (0,1,0);
						\coordinate (4) at (0,1,1);
						\coordinate (5) at (1,0,0);
						\coordinate (6) at (1,0,1);
						\coordinate (7) at (1,1,0);
						\coordinate (8) at (1,1,1);
						\draw (2) -- (4) -- (8) -- (6) -- (2) (6) -- (5) -- (7) -- (8) (4) -- (3) -- (7);
						\draw[dashed] (3) -- (1) -- (2) (1) -- (5);

						\node[anchor=south west] () at (1) {1};
						\node[anchor=north east] () at (2) {2};
						\node[anchor=south] () at (3) {3};
						\node[anchor=east] () at (4) {4};
						\node[anchor=north east] () at (8) {1};
						\node[anchor=west] () at (7) {2};
						\node[anchor=north] () at (6) {3};
						\node[anchor=west] () at (5) {4};

						\draw (.5,1.5,.5) edge (.5,1,.5) (.5,1,.5) edge[dashed] (.5,-.2,.5) (.5,-.2,.5) edge (.5,-.5,.5);
						\begin{scope}[canvas is xz plane at y=1]
							\fill (.5,.5) circle (1.5pt);
						\end{scope}
						\begin{scope}[canvas is xz plane at y=0]
							\fill (.5,.5) circle (1.5pt);
						\end{scope}
						\begin{scope}[canvas is xz plane at y=1.35]
							\draw[->] (.6,.5) arc(-90:180:.1);
						\end{scope}
					\end{tikzpicture}
					\quad$\longrightarrow$\quad
					\begin{tikzpicture}[scale=1.5]
						%cube
						\coordinate (1) at (0,0,0);
						\coordinate (2) at (0,0,1);
						\coordinate (3) at (0,1,0);
						\coordinate (4) at (0,1,1);
						\coordinate (5) at (1,0,0);
						\coordinate (6) at (1,0,1);
						\coordinate (7) at (1,1,0);
						\coordinate (8) at (1,1,1);
						\draw (2) -- (4) -- (8) -- (6) -- (2) (6) -- (5) -- (7) -- (8) (4) -- (3) -- (7);
						\draw[dashed] (3) -- (1) -- (2) (1) -- (5);

						\node[anchor=south west] () at (1) {2};
						\node[anchor=north east] () at (2) {3};
						\node[anchor=south] () at (3) {4};
						\node[anchor=east] () at (4) {1};
						\node[anchor=north east] () at (8) {2};
						\node[anchor=west] () at (7) {3};
						\node[anchor=north] () at (6) {4};
						\node[anchor=west] () at (5) {1};
					\end{tikzpicture}

					\begin{tikzpicture}[scale=1.5]
						%cube
						\coordinate (1) at (0,0,0);
						\coordinate (2) at (0,0,1);
						\coordinate (3) at (0,1,0);
						\coordinate (4) at (0,1,1);
						\coordinate (5) at (1,0,0);
						\coordinate (6) at (1,0,1);
						\coordinate (7) at (1,1,0);
						\coordinate (8) at (1,1,1);
						\draw (2) -- (4) -- (8) -- (6) -- (2) (6) -- (5) -- (7) -- (8) (4) -- (3) -- (7);
						\draw[dashed] (3) -- (1) -- (2) (1) -- (5);

						\node[anchor=south west] () at (1) {2};
						\node[anchor=north east] () at (2) {4};
						\node[anchor=south] () at (3) {3};
						\node[anchor=north east] () at (4) {1};
						\node[anchor=north east] () at (8) {2};
						\node[anchor=west] () at (7) {4};
						\node[anchor=north] () at (6) {3};
						\node[anchor=south west] () at (5) {1};

						\draw (-.25,1.25,1.25) edge (0,1,1) (0,1,1) edge[dashed] (1,0,0) (1,0,0) edge (1.25,-.25,-.25);
					\end{tikzpicture}
					\quad$\longrightarrow$\quad
					\begin{tikzpicture}[scale=1.5]
						%cube
						\coordinate (1) at (0,0,0);
						\coordinate (2) at (0,0,1);
						\coordinate (3) at (0,1,0);
						\coordinate (4) at (0,1,1);
						\coordinate (5) at (1,0,0);
						\coordinate (6) at (1,0,1);
						\coordinate (7) at (1,1,0);
						\coordinate (8) at (1,1,1);
						\draw (2) -- (4) -- (8) -- (6) -- (2) (6) -- (5) -- (7) -- (8) (4) -- (3) -- (7);
						\draw[dashed] (3) -- (1) -- (2) (1) -- (5);

						\node[anchor=south west] () at (1) {4};
						\node[anchor=north east] () at (2) {3};
						\node[anchor=south] () at (3) {2};
						\node[anchor=north east] () at (4) {1};
						\node[anchor=north east] () at (8) {4};
						\node[anchor=west] () at (7) {3};
						\node[anchor=north] () at (6) {2};
						\node[anchor=south west] () at (5) {1};
					\end{tikzpicture}

					\begin{tikzpicture}[scale=1.5]
						%cube
						\coordinate (1) at (0,0,0);
						\coordinate (2) at (0,0,1);
						\coordinate (3) at (0,1,0);
						\coordinate (4) at (0,1,1);
						\coordinate (5) at (1,0,0);
						\coordinate (6) at (1,0,1);
						\coordinate (7) at (1,1,0);
						\coordinate (8) at (1,1,1);
						\draw (2) -- (4) -- (8) -- (6) -- (2) (6) -- (5) -- (7) -- (8) (4) -- (3) -- (7);
						\draw[dashed] (3) -- (1) -- (2) (1) -- (5);

						\node[anchor=south west] () at (1) {2};
						\node[anchor=north east] () at (2) {3};
						\node[anchor=south] () at (3) {4};
						\node[anchor=east] () at (4) {1};
						\node[anchor=north east] () at (8) {2};
						\node[anchor=west] () at (7) {3};
						\node[anchor=north] () at (6) {4};
						\node[anchor=west] () at (5) {1};

						\draw (-.2,1.2,.5) edge (0,1,.5) (0,1,.5) edge[dashed] (1,0,.5) (1,0,.5) edge (1.2,-.2,.5);
					\end{tikzpicture}
					\quad$\longrightarrow$\quad
					\begin{tikzpicture}[scale=1.5]
						%cube
						\coordinate (1) at (0,0,0);
						\coordinate (2) at (0,0,1);
						\coordinate (3) at (0,1,0);
						\coordinate (4) at (0,1,1);
						\coordinate (5) at (1,0,0);
						\coordinate (6) at (1,0,1);
						\coordinate (7) at (1,1,0);
						\coordinate (8) at (1,1,1);
						\draw (2) -- (4) -- (8) -- (6) -- (2) (6) -- (5) -- (7) -- (8) (4) -- (3) -- (7);
						\draw[dashed] (3) -- (1) -- (2) (1) -- (5);

						\node[anchor=south west] () at (1) {2};
						\node[anchor=north east] () at (2) {3};
						\node[anchor=south] () at (3) {1};
						\node[anchor=east] () at (4) {4};
						\node[anchor=north east] () at (8) {2};
						\node[anchor=west] () at (7) {3};
						\node[anchor=north] () at (6) {1};
						\node[anchor=west] () at (5) {4};
					\end{tikzpicture}
					\caption{Repr\'esentation des rotations de l'exemple pr\'ec\'edent}
				\end{figure}
			\end{exem}

			Cet homomorphisme est surjectif et injectif, donc bijectif.
		\end{subsubproof}
	\end{prop}
	\begin{prop}~

		$\overset{+}{\operatorname{Isom}}(\mathrm{t\acute etrah\grave edre})$ est isomorphe \`a $A_4$, le sous-groupe de $S_4$ contenant seulement les permutations de signe positif.
	\end{prop}

	\cours
	\begin{rappel}~

		\begin{ulist}
			\item Orbite-stabilisateur
			\begin{ulist}
				\item $\card{G} = \card{\stab(x)} \cdot \card{G \bullet x}$
				\begin{ulist}
					\item id\'ee: $\begin{array}{rcl}
						\sfrac{G}{\stab(x)}&\to&G \bullet x\\
						a(\stab(x))&\mapsto&a \bullet x
					\end{array}$ est une bijection.
				\end{ulist}
			\end{ulist}
			\item $\overset{+}{\operatorname{Isom}}(\mathrm{cube}) \cong S_4$, paires de sommets oppos\'es
			\item $\overset{+}{\operatorname{Isom}}(\mathrm{t\acute etrah\grave edre}) \cong A_4$
		\end{ulist}
	\end{rappel}
	\begin{exem}~

		$G = \overset{+}{\operatorname{Isom}(\mathrm{t\acute etrah\grave edre})}$ agit sur les $4$ faces, avec $F = \{\text{faces}\}$. Il y a une seule orbite de taille $4$.

		$\stab(f_1) = \{\varepsilon, \theta, \phi\}$.

		$\card{\stab(f_1)}=3$, $\card{G} = \card{F} \cdot \card{\stab(f_1)} = 4 \times 3 = 12$.

		\begin{figure}[h]
			\centering
			\begin{tikzpicture}[scale=2]
				\coordinate (1) at (0,0,0);
				\coordinate (2) at (1,0,0);
				\coordinate (3) at (.5,0,.866);
				\coordinate (4) at (.5,.816,.289);
				\draw (4) edge (1) edge (2) edge (3) (1) edge[dashed] (2) (3) edge (1) edge (2);

				\begin{scope}[canvas is xz plane at y=0]
					\fill[fill opacity=.6, col1] (1) -- (2) -- (3) -- cycle;
					\fill (.5,.289) circle (1pt);
				\end{scope}

				\draw (.5,1,.289) edge (4) (4) edge[dashed] (.5,-.14,.289) (.5,-.14,.289) edge (.5,-.2,.289);

				\begin{scope}[canvas is xz plane at y=.9]
					\draw[->] (.5,.35) arc(90:270:.2);
				\end{scope}
			\end{tikzpicture}
			\hspace{2cm}
			\begin{tikzpicture}[scale=3]
				\coordinate (4) at (0,0);
				\coordinate (1) at (90:.433);
				\coordinate (2) at (210:.433);
				\coordinate (3) at (-30:.433);
				\draw (1) -- (2) -- (3) -- cycle;
				\draw (4) edge (1) edge (2) edge (3);

				\draw[->] (95:.5) arc(95:205:.5);
				\draw[->] (85:.5) arc(85:-25:.5);

				\node () at (150:.6) {$\theta$};
				\node () at (30:.6) {$\phi$};
			\end{tikzpicture}
			\caption{Repr\'esentation du t\'etrah\`edre, de l'axe de rotation et des rotations $\theta$ et $\phi$}
		\end{figure}
	\end{exem}
	\begin{exem}~

		Les cinq solides platoniques:
		\begin{multicols}{5}
			\begin{ulist}
				\item cube
				\item t\'etrah\`edre
				\item octah\`edre
				\item dod\'ecah\`edre
				\item icosah\`edre
			\end{ulist}
		\end{multicols}
		\begin{figure}[h]
			\centering
			\includegraphics[height=5cm]{figs/Platonic_Solids_Transparent.svg.png}
			\caption{Solides platoniques. Image tir\'ee de Wikipedia, \url{https://en.wikipedia.org/wiki/Platonic_solid}}
		\end{figure}
	\end{exem}
	\begin{defin}[Dualit\'e]~

		On met un sommet au milieu de chaque face et on relie les sommets \`a ceux associ\'es aux faces adjacentes.

		On associe les faces \`a des sommets, les ar\^etes \`a des ar\^etes, et les sommets \`a des faces.

		Le cube et l'octah\`edre sont duaux.

		Le t\'etrah\`edre est dual \`a lui-m\^eme.

		Le dod\'ecah\`edre et l'icosah\`edre sont duaux.

		\begin{figure}[h]
			\centering
			\begin{tikzpicture}[scale=2]
				%cube
				\coordinate (1) at (0,0,0);
				\coordinate (2) at (0,0,1);
				\coordinate (3) at (0,1,0);
				\coordinate (4) at (0,1,1);
				\coordinate (5) at (1,0,0);
				\coordinate (6) at (1,0,1);
				\coordinate (7) at (1,1,0);
				\coordinate (8) at (1,1,1);
				\draw (2) -- (4) -- (8) -- (6) -- (2) (6) -- (5) -- (7) -- (8) (4) -- (3) -- (7);
				\draw[dashed] (3) -- (1) -- (2) (1) -- (5);

				%octahedre
				\coordinate (a) at (.5,.5,0);
				\coordinate (b) at (0,.5,.5);
				\coordinate (c) at (.5,0,.5);
				\coordinate (d) at (.5,.5,1);
				\coordinate (e) at (1,.5,.5);
				\coordinate (f) at (.5,1,.5);
				\begin{scope}[red]
					\draw[dashed] (a) edge (b) edge (f) edge (e) edge (c);
					\draw (d) edge (b) edge (f) edge (e) edge (c);
					\draw (c) -- (e) -- (f) -- (b) -- cycle;
				\end{scope}
			\end{tikzpicture}
			\caption{Repr\'esentation de la dualit\'e du cube et de l'octah\`edre}
		\end{figure}
	\end{defin}
	\begin{prop}~

		$\overset{+}{\operatorname{Isom}}(\mathrm{dod\acute ecah\grave edre}) \cong \overset{+}{\operatorname{Isom}}(\mathrm{icosah\grave edre}) \cong A_5$, qui contient $60$ \'el\'ements.

		Les sommets d'un icosah\`edre sont les m\^emes que $5$ cubes, donc chaque rotation permute ces $5$ cubes.
	\end{prop}
	\begin{exem}[Combinaisons]
		Le nombre de mani\`eres de choisir $k$ \'el\'ements parmi $n$ sans remise.

		On veut compter les sous-ensembles de taille $k$ dans un ensemble de taille $n$.

		Prenons $\{1,2,3,4\}$, $n=4, k=2$. On a
		\[
		\{1,2\}; \{1,3\}; \{1,4\}; \{2,3\}; \{2,4\}; \{3,4\} \Rightarrow \binom{4}{2}=6
		\]

		Le groupe $S_n$ agit sur $E_{n,k}$ des sous-ensembles de taille $k$ dans $\{1,2,\dots,n\}$.

		Par exemple
		\[
		\begin{pmatrix}
			1&2&3
		\end{pmatrix} \bullet \{1,4\} = \left\{\begin{pmatrix}
			1&2&3
		\end{pmatrix} 1, \begin{pmatrix}
			1&2&3
		\end{pmatrix} 4\right\} = \{2,4\}
		\]

		Tout ensemble de $k$ \'el\'ements peut \^etre envoy\'e \`a tout autre par une permutation. En effet,
		\[
		\sigma = \begin{pmatrix}
			1&2&3&\dotsb&k&k+1&\dotsb&n\\
			a_1&a_2&a_3&\dotsb&a_k
		\end{pmatrix}
		\]
		avec n'importe quelle combinaison des autres \'el\'ements pour compl\'eter $\sigma$.

		$\sigma \bullet \{1,2,\dots,k\} = \{a_1,a_2,\dots,a_k\}$, donc l'orbite de $\{1,2,\dots,k\}$ est $E_{n,k}$. Il n'y a qu'une seule orbite.

		De plus, $\stab(\{1,2,\dots,k\}) \cong S_k \times S_{n-k}$ en permutant les $k$ premiers \'el\'ements, puis les $n-k$ derniers \'el\'ements.

		Donc, orbite-stabilisateur nous dit $\card{S_n} = \card{E_{n,k}} \cdot \card{\stab(\{1,2,\dots,k\})}$, c'est-\`a-dire, $n! = \card{E_{n,k}} \cdot (k!)(n-k)!$, d'o\`u $\card{E_{n,k}} = \dfrac{n!}{k!(n-k)!}$.
	\end{exem}
	\begin{exem}[motivation du prochain th\'eor\`eme]~

		Combien y a-t-il de fa\c cons de colorer les faces d'un cube avec deux couleurs, \`a sym\'etrie pr\`es?
		\begin{nlist}
			\item Toutes les faces de la m\^eme couleur,

			$2$ possibilit\'es;
			\item Une face d'une couleur et cinq de l'autre,

			$2$ possibilit\'es;
			\item Deux faces d'une couleur et quatre de l'autre

			$4$ possibilit\'es;
			\item Trois faces de chaque couleur

			$2$ possibilit\'es.
		\end{nlist}

		Donc, au total, $2 + 2 + 4 + 2 = 10$ coloriages possibles.
	\end{exem}
	\begin{exem}[autre motivation]~

		Combien y a-t-il de fa\c cons de colorer les sommets d'un hexagone avec un certain nombre de couleurs, \`a sym\'etrie pr\`es?
	\end{exem}

	\cours
	\begin{rappel}~

		\begin{ulist}
			\item $\overset{+}{\operatorname{Isom}}(\mathrm{cube}) \cong \overset{+}{\operatorname{Isom}}(\mathrm{octah\grave edre}) \cong S_4$, ordre $24$;
			\item $\overset{+}{\operatorname{Isom}}(\mathrm{t\acute etrah\grave edre}) \cong A_4$, ordre $12$;
			\item $\overset{+}{\operatorname{Isom}}(\mathrm{icosah\grave edre}) \cong \overset{+}{\operatorname{Isom}}(\mathrm{dod\acute ecah\grave edre}) \cong A_5$, ordre $60$.
		\end{ulist}
	\end{rappel}
	\begin{nota}
		$\sfrac{E}{G} = \{G \bullet x \mid x \in E\}$ est l'ensemble des orbites.
	\end{nota}
	\begin{exem}
		Combien de fa\c cons de colorer les faces d'un cube avec $k$ couleurs.

		$G = \overset{+}{\operatorname{Isom}}(\mathrm{cube})$, $E = \{\text{cubes color\'es}\}$.

		On veut compter $\card{\sfrac{E}{G}}$.
	\end{exem}
	\begin{nota}
		$E^g = \{x \in E \mid g \bullet x = x\}$ est l'ensemble des \'el\'ements de $E$ \emph{invariants} par $g \in G$.
	\end{nota}
	\begin{exem}
		Avec $g$ une rotation de $90\degree$ autour d'un axe passant par le centre de deux faces oppos\'ees.

		Dans $E^g$, les quatre faces des c\^ot\'es sont de la m\^eme couleur et les deux autres faces sont libres.
	\end{exem}
	\begin{exem}
		$G = \mathbb{D}_3$ qui agit sur $E = \{\text{les sommets d'un triangle}\}$.
		\begin{align*}
			E^\rho&= \emptyset\\
			E^\varepsilon&= E\\
			E^\alpha&= \{1\}
		\end{align*}
	\end{exem}
	\phantomsection\section*{Lemme de Burnside}\addcontentsline{toc}{section}{Lemme de Burnside}
	\begin{thm}[Lemme de Burnside]
		\begin{equation*}
			\card{\sfrac{E}{G}} = \dfrac{1}{\card{G}} \displaystyle\sum_{g \in G} \card{E^g}
		\end{equation*}
		\begin{proof}~

			Consid\'erons $X = \{(g,x) \in G \times E \mid g \bullet x = x\}$.

			Comptons le nombre d'\'el\'ements de $X$ de deux mani\`eres diff\'erentes.
			\begin{nlist}
				\item On fixe $g \in G$. On a $\card{E^g}$ \'el\'ements $x \in E$ t.q. $g \bullet x = x$. Donc
				\begin{equation*}
					\card{X} = \sum_{g \in G} \card{E^g}
				\end{equation*}
				\item On fixe $x \in E$. Les \'el\'ements $g \in G$ t.q. g $\bullet$ x = x sont exactement ceux de $\stab(x)$. Donc
				\begin{equation*}
					\card{X} = \sum_{x \in E} \card{\stab(x)}\tag{$*$}
				\end{equation*}

				Si deux $x,y \in E$ sont dans une m\^eme orbite, $\card{\stab(x)} = \card{\stab(y)}$.

				Soit $x_1, x_2, \dots, x_m$ un ensemble de repr\'esentants des orbites, avec $m = \card{\sfrac{E}{G}}$.

				On a donc
				\begin{align*}
					\card{X}&= \sum_{i=1}^{m} \card{\stab(x_i)} \card{G \bullet x_i}\\
					\shortintertext{du th\'eor\`eme orbite-stabilisateur}
					&= \sum_{i=1}^{m} \card{G}\\
					&= m \cdot \card{G}\tag{$**$}
				\end{align*}
			\end{nlist}

			Comme $(*)=(**)$, on a
			\begin{align*}
				m \card{G}&= \sum_{g \in G} \card{E^g}\\
				\card{\sfrac{E}{G}}&= \dfrac{1}{\card{G}} \sum_{g \in G} \card{E^g}
			\end{align*}
		\end{proof}
	\end{thm}
	\begin{exem}
		Combien peut-on faire de colliers \`a six billes avec $k$ sortes de billes?

		$G = \mathbb{D}_6$, $E = \{\text{colliers \`a six billes  \emph{fixes}}\}$.

		$\card{G} = 12$.
		\begin{align*}
			\card{\sfrac{E}{G}}&= \dfrac{1}{12} \left( \underbrace{k^6}_{\varepsilon} + \underbrace{2k}_{\pm\rho} + \underbrace{2k^2}_{\pm\rho^2} + \underbrace{k^3}_{\rho^3} + \underbrace{3k^3}_{\alpha} + \underbrace{3k^4}_{\beta} \right)\\
			&= \dfrac{1}{12} \left( k^6 + 2k + 2k^2 + 4k^3 + 3k^4 \right)
		\end{align*}
		\begin{figure}[h]
			\centering
			\begin{tikzpicture}[scale=1.5]
				\coordinate (e) at (0:1);
				\coordinate (ne) at (60:1);
				\coordinate (nw) at (120:1);
				\coordinate (w) at (180:1);
				\coordinate (sw) at (240:1);
				\coordinate (se) at (300:1);
				\draw (e) -- (ne) -- (nw) -- (w) -- (sw) -- (se) -- cycle;

				\fill (e) circle (2pt);
				\fill (ne) circle (2pt);
				\fill (nw) circle (2pt);
				\fill (w) circle (2pt);
				\fill (sw) circle (2pt);
				\fill (se) circle (2pt);
			\end{tikzpicture}
			\qquad
			\begin{tikzpicture}[scale=1.5]
				\coordinate (e) at (0:1);
				\coordinate (ne) at (60:1);
				\coordinate (nw) at (120:1);
				\coordinate (w) at (180:1);
				\coordinate (sw) at (240:1);
				\coordinate (se) at (300:1);
				\draw (e) -- (ne) -- (nw) -- (w) -- (sw) -- (se) -- cycle;

				\fill (e) circle (2pt);
				\fill (ne) circle (2pt);
				\fill (nw) circle (2pt);
				\fill (w) circle (2pt);
				\fill (sw) circle (2pt);
				\fill (se) circle (2pt);

				\draw[->] (120:1.4) arc(120:60:1.4);
				\draw[->] (120:1.2) arc(120:0:1.2);
				\node () at (90:1.6) {$\rho$};
				\node () at (30:1.4) {$\rho^2$};
			\end{tikzpicture}
			\qquad
			\begin{tikzpicture}[scale=1.5]
				\coordinate (e) at (0:1);
				\coordinate (ne) at (60:1);
				\coordinate (nw) at (120:1);
				\coordinate (w) at (180:1);
				\coordinate (sw) at (240:1);
				\coordinate (se) at (300:1);
				\draw (e) -- (ne) -- (nw) -- (w) -- (sw) -- (se) -- cycle;

				\fill (e) circle (2pt);
				\fill (ne) circle (2pt);
				\fill (nw) circle (2pt);
				\fill (w) circle (2pt);
				\fill (sw) circle (2pt);
				\fill (se) circle (2pt);

				\draw (90:1) -- (270:1) (30:1) -- (210:1) (330:1) -- (150:1);
				\node () at (90:1.2) {$\alpha_1$};
				\node () at (30:1.2) {$\alpha_2$};
				\node () at (330:1.2) {$\alpha_3$};
			\end{tikzpicture}
			\qquad
			\begin{tikzpicture}[scale=1.5]
				\coordinate (e) at (0:1);
				\coordinate (ne) at (60:1);
				\coordinate (nw) at (120:1);
				\coordinate (w) at (180:1);
				\coordinate (sw) at (240:1);
				\coordinate (se) at (300:1);
				\draw (e) -- (ne) -- (nw) -- (w) -- (sw) -- (se) -- cycle;

				\fill (e) circle (2pt);
				\fill (ne) circle (2pt);
				\fill (nw) circle (2pt);
				\fill (w) circle (2pt);
				\fill (sw) circle (2pt);
				\fill (se) circle (2pt);

				\draw (120:1.2) -- (300:1.2) (0:1.2) -- (180:1.2) (60:1.2) -- (240:1.2);
				\node[anchor=west] () at (300:1.2) {$\beta_1$};
				\node[anchor=west] () at (0:1.2) {$\beta_2$};
				\node[anchor=west] () at (60:1.2) {$\beta_3$};
			\end{tikzpicture}
			\caption{Repr\'esentation du collier et de ses sym\'etries}
		\end{figure}
	\end{exem}
	\begin{exem}
		Combien de coloriages possibles pour les faces d'un cube avec $k$ couleurs?

		$G = \overset{+}{\operatorname{Isom}}(\mathrm{cube}) \cong S_4$, $E = \{\text{coloriages d'un cube \emph{fixe}}\}$.
		\begin{align*}
			\card{\sfrac{E}{G}}&= \dfrac{1}{24} \left( \underbrace{k^6}_{e} + \underbrace{6k^3}_{\alpha} + \underbrace{3k^4}_{\beta} + \underbrace{8k^2}_{\gamma} + \underbrace{6k^3}_{\delta} \right)\\
			&= \dfrac{1}{24} \left( k^6 + 8k^2 + 12k^3 + 3k^4 \right)
		\end{align*}
		\begin{figure}[h]
			\centering
			\begin{tikzpicture}[scale=1.5]
				%cube
				\coordinate (1) at (0,0,0);
				\coordinate (2) at (0,0,1);
				\coordinate (3) at (0,1,0);
				\coordinate (4) at (0,1,1);
				\coordinate (5) at (1,0,0);
				\coordinate (6) at (1,0,1);
				\coordinate (7) at (1,1,0);
				\coordinate (8) at (1,1,1);
				\draw (2) -- (4) -- (8) -- (6) -- (2) (6) -- (5) -- (7) -- (8) (4) -- (3) -- (7);
				\draw[dashed] (3) -- (1) -- (2) (1) -- (5);
			\end{tikzpicture}
			\qquad
			%a:90 faces
			\begin{tikzpicture}[scale=1.5]
				\coordinate (1) at (0,0,0);
				\coordinate (2) at (0,0,1);
				\coordinate (3) at (0,1,0);
				\coordinate (4) at (0,1,1);
				\coordinate (5) at (1,0,0);
				\coordinate (6) at (1,0,1);
				\coordinate (7) at (1,1,0);
				\coordinate (8) at (1,1,1);
				\draw (2) -- (4) -- (8) -- (6) -- (2) (6) -- (5) -- (7) -- (8) (4) -- (3) -- (7);
				\draw[dashed] (3) -- (1) -- (2) (1) -- (5);
				\draw (.5,1.5,.5) edge (.5,1,.5) (.5,1,.5) edge[dashed] (.5,-.2,.5) (.5,-.2,.5) edge (.5,-.5,.5);
				\begin{scope}[canvas is xz plane at y=1]
					\fill (.5,.5) circle (1.5pt);
				\end{scope}
				\begin{scope}[canvas is xz plane at y=0]
					\fill (.5,.5) circle (1.5pt);
				\end{scope}
				\begin{scope}[canvas is xz plane at y=1.35]
					\draw[->] (.5,.5)++(240:.1) arc(240:0:.1);
				\end{scope}

				\node () at (.3,-.2,1) {$\alpha:90\degree$};
				\node () at (.3,-.5,1) {$\beta:180\degree$};
			\end{tikzpicture}
			%b:180 faces
			\qquad
			%c:120 sommets
			\begin{tikzpicture}[scale=1.5]
				%cube
				\coordinate (1) at (0,0,0);
				\coordinate (2) at (0,0,1);
				\coordinate (3) at (0,1,0);
				\coordinate (4) at (0,1,1);
				\coordinate (5) at (1,0,0);
				\coordinate (6) at (1,0,1);
				\coordinate (7) at (1,1,0);
				\coordinate (8) at (1,1,1);
				\draw (2) -- (4) -- (8) -- (6) -- (2) (6) -- (5) -- (7) -- (8) (4) -- (3) -- (7);
				\draw[dashed] (3) -- (1) -- (2) (1) -- (5);

				\draw (-.25,1.25,1.25) edge (0,1,1) (0,1,1) edge[dashed] (1,0,0) (1,0,0) edge (1.25,-.25,-.25);

				\node[anchor=north west] () at (1,0,.5) {$\gamma:120\degree$};
			\end{tikzpicture}
			\qquad
			%d:180 aretes
			\begin{tikzpicture}[scale=1.5]
				%cube
				\coordinate (1) at (0,0,0);
				\coordinate (2) at (0,0,1);
				\coordinate (3) at (0,1,0);
				\coordinate (4) at (0,1,1);
				\coordinate (5) at (1,0,0);
				\coordinate (6) at (1,0,1);
				\coordinate (7) at (1,1,0);
				\coordinate (8) at (1,1,1);
				\draw (2) -- (4) -- (8) -- (6) -- (2) (6) -- (5) -- (7) -- (8) (4) -- (3) -- (7);
				\draw[dashed] (3) -- (1) -- (2) (1) -- (5);

				\draw (-.2,1.2,.5) edge (0,1,.5) (0,1,.5) edge[dashed] (1,0,.5) (1,0,.5) edge (1.2,-.2,.5);

				\node[anchor=north west] () at (1,0,0) {$\delta:180\degree$};
			\end{tikzpicture}
			\caption{Repr\'esentation du cube et de ses rotations}
		\end{figure}
	\end{exem}
	\begin{exem}
		Combien de mani\`eres y a-t-il de colorer les quatre sommets d'un t\'etra\`edre avec $k$ couleurs?

		$G = \overset{+}{\operatorname{Isom}}(\mathrm{t\acute etrah\grave edre}) \cong A_4$, $E = \{\text{coloriages d'un t\'etra\`edre \emph{fixe}}\}$.
		\begin{align*}
			\card{\sfrac{E}{G}}&= \dfrac{1}{12} \left( \underbrace{k^4}_{e} + \underbrace{8k^2}_{\alpha} + \underbrace{3k^2}_{\beta} \right)\\
			&= \dfrac{1}{12} \left( k^4 + 11k^2 \right)
		\end{align*}
		\begin{figure}[h]
			\centering
			\begin{tikzpicture}[scale=2]
				\coordinate (1) at (0,0,0);
				\coordinate (2) at (1,0,0);
				\coordinate (3) at (.5,0,.866);
				\coordinate (4) at (.5,.816,.289);
				\draw (4) edge (1) edge (2) edge (3) (1) edge[dashed] (2) (3) edge (1) edge (2);
			\end{tikzpicture}
			\hspace{3cm}
			\begin{tikzpicture}[scale=2]
				\coordinate (1) at (0,0,0);
				\coordinate (2) at (1,0,0);
				\coordinate (3) at (.5,0,.866);
				\coordinate (4) at (.5,.816,.289);
				\draw (4) edge (1) edge (2) edge (3) (1) edge[dashed] (2) (3) edge (1) edge (2);

				\begin{scope}[canvas is xz plane at y=0]
					\fill (.5,.289) circle (1pt);
				\end{scope}

				\draw (.5,1,.289) edge (4) (4) edge[dashed] (.5,-.14,.289) (.5,-.14,.289) edge (.5,-.2,.289);

				\begin{scope}[canvas is xz plane at y=.9]
					\draw[->] (.5,.289)++(90:.2) arc(90:290:.2);
				\end{scope}

				\node[anchor=west] () at (.6,-.2,.289) {$\alpha:120\degre$};
			\end{tikzpicture}
			\hspace{3cm}
			%a:120 sommet-face
			\begin{tikzpicture}[scale=2]
				\coordinate (1) at (0,0,0);
				\coordinate (2) at (1,0,0);
				\coordinate (3) at (.5,0,.866);
				\coordinate (4) at (.5,.816,.289);
				\draw (4) edge (1) edge (2) edge (3) (1) edge[dashed] (2) (3) edge (1) edge (2);

				\coordinate (a) at ($(1)!.5!(3)$);
				\coordinate (b) at ($(2)!.5!(4)$);
				\coordinate (c) at ($(a)!-.2!(b)$);
				\coordinate (d) at ($(a)!1.2!(b)$);

				\draw (c) edge (a) (a) edge[dashed] (b) (b) edge (d);

				\node[anchor=north west] () at (d) {$\beta:180\degree$};
			\end{tikzpicture}
			%b:180 aretes
			\caption{Repr\'esentation du t\'etrah\`edre et de ses rotations}
		\end{figure}
	\end{exem}
\end{document}
